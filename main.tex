\documentclass{article}

\usepackage{luatexja}

\usepackage{amsmath, amssymb, amsthm}
\usepackage{bm}
\usepackage{mathtools}
\usepackage{ytableau}

\theoremstyle{definition}
\newtheorem{ans}{}
\numberwithin{ans}{subsection}

\newcommand{\fakesection}[1]{%
  \par\refstepcounter{section}% Increase section counter
  \sectionmark{#1}% Add section mark (header)
  \addcontentsline{toc}{section}{\protect\numberline{\thesection}#1}% Add section to ToC
  % Add more content here, if needed.
}
\newcommand{\fakesubsection}[1]{%
  \par\refstepcounter{subsection}% Increase subsection counter
  \subsectionmark{#1}% Add subsection mark (header)
  \addcontentsline{toc}{subsection}{\protect\numberline{\thesubsection}#1}% Add subsection to ToC
  % Add more content here, if needed.
}

\DeclareMathOperator{\id}{id}
\DeclareMathOperator{\tr}{tr}

\newcommand{\transpose}[1]{{\prescript{t}{}{#1}}}
\newcommand{\Sp}[1]{\mathrm{Sp}(#1)}
\newcommand{\U}[1]{\mathrm{U}(#1)}
\newcommand{\SO}[1]{\mathrm{SO}(#1)}
\newcommand{\Z}[1]{\mathbb{Z}/#1\mathbb{Z}}
\newcommand{\Ad}[1]{\mathrm{Ad}(#1)}
\newcommand{\Ker}[1]{\mathrm{Ker}(#1)}

\DeclarePairedDelimiter{\gen}{\langle}{\rangle}
\DeclarePairedDelimiter{\order}{\lvert}{\rvert}
\DeclarePairedDelimiter{\abs}{\lvert}{\rvert}
\DeclarePairedDelimiter{\norm}{\lVert}{\rVert}
\DeclarePairedDelimiterX{\genrel}[2]{\langle}{\rangle}{#1 \mid #2}
\DeclarePairedDelimiter\floor{\lfloor}{\rfloor}

\begin{document}

\noindent(第2版第1刷をもとにしている.)

\fakesection{第1章 集合論}
\fakesubsection{1.1 集合と論理の復習}

\begin{ans}
  $f: g,\ A: X,\ B: X$
\end{ans}

\begin{ans}
  (1) $f(S) = \{3, 4\}$
  (2) $f^{-1}(S_1) = \emptyset,\ f^{-1}(S_2) = \{1, 3, 4, 5\}$
  (3) $f(a) = 2 (\in B)$であるような$a \in A$が存在しないので, 全射ではない.
  (4) $f(1) = f(4)$なので, 単射ではない.
\end{ans}

\begin{ans}
  写像$f: A \rightarrow B$について考える.
  全射: $B$のすべての要素が集合$A$から来ている.\\
  単射: $A$の異なる要素は$B$の異なる要素に行く.
\end{ans}

\begin{ans}
  全射: $f(x) = x,\ x\sin x,\ x^3 - x$
  単射: $f(x) = x,\ e^x,\ \arctan x$
\end{ans}

\begin{ans}
  (逆写像$\Rightarrow$全単射) $f \circ g = \id_B$が全射なので, 次問より, $f$は全射.
  同様に$g \circ f = \id_A$が単射なので, $f$は単射.\\
  (全単射$\Rightarrow$逆写像)
  $b \in B$を任意に取ると, $f$の全射性から$f(a) = b$なる$a \in A$がある.
  しかも$f$の単射性により, このような$a$は一意である.
  そこで$g: B \rightarrow A$を$g(b) = a$と定めれば,
  $f \circ g(b) = f(a) = b$, $g \circ f(a) = g(b) = a$
  が成り立つので, $f \circ g = \id_B$, $g \circ f = \id_A$.
\end{ans}

\begin{ans}
  (1) $c \in C$を任意に取る.
  $g$の全射性から$g(b) = c$となる$b \in B$があり,
  さらに$f$の全射性から$f(a) = b$となる$a \in A$がある.
  すなわち$g \circ f(a) = c$となる$a \in A$がある.
  (2) $g \circ f(a) = g \circ f(a^\prime)$であるとする.
  このとき$g$の単射性から$f(a) = f(a^\prime)$,
  さらに$f$の単射性から$a = a^\prime$.
  (3) $g \circ f$が全射なので, 任意の$c \in C$に対して
  $g \circ f(a) = c$となる$a \in A$がある.
  すなわち$g(f(a)) = c$となる$f(a) \in B$がある.
  (4) $f(a) = f(a^\prime)$であるとする.
  このとき$g \circ f(a) = g \circ f(a^\prime)$であり,
  $g \circ f$の単射性より$a = a^\prime$.
\end{ans}

\begin{ans}
  (全射$\Rightarrow \forall S\ f(f^{-1}(S)) = S$).
  まず$f(f^{-1}(S)) \subset S$を示す.
  $b \in f(f^{-1}(S))$ならば,
  $f(a) = b,\ a \in f^{-1}(S)$となる$a$が存在する.
  逆像の定義から$f(a) \in S$なので, $b \in S$.
  (注意: ここでは$f$の全射性は用いなかった.)
  次に$f(f^{-1}(S)) \supset S$を示す.
  $b \in S$であるとすると, $f$の全射性から$f(a) = b$となる$a \in A$がある.
  $a \in f^{-1}(S)$なので, $b \in f(f^{-1}(S))$.
  \\
  ($\forall S\ f(f^{-1}(S)) = S \Rightarrow$全射).
  任意の$b \in B$に対して$\{b\} \subset f(f^{-1}(\{b\})) \subset f(A)$.
\end{ans}

\begin{ans}
  前問で示したように, $f(f^{-1}(S)) \subset S$なので,
  $f^{-1}(f(f^{-1}(S))) \subset f^{-1}(S)$.
  逆に$f^{-1}(f(f^{-1}(S))) \supset f^{-1}(S)$は次のようにして示される.
  $a \in f^{-1}(S)$ならば, $f(a) = b,\ b \in S$となる$b$が存在する.
  ふたたび$a \in f^{-1}(S)$より, $b \in f(f^{-1}(S))$.
  さらに$f(a) = b$より, $a \in f^{-1}(f(f^{-1}(S)))$.
\end{ans}

\begin{ans}
  (1) $x = 4.1$
  (2) $A = \mathbb{N},\ B = \emptyset$
  (3) $f: \{0, 1, 2\} \rightarrow \{0, 1\}$を
  $f(0) = f(2) = 0,\ f(1) = 1$と定めて,
  $S_0 = \{0, 1\},\ S_1 = \{1, 2\}$
  とすると, $f(S_0 \cap S_1) = f(\{1\}) = \{1\}$,
  $f(S_0) \cap f(S_1) = \{0, 1\} \cap \{0, 1\} = \{0, 1\}$.
\end{ans}

\begin{ans}
  (1) (d).
  $x = 4$が$A \Rightarrow B$の反例,
  $x = 1$が$B \Rightarrow A$の反例である.
  (2) (a).
  $B \Rightarrow A$は真だが,
  $x = 4$が$A \Rightarrow B$の反例である.
  (3) (b). $A \Rightarrow B$は真だが,
  $X = \mathbb{N},\ Y = \emptyset$が$B \Rightarrow A$の反例である
\end{ans}

\begin{ans}
  (1) $A$が成り立たないか$B$が成り立たない, かつ$C$が成り立たない
  (2) $A$が成り立ち, かつ$B$も$C$も成り立たない
  (3) $A$か$B$の一方のみが成り立つ
  (4) 自然数$n$があって, 任意の実数$x$に対して$x \le 0$または$\frac{1}{n} \le x$が成り立つ
  (5) ある$\varepsilon > 0$について, 任意の$\delta > 0$に対して$x, y \in [0, 1]$で
  $\abs{x - y} < \delta$かつ$\abs{f(x) - f(y)} \ge \varepsilon$
  を満たすものがある.
\end{ans}

\begin{ans}
  (1) $4 + 5 = 9 \ge 3$なので, 関係$R$がある.
  (2) $1 + (-1) = 0 < 3$なので, 関係$R$はない.
\end{ans}

\begin{ans}
  (1) $X = \mathbb{R},\ R = \{(x, y) \mid y = x^2\}$\\
  (2) $X = \mathbb{Z},\ R = \{(x, y) \mid x\text{は}y\text{で割り切れる.}\}$,
  この$R$は順序である.\\
  (3) $X = \mathbb{R}^n\setminus\{\bm{0}\},\ R = \{(\bm{x}, \bm{y}) \mid \exists \alpha \in \mathbb{R}\setminus\{0\},\ \bm{y} = \alpha \bm{x} \}$,
  この$R$は同値関係 (p47) である.
\end{ans}

\fakesubsection{1.2 well-definedと自然な対象}

\begin{ans}
  線形写像$f:V \rightarrow V$に対してトレース$\tr(f)$を
  $f$の行列表示$M = (m_{ij}) \in M_n(\mathbb{R})$を$1$つとって
  $\tr(f) = \tr(M) = \sum_{i = 1}^n m_{ii}$と定義したいとする.
  このとき, $\tr(f)$がwell-definedであること,
  すなわち行列表示のとり方によらず$\tr(f)$が定まるかどうかが問題となる.
  これがwell-definedであることは, 行列のトレースについて$\tr(AB) = \tr(BA)$が成り立つことを用いて,
  別の基底での$f$の行列表示$P^{-1}MP$について
  $\tr(P^{-1}MP) = \tr(MPP^{-1}) = \tr(M)$
  であることからしたがう.
\end{ans}

\begin{ans}
  解答例にないものでは, 双対空間$V^\ast$, テンソル積$V \otimes V$など.
\end{ans}

\fakesubsection{1.3 値が写像である写像}
\fakesubsection{1.4 選択公理とツォルンの補題}

\begin{ans}
  (1) $X$の全順序部分集合を任意に取り,
  適当に添え字をつけて$\{(S_\lambda, f_\lambda)\}_{\lambda \in \Lambda}$
  と書くことにする.
  (上界の候補として) $S = \bigcup_{\lambda \in \Lambda} S_\lambda$と定めて,
  写像$f: S \rightarrow B$を,
  任意の$x \in S$に対して$x \in S_\lambda$なる$\lambda$を$1$つとって$f(x) = f_\lambda(x)$
  として定義したい. まずこれがwell-definedであることを見るために,
  別の$\lambda^\prime$に対して$x \in S_\lambda^\prime$であるとする.
  このとき$\{(S_\lambda, f_\lambda)\}_{\lambda \in \Lambda}$が全順序であることから,
  「$S_\lambda \subset S_{\lambda^\prime}$かつ$f_{\lambda^\prime}$は$f_\lambda$の拡張」(またはここで$\lambda$と$\lambda^\prime$を入れ替えたもの)
  が成り立つ. よって$f_\lambda(x) = f_{\lambda^\prime}(x)$なので,
  $f$は$\lambda$のとり方によらず, well-definedである.
  つぎに$f$が単射であることをみる.
  $f(x) = f(y)$とすると, ある$\lambda$, $\lambda^{\prime}$に対して
  $f_\lambda(x) = f_{\lambda^\prime}(y)$である.
  $\{(S_\lambda, f_\lambda)\}_{\lambda \in \Lambda}$が全順序であることから
  一方は他方の拡張であり, $f_\lambda(x) = f_\lambda(y)$または$f_{\lambda^\prime}(x) = f_{\lambda^\prime}(y)$.
  いずれの場合も$f_\lambda, f_{\lambda^\prime}$が単射であることから$x = y$.
  したがって$f$は単射である.
  さらに$(S, f)$が$\{(S_\lambda, f_\lambda)\}_{\lambda \in \Lambda}$の上界であることをみる.
  任意の$\lambda \in \Lambda$に対して
  $S_\lambda \subset S$であり, また$x \in S_\lambda$ならば$f(x) = f_\lambda(x)$が成り立つ.
  よって$(S_\lambda, f_\lambda) \le (S, f)$なので, $(S, f)$は上界である.
  以上により, $X$の任意の全順序部分集合が上界を持つことが示せたから,
  ツォルンの補題が適用でき, $X$に極大元が存在することがわかる.\\
  (2) $S_0 \neq A$かつ$f_0(S_0) \neq B$であると仮定して矛盾を導く.
  $a \in A \setminus S_0$と$b \in B \setminus f_0(S_0)$をとり,
  $f: S_0 \cup \{a\} \rightarrow B$を,
  $S_0$上では$f = f_0$, $f(a) = b$と定めると,
  $(S_0 \cup \{a\}, f) \in X$かつ$(S_0, f_0) \leq (S_0 \cup \{a\}, f)$かつ$(S_0, f_0) \neq (S_0 \cup \{a\}, f)$となって,
  $(S_0, f_0)$が極大元であることと矛盾する.
  したがって$S_0 = A$または$f_0(S_0) = B$.
\end{ans}

\fakesection{第2章 群の基本}
\fakesubsection{2.1 群の定義}

\begin{ans}
  $1$が単位元だが$0$の逆元が存在しないので群ではない.
  (単位元の存在に加えて結合法則が成り立つのでモノイドである.)
\end{ans}

\begin{ans}
  $0$が単位元なので,
  $a \in \mathbb{R}$の逆元が$b \in \mathbb{R}$であるならば,
  $a + b + ab = 0$より$(1 + a)b = -a$が成り立つはずである.
  ところが$a = -1$のときこの等式はどんな$b$に対しても成り立たないから,
  $-1$の逆元は存在しない.
  したがってこの演算では群にならない.
  (結合法則は成り立つのでモノイドである.)
\end{ans}

\begin{ans}
  略.
\end{ans}

\begin{ans}
  $((ab)c)d = (a(bc))d = a((bc)d)$.
\end{ans}

\begin{ans}
  両辺に左から$a^{-1}b^{-1}$, 右から$d^{-1}$を掛けると
  $c^{-1} = a^{-1}b^{-1}ab$.
  この逆元をとって$c = b^{-1}a^{-1}ba$.
\end{ans}

\begin{ans}
  (1) $\begin{pmatrix}
    4 & 1 & 2 & 3 \\
    1 & 2 & 3 & 4
  \end{pmatrix} =
  \begin{pmatrix}
    1 & 2 & 3 & 4 \\
    2 & 3 & 4 & 1
  \end{pmatrix} = (1\ 2\ 3\ 4)$ \\
  (2) $(2\ 4)^{-1}(1\ 3)^{-1} = (2\ 4)(1\ 3)$ \\
  (3) $\begin{pmatrix}
    1 & 3 & 4 & 2 \\
    4 & 2 & 3 & 1
  \end{pmatrix}\begin{pmatrix}
    1 & 2 & 3 & 4 \\
    1 & 3 & 4 & 2
  \end{pmatrix} = \begin{pmatrix}
    1 & 2 & 3 & 4 \\
    4 & 2 & 3 & 1
  \end{pmatrix} = (1\ 4)$ \\
  (4) $(2\ 4)(1\ 3)(1\ 3) = (2\ 4)$ \\
  (5) $\begin{pmatrix}
    1 & 2 & 3 & 4 \\
    1 & 3 & 4 & 2
  \end{pmatrix}\begin{pmatrix}
    1 & 2 & 3 & 4 \\
    4 & 1 & 2 & 3
  \end{pmatrix}\begin{pmatrix}
    1 & 3 & 4 & 2 \\
    1 & 2 & 3 & 4
  \end{pmatrix} = \begin{pmatrix}
    4 & 3 & 1 & 2 \\
    2 & 4 & 1 & 3
  \end{pmatrix}\begin{pmatrix}
    1 & 4 & 2 & 3 \\
    4 & 3 & 1 & 2
  \end{pmatrix}\\
  \begin{pmatrix}
    1 & 2 & 3 & 4 \\
    1 & 4 & 2 & 3
  \end{pmatrix} = \begin{pmatrix}
    1 & 2 & 3 & 4 \\
    2 & 4 & 1 & 3
  \end{pmatrix} = (1\ 2\ 4\ 3)$\\
  (6) $(2\ 4)(1\ 3)(1\ 3)(1\ 3)(2\ 4) = (2\ 4)(1\ 3)(2\ 4) = (1\ 3)(2\ 4)(2\ 4) = (1\ 3)$
\end{ans}

\fakesubsection{2.2 環・体の定義}

\begin{ans}
  (1) $9 \equiv 2 \pmod{7}$より, $\overline{2}$.
  (2) $-3 \equiv 4 \pmod{7}$より, $\overline{4}$.
  (3) $20 \equiv 6 \pmod{7}$より, $\overline{6}$.
  (4) $125 \equiv 6 \pmod{7}$より, $\overline{6}$.
  (5) $4^3 \equiv 64 \equiv 1 \pmod{7}$より, $4^{32} \equiv 4^2 \equiv 2 \pmod{7}$であるから, $\overline{2}$.
\end{ans}

\begin{ans}
  (1) $34 \cdot 21 \cdot 11 \equiv 8 \cdot 8 \cdot 11 \equiv 64 \cdot 11 \equiv 12 \cdot 11 \equiv 132 \equiv 2 \pmod{13}$
  より, $34 \cdot 21 \cdot 33 \equiv 2 \cdot 3 \equiv 6 \pmod{39}$. よって$\overline{6}$.\\
  (2) $18 \cdot 13 = 6 \cdot 39 \equiv 0 \pmod{39}$より, $\overline{0}$.\\
  (3) $16^8 \equiv (13 + 3)^8 \equiv 13^8 + 3^8 \pmod{39}$であるが,
  $13^2 \equiv 13 \pmod{39}$, $3^4 \equiv 3 \pmod{39}$であるから,
  $13^8 + 3^8 \equiv 13 + 3^2 \equiv 22$.
  よって$\overline{22}$.\\
  (4) (3) と同様にして,
  $16^{34} \equiv 13^{34} + 3^{34} \equiv 13 + 3^{10} \equiv 13 + 3^4 \equiv 13 + 3 \equiv 16$.
  (冪を$4$で割った商と余りを足していく.)
  よって$\overline{16}$.\\
\end{ans}

\fakesubsection{2.3 部分群と生成元}

\begin{ans}
  命題2.3.2の条件と同値であることを確かめればよい.
  $H$が部分群であるとすると, $x, y \in H$ならば条件 (3) より$x^{-1} \in H$.
  さらに条件 (2) より$x^{-1}y \in H$.
  逆に, 群$G$の空でない部分集合$H$について
  「$x, y \in H \Rightarrow x^{-1}y \in H$」
  が成り立っているとする. $H$は空でないから, ある元$x_0 \in H$が存在する.
  よって, $1_G = x_0^{-1}x_0 \in H$. すなわち条件 (1) が成り立つ.
  $x \in H$ならば, $x, 1_G \in H$について$x^{-1} = x^{-1}1_G \in H$. すなわち条件 (2) が成り立つ.
  $x, y \in H$ならば, $x^{-1} \in H$なので$xy = (x^{-1})^{-1}y \in H$. すなわち条件 (3) が成り立つ.
\end{ans}

\begin{ans}
  $\transpose{I_{2n}}J_nI_{2n} = J_n$より, $I_{2n} \in \Sp{2n}$.
  $h_1, h_2 \in \Sp{2n}$ならば,
  $\transpose{(h_1h_2)}J_n(h_1h_2) = \transpose{h_2}(\transpose{h_1}J_nh_1)h_2 = \transpose{h_2}J_nh_2 = J_n$より
  $h_1h_2 \in \Sp{2n}$.
  また, $h \in \Sp{2n}$ならば,
  $\transpose{(h^{-1})}J_nh^{-1} = \transpose{(h^{-1})}(\transpose{h}J_nh)h^{-1} = J_n$.
  よって$h^{-1} \in \Sp{2n}$.
\end{ans}

\begin{ans}
  $\overline{AB} = \overline{A}\,\overline{B}$であることに注意する.
  (このことから, $\overline{g}^{-1} = \overline{g^{-1}}$であることも分かる.)
  $\transpose{\overline{I_n}I_n} = I_nI_n = I_n$より$I_n \in \U{n}$.
  $h_1, h_2 \in \U{n}$ならば,
  $\transpose{\overline{h_1h_2}}h_1h_2
  = \transpose{\overline{h_2}}\,\transpose{\overline{h_1}}h_1h_2
  = \transpose{\overline{h_2}}I_nh_2 = I_n$より$h_1h_2 \in \U{n}$.
  また, $h \in \U{n}$ならば,
  $\transpose{\overline{h^{-1}}}h^{-1}
  = \transpose{\overline{h}^{-1}}(\transpose{\overline{h}}h)h^{-1}
  = I_nI_n = I_n$.
  よって$h^{-1} \in \U{n}$.
\end{ans}

\begin{ans}
  (1) 明らかに$I_n \in B$.
  $b = (b_{ij}),\ b^\prime = (b^\prime_{ij})$をともにBの元とすると,
  $bb^\prime$の$(i, j)$成分は$\sum_{k = 1}^nb_{ik}b^\prime_{kj}$である.
  もし$i < j$ならば, どんな$k$に対しても$i < k$または$k < j$が成り立つので,
  各項が$0$となり$bb^\prime$の$(i, j)$成分は$0$である. すなわち$bb^\prime \in B$.
  $b \in B$とすると, $b$に
  (i) 行を$\lambda (\neq 0)$倍する
  (ii) 第$i$行に第$j$行の$\lambda$倍を足す ($i > j$),
  という$2$つの操作 (基本変形の一部) を繰り返すことにより, 単位行列にできる.
  これらの操作は$b$に$B$のある元を左から掛けることに対応しており, それらの積が$b^{-1}$に他ならない.
  $B$が積で閉じていることはすでに示していたから, $b^{-1} \in B$.\\
  (2) $n = 1$のとき$B = G = \mathbb{R}^\times$は可換群である.
  $n \ge 2$のとき, $B$は可換群ではない. たとえば$n = 2$では
  $\begin{pmatrix}
    1 & 0 \\
    1 & 1
  \end{pmatrix}\begin{pmatrix}
    1 & 0 \\
    0 & 2
  \end{pmatrix} \neq \begin{pmatrix}
    1 & 0 \\
    0 & 2
  \end{pmatrix}\begin{pmatrix}
    1 & 0 \\
    1 & 1
  \end{pmatrix}$.
  一般の$n \ge 2$の場合にも, たとえば単位行列の定数倍でない対角行列と,
  左下の成分がすべて$1$である行列との積は非可換である.
\end{ans}

\begin{ans}
  $1 \in \mathbb{R}_>$,
  また$x, y \in \mathbb{R}_>$ならば$xy \in \mathbb{R}_>$,
  また$x \in \mathbb{R}_>$ならば$x^{-1} = \frac{1}{x} \in {R}_>$.
\end{ans}

\begin{ans}
  $0 \notin \mathbb{R}_>$なので, 部分群ではない.
\end{ans}

\begin{ans}
  $H$の元$h$は, 絶対値が$1$なので$h = e^{i\theta}$と書ける.
  さらに, $h^n = 1$より$\theta = \frac{2\pi m}{n}\ (m \in \mathbb{Z})$である.
  $e^\frac{2\pi im}{n} = e^\frac{2\pi i m^\prime}{n}$が成り立つのは,
  $m^\prime = m + kn\ (k \in \mathbb{Z})$と書けるとき, またそのときに限るので,
  $H = \{1 = e^0, e^\frac{2\pi i}{n}, e^\frac{2\pi i \cdot 2}{n},..., e^\frac{2\pi i(n - 1)}{n}\}$.
  よって$H$の位数は$n$.
  また, $\gen{e^\frac{2\pi i}{n}} = H$なので, $H$は巡回群.
\end{ans}

\begin{ans}
  (1) $\mathfrak{S}_3$は可換群ではないから巡回群ではない.
  (別解: $\mathfrak{S}_3$には位数$2$の元が$3$つあるが, 巡回群$\Z{6}$には$1$つしかないから, 両者は一致しない.) \\
  (2) $\mathbb{Q}$が巡回群であると仮定して, 生成元が$r \neq 0$であるとする.
  このとき, $\mathbb{Q}$の任意の元は$nr\ (n \in \mathbb{Z})$という形で表せるはずだが, $\frac{r}{2} \in \mathbb{Q}$はそうではないので矛盾. \\
  (3) $\mathbb{Q}$の場合と同様. (別解: 巡回群の濃度はたかだか可算だが, $\mathbb{R}$の濃度は非可算なので一致しない.) \\
  (4) $\mathbb{Q}^\times$が巡回群であると仮定して, 生成元が$r$であるとする.
  明らかに$r \neq \pm 1$である. 一方$r^{n} = -1$となるような$n \in \mathbb{Z}$が存在するはずであるが,
  この式を満たすような$r$は$-1$しかない ($r$を既約分数で表示してみればこのことが分かる) ので矛盾.
  \\
  (5) $\mathbb{Z} \times \mathbb{Z}$が巡回群であると仮定して, 生成元が$(x, y)$であるとする.
  $(nx, ny)\ (n \in \mathbb{Z})$が$\mathbb{Z} \times \mathbb{Z}$のすべての元を渉っている必要があるので,
  $x, y$それぞれが$\mathbb{Z}$の生成元でなれけばならない. よって少なくとも$x \neq 0$である.
  一方, $(x, y + 1) \in \mathbb{Z} \times \mathbb{Z}$を考えると,
  $(x, y + 1) = (nx, ny)$を満たす$n \in \mathbb{Z}$が存在するはずだが,
  第$1$成分について$x = nx$かつ$x \neq 0$より$n = 1$. 一方第$2$成分は$y + 1 = y$となり矛盾.
\end{ans}

\begin{ans}
  (1) 数学的帰納法により示す. $n = 1$のときは明らか.
  $n = k$のときに正しいとして, $n = k + 1$の場合を示す.
  任意の$\sigma \in \mathfrak{S}_{k + 1}$をとり.
  $\sigma(1) = i$であるとする.
  $\tau = (1\ 2)(2\ 3)\cdots(i-1\ i)\sigma$とおけば$\tau(1) = 1$であるから,
  $n = k$の場合に帰着できる. よって$n = k + 1$の場合にも正しい. \\
  (2) $(1\ 2\ \cdots\ n)^{-(i-1)}(1\ 2)(1\ 2\ \cdots\ n)^{i-1} = (i\ i + 1)$なので,
  (1) よりこれらも$\mathfrak{S}_n$を生成する.
\end{ans}

\fakesubsection{2.4 元の位数}

\begin{ans}
  (1) 12, 144 (2) yes
\end{ans}

\begin{ans}
  (1)
  $395 = 1 \cdot 265 + 130$,
  $265 = 2 \cdot 130 + 5$,
  $130 = 26 \cdot 5$.
  よって$d = 5$.\\
  (2)
  $5
  = 265 - 2 \cdot 130
  = 265 - 2 \cdot (395 - 1 \cdot 265)
  = -2 \cdot 395 + 3 \cdot 265$.
  よって$(x, y) = (-2, 3)$が$1$つの解である.
\end{ans}

\begin{ans}
  (1) 位数が少ないので単純に総当りで調べる.
  $\overline{2}^{-1} = \overline{4}$,
  $\overline{3}^{-1} = \overline{5}$,
  $\overline{4}^{-1} = \overline{2}$,
  $\overline{5}^{-1} = \overline{3}$,
  $\overline{6}^{-1} = \overline{6}$.\\
  (2) $284x + 3y = 1$の$1$つの解をユークリッドの互除法によって求めると,
  $(x, y) = (-1, 95)$となる. よって$\overline{3}^{-1} = \overline{95}$.
\end{ans}

\begin{ans}
  $1,..., p^n$のうち$p^n$と互いに素でないのは$p$の倍数で, $p^{n-1}$個ある.
  よって位数は$p^n - p^{n-1}$.
\end{ans}

\begin{ans}
  $x^{35d} = 1$であることと$35d$が$60$の倍数であることは同値.
  さらにこのことは$7d$が$12$の倍数であることと同値.
  $7$と$12$は互いに素なので, これは$d$が$12$の倍数であることと同値である.
  よって位数は$12$.
\end{ans}

\begin{ans}
  $x^{nm} = 1$であることと$nm$が$d$の倍数であることは同値.
  さらにこのことは$nm/\gcd(n, d)$が$d/\gcd(n, d)$の倍数であることと同値.
  $n/\gcd(n, d)$と$d/\gcd(n, d)$は互いに素なので, これは$m$が$d/\gcd(n, d)$の倍数であることと同値である.
  よって位数は$d/\gcd(n, d)$.
\end{ans}

\begin{ans}
  位数$d$の群$G$を生成する元の位数は$d$である.
  前問より$\overline{n}$の位数は$d/\gcd(d, n)$であるから,
  $d$と互いに素であるような$n$についての$\overline{n}$を挙げればよい.
  たとえば$\Z{15}$では
  $\overline{1}, \overline{2}, \overline{4}, \overline{7}, \overline{8}, \overline{11}, \overline{13}, \overline{14}$.
  他も同様.
\end{ans}

\begin{ans}
  任意の$g, h \in G$に対し,
  $gh = (gh)^{-1} = h^{-1}g^{-1} = hg$.
\end{ans}

\begin{ans}
  (1) $g^2 = \begin{pmatrix}
    -1 & 0 \\
    0 & -1
  \end{pmatrix},\
  g^4 = \begin{pmatrix}
    1 & 0 \\
    0 & 1
  \end{pmatrix}$.
  よって$g$の位数は$4$の約数であり, かつ$1$, $2$ではないから, 位数は$4$.
  また$h$については,
  $h^2 = \begin{pmatrix}
    0 & 1 \\
    -1 & -1
  \end{pmatrix},\
  h^3 = \begin{pmatrix}
    -1 & 0 \\
    0 & -1
  \end{pmatrix},\
  h^6 = \begin{pmatrix}
    1 & 0 \\
    0 & 1
  \end{pmatrix}$.
  よって位数は$6$の約数であり, かつ$1$, $2$, $3$ではないから, 位数は$6$.\\
  (2) $gh = \begin{pmatrix}
    1 & 0 \\
    1 & 1
  \end{pmatrix} = \begin{pmatrix}
    1 & 0 \\
    0 & 1
  \end{pmatrix} + \begin{pmatrix}
    0 & 0 \\
    1 & 0
  \end{pmatrix}$
  と分解できるが, $\begin{pmatrix}
    0 & 0 \\
    1 & 0
  \end{pmatrix}^2 = 0$であることに注意すると, 任意の正の整数$n$に対して
  $(gh)^n = \begin{pmatrix}
    1 & 0 \\
    0 & 1
  \end{pmatrix} + n \cdot \begin{pmatrix}
    0 & 0 \\
    1 & 0
  \end{pmatrix} \neq \begin{pmatrix}
    1 & 0 \\
    0 & 1
  \end{pmatrix}$.
\end{ans}

\begin{ans}
  (1) $a^n = 1, b^m = 1$ならば, $(ab)^{nm} = a^{nm}b^{nm} = 1$.
  (2) $1$の位数は$0$なので$1 \in H$.
  また$x$の位数は$x^{-1}$の位数に等しいので, $x \in H$ならば$x^{-1} \in H$.
\end{ans}

\fakesubsection{2.5 準同型と同型}

\begin{ans}
  (1) 「…」が成り立つとする.
  $i_1 = m, i_2 = 0$とすると, $x^{i_1} = x^{i_2} = 1$.
  よって$y^{i_1} = y^{i_2}$, すなわち$y^m = 1$.
  命題2.4.19より, $m$は$n$の倍数である.
  逆に, $m$が$n$の倍数であるとする.
  $x^{i_1} = x^{i_2}$であるような任意の$i_1, i_2 \in \mathbb{Z}$に対して,
  $i_1 - i_2$は命題2.4.19より$m$の倍数であり, 仮定からこれは$n$の倍数でもある.
  よって$y^{i_1 - i_2} = 1$, すなわち$y^{i_1} = y^{i_2}$であるから, 「…」が成り立つ.
  以上により, 「…」が成り立つための必要十分条件は, $m$が$n$の倍数であることである.\\
  (2) $m$, $n$が (1) の性質を満たすことから, $\phi(x^i) = y^i$と定めると$\phi$はwell-definedである.
  $\phi(x^ix^j) = \phi(x^{i+j}) = y^{i+j} = y^iy^j = \phi(x^i)\phi(x^j)$
  より, $\phi$は準同型.
\end{ans}

\begin{ans}
  $\phi_n(gh) = (gh)^n = g^nh^n = \phi_n(gh)$.
\end{ans}

\begin{ans}
  (1) $g$の位数を$n$とすると,$\phi(g)^n = \phi(g^n) = \phi(1_G) = 1_H$.
  よって$\phi(g)$の位数は$n$の約数である.
  (2) $\phi(g)$の位数を$m$とすると, $\phi(g^m) = \phi(g)^m = 1_H$.
  $\phi$が単射なので, $g^m = 1_G$. よって$g$の位数は$m$の約数である.
  (1) と合わせて, $g$の位数と$\phi(g)$の位数は等しい.
\end{ans}

\begin{ans}
  $\Z{4}$には位数$4$の元があるが,
  $\Z{2} \times \Z{2}$にはないので,
  これらは同型ではない.
\end{ans}

\begin{ans}
  $n \ge 0$の場合を数学的帰納法により示す.
  $n = 0$の場合は明らかに真である.
  $n = k$で真であるとすると,
  $(xyx^{-1})^{k+1} = (xyx^{-1})^k(xyx^{-1}) = (xy^kx^{-1})(xyx^{-1}) = xy^{k+1}x^{-1}$より,
  $n = k + 1$についても真である.
  $n < 0$の場合は, $(xyx^{-1})^n = ((xyx^{-1})^{-n})^{-1} = (xy^{-n}x^{-1})^{-1} = xy^nx^{-1}$.
\end{ans}

\begin{ans}
  $\begin{pmatrix}
    a & b \\
    c & d
  \end{pmatrix}$を$\mathrm{GL}_2(\mathbb{R})$または$\mathrm{GL}_2(\mathbb{C})$の元として,
  \[
    \begin{pmatrix}
      1 & 1 \\
      0 & 1
    \end{pmatrix} = \begin{pmatrix}
      a & b \\
      c & d
    \end{pmatrix} \begin{pmatrix}
      1 & 0 \\
      1 & 1
    \end{pmatrix} \cdot \frac{1}{ad - bc} \begin{pmatrix}
      d & -b \\
      -c & a
    \end{pmatrix} = \frac{1}{ad - bc} \begin{pmatrix}
      a + b & b \\
      c + d & d
    \end{pmatrix} \begin{pmatrix}
      d & -b \\
      -c & a
    \end{pmatrix}
  \]
  であるとする. $(2, 1)$成分を比較すると$0 = d^2$であるから, $d = 0$.
  よって
  \[
    \begin{pmatrix}
      1 & 1 \\
      0 & 1
    \end{pmatrix} = - \frac{1}{bc} \begin{pmatrix}
      a + b & b \\
      c & 0
    \end{pmatrix} \begin{pmatrix}
      0 & -b \\
      -c & a
    \end{pmatrix} = - \frac{1}{bc} \begin{pmatrix}
      -bc & -b^2 \\
      0 & -bc
    \end{pmatrix} = \begin{pmatrix}
      1 & \frac{b}{c} \\
      0 & 1
    \end{pmatrix}
  \]
  $(1, 2)$成分を比較して$b = c$である.
  逆に, $d = 0$, $b = c$, $ad - bc \neq 0$となるように$a, b, c, d$を定めれば,
  \[
    \begin{pmatrix}
      1 & 1 \\
      0 & 1
    \end{pmatrix} = \begin{pmatrix}
      a & b \\
      c & d
    \end{pmatrix} \begin{pmatrix}
      1 & 0 \\
      1 & 1
    \end{pmatrix} \begin{pmatrix}
      a & b \\
      c & d
    \end{pmatrix}^{-1}
  \]
  が成り立つことが確かめられる.
  したがって$A$, $B$は$\mathrm{GL}_2(\mathbb{R})$では共役である.
  $d = 0$, $b = c$, $ad - bc \neq 0$という条件のもとで, もし$b \in \mathbb{R}$ならば$
  \begin{vmatrix}
    a & b \\
    c & d
  \end{vmatrix} = - b^2 < 0
  $なので, $A$, $B$は$\mathrm{SL}_2(\mathbb{R})$では共役でない.
  一方$b = i$とすれば$
  \begin{vmatrix}
    a & b \\
    c & d
  \end{vmatrix} = 1
  $なので, $\mathrm{SL}_2(\mathbb{C})$では共役である.
\end{ans}

\begin{ans}
  $G = \Z{15}$の場合のみを考える. (他も同様である.)
  自己準同型$\phi: G \rightarrow G$は,
  $1$つの生成元での値$\phi(\overline{1})$を与えれば一意に定まる.
  そうして定めた$\phi$が同型であるための必要十分条件は, $\phi(\overline{1})$がまた$G$の生成元であることである.
  $G$の生成元は演習問題2.4.7で求めたとおり,
  $\overline{1}, \overline{2}, \overline{4}, \overline{7}, \overline{8}, \overline{11}, \overline{13}, \overline{14}$
  の$8$通りなので, $\mathrm{Aut}(G)$の位数は$8$.
  また, $\overline{1} \mapsto \overline{k}$なる自己同型は$x \mapsto \overline{k}x$と書けるので,
  明らかに$\mathrm{Aut}(G)$はアーベル群である.
  有限生成アーベル群の基本定理 (定理4.8.2) より,
  $\mathrm{Aut}(G) \simeq \Z{8}$
  または
  $\mathrm{Aut}(G) \simeq \Z{2} \times \Z{4}$
  または
  $\mathrm{Aut}(G) \simeq \Z{2} \times \Z{2} \times \Z{2}$
  であるが, $\mathrm{Aut}(G)$に単一の生成元がないこと,
  および$\overline{1} \mapsto \overline{2}$なる自己同型の位数が$4$であることが計算により確かめられるので,
  $\mathrm{Aut}(G) \simeq \Z{2} \times \Z{4}$.
\end{ans}

\begin{ans}
  (1) $b(ab)b^{-1} = ba$.
  (2) $(ab)^n = 1$ならば$(ba)^n = (b(ab)b^{-1})^n = b(ab)^nb^{-1} = 1$.
  同様に$(ba)^n = 1$ならば$(ab)^n = 1$なので, $ab$と$ba$の位数は等しい.
\end{ans}

\begin{ans}
  $G$は互換$(1\ 2)$と$(2\ 3)$で生成されるので,
  $\mathrm{Aut}(G)$の元は$(1\ 2), (2\ 3)$の行き先を定めることで決定できる.
  これらは位数$2$の元なので, 同型写像による行き先は同じく位数$2$の異なる元でなければならず, 候補は$3$つある.
  したがって, $\mathrm{Aut}(G)$の位数はたかだか${}_3 \mathrm{P}_2 = 6$である.
  $G$の位数も$6$なので, $\phi$が単射 ($\Ker{\phi} = \{1\}$) であることを確かめれば, $\phi$が同型であることが分かる.
  $\sigma \in \Ker{\phi}$ならば, 任意の$\tau \in G$に対して
  $\sigma \tau \sigma^{-1} = \tau$, すなわち$\sigma$と$\tau$は可換である.
  互換$(1\ 2)$と可換な元は$1$と$(1\ 2)$のみであることが計算で確かめられ,
  また$(1\ 3)$についても同様なので, $G$のすべての元と可換であるのは$1$のみである.
  よって$\sigma = 1$.
\end{ans}

\begin{ans}
  (1) 次のように, $G$の元と$U, L$の元との積と, 行列の基本変形とが対応していることに注意する.
  \begin{itemize}
    \item $U$の元を左から掛ける $\Leftrightarrow$ 第$2$行の$x$倍を第$1$行に足す
    \item $U$の元を右から掛ける $\Leftrightarrow$ 第$1$列の$x$倍を第$2$列に足す
    \item $L$の元を左から掛ける $\Leftrightarrow$ 第$1$行の$x$倍を第$2$行に足す
    \item $L$の元を右から掛ける $\Leftrightarrow$ 第$2$列の$x$倍を第$1$列に足す
  \end{itemize}
  この変形を任意の$g = \begin{pmatrix}
    a & b \\
    c & d \\
  \end{pmatrix} \in G$に対して次のように適用する. (ただし$x$はその都度適当な値を選ぶ) \\
  (i) もし$a = 0$ならば$c \neq 0$なので, 第$2$行の$x$倍を第$1行$に足すことで$(1, 1)$成分を非零にする \\
  (ii) 第$1$行の$x$倍を第$2$行に足して, $(2, 1)$成分を非零にする \\
  (iii) 第$2$行の$x$倍を第$1$行に足して, $(1, 1)$成分を$1$にする \\
  (iv) 第$1$行の$x$倍を第$2$行に足して, $(2, 1)$成分を$0$にする \\
  (vv) 第$1$列の$x$倍を第$2$列に足して, $(1, 2)$成分を$0$にする \\
  これらの変形のあと, $\det$が$1$であることから行列は単位行列になっている.
  よって, $g$は$U$, $L$の元の積で表せる.\\
  (2) $U$, $L$の任意の元$g$に対して$\phi(g) = 1$であることを示せばよい.
  まず$g = \begin{pmatrix}
    1 & x \\
    0 & 1
  \end{pmatrix} \in U$とする.
  もし$\abs{\phi(g)} > 1$ならば, $\phi(g)^n\ (n = 1, 2,...)$の絶対値は指数的に増大するが,
  一方で$g^n = \begin{pmatrix}
    1 & nx \\
    0 & 1
  \end{pmatrix}$より$\phi(g^n)$は$n$の多項式なので矛盾.
  $\abs{\phi(g)} < 1$としても同様に矛盾するので, $\abs{\phi(g)} = 1$である.
  また, 通常の位相により$U$は連結かつ$\phi$は連続なので, $\phi(U)$は連結である.
  したがって$\phi(g) = 1$.
  $L$の元についても同様である.
\end{ans}

\fakesubsection{2.6 同値関係と剰余類}

\begin{ans}
  同値関係ではない.
  $\{(x, x) \mid x \in \mathbb{R}\} \subset R$なので, 反射律は満たされている.
  また
  $(a, b) \in \{(x, x) \mid x \in \mathbb{R}\}$ならば$(b, a) \in \{(x, x) \mid x \in \mathbb{R}\}$,
  $(a, b) \in \{(x, 2x) \mid x \in \mathbb{R}\}$ならば$(b, a) \in \{(2x, x) \mid x \in \mathbb{R}\}$,
  $(a, b) \in \{(2x, x) \mid x \in \mathbb{R}\}$ならば$(b, a) \in \{(x, 2x) \mid x \in \mathbb{R}\}$
  なので, 対称律も満たされている.
  しかし, 推移律は満たされていない.
  たとえば$(1, 2), (2, 4) \in R$であるが$(1, 4) \notin R$.
\end{ans}

\begin{ans}
  (反射律) $a = 1a1^{-1}$.
  (対称律) $a = gbg^{-1}$ならば$b = g^{-1}a(g^{-1})^{-1}$.
  (推移律) $a = gbg^{-1}$, $b = hch^{-1}$ならば$a = g(hch^{-1})g^{-1} = (gh)c(gh)^{-1}$.
\end{ans}

\begin{ans}
  系2.6.21よりしたがう.
\end{ans}

\begin{ans}
  $H \cap K$は$H$, $K$の部分群なので$\abs{H \cap K}$は$\abs{H}$と$\abs{K}$の公約数である.
  よって$\abs{H \cap K} = 1$なので$H \cap K = \{1_G\}$.
\end{ans}

\fakesubsection{2.7 両側剰余類}

\begin{ans}
  $\sigma \in G$とする. もし$\sigma(4) = 4$なら, $\sigma \in H$より
  $\sigma \in H1_GH$である.
  もし$\sigma(4) = i (i \neq 4)$なら, $\sigma (i\ 3)(3\ 4) \in H$すなわち$\sigma \in H(3\ 4)(i\ 3)$より
  $\sigma \in H(3\ 4)H$である.
  以上により, 任意の$\sigma \in G$に対して$\sigma \in H1_GH$または$\sigma \in H(3\ 4)H$が成り立つ.
  $(3\ 4) \notin H1_GH$より, $\{1_G, (3\ 4)\}$が$1$つの完全代表系である.
\end{ans}

\begin{ans}
  $P$の元を左または右から掛けることは, 行列の基本変形 (の一部) に対応している.
  具体的には, 行列の基本変形のうち以下のものを\textbf{除いた}操作のみが許される.
  \begin{itemize}
    \item 第$3$行の$x$倍を第$1$(or $2$)行に足す
    \item 第$1$(or $2$)列の$x$倍を第$3$列に足す
    \item 第$3$行 (列) をほかの行 (列) と交換する
  \end{itemize}
  そこで, $G$の任意の元$g$に対して, その第$3$列の値に注目して,
  上\textbf{以外}の基本変形によって$w_1$または$w_2$に移せることを示す.\\
  (i) $g$の$(1, 3)$成分または$(2, 3)$成分が非零の場合.\\
  必要ならば第$1$行と第$2$行を入れ替えることで, $(1, 3)$成分が非零としてよい.
  このとき, 第$1$行と第$3$列を掃き出して
  \begin{align*}
    \begin{pmatrix}
      0 & 0 & 1 \\
      ? & ? & 0 \\
      ? & ? & 0
    \end{pmatrix}
  \end{align*}
  とできる. $(2, 1)$成分と$(2, 2)$成分のいずれかは非零なので, そのいずれかに応じて
  \begin{align*}
    \begin{pmatrix}
      0 & 0 & 1 \\
      1 & 0 & 0 \\
      0 & 1 & 0
    \end{pmatrix}または
    \begin{pmatrix}
      0 & 0 & 1 \\
      0 & 1 & 0 \\
      1 & 0 & 0
    \end{pmatrix}
  \end{align*}
  と変形できる. 後者からはさらに
  \begin{align*}
    \begin{pmatrix}
      0 & 0 & 1 \\
      0 & 1 & 0 \\
      1 & 0 & 0
    \end{pmatrix}
    \rightarrow
    \begin{pmatrix}
      1 & 0 & 1 \\
      0 & 1 & 0 \\
      1 & 0 & 0
    \end{pmatrix}
    \rightarrow
    \begin{pmatrix}
      1 & 0 & 0 \\
      0 & 1 & 0 \\
      1 & 0 & -1
    \end{pmatrix}
    \rightarrow
    \begin{pmatrix}
      1 & 0 & 0 \\
      0 & 1 & 0 \\
      0 & 0 & -1
    \end{pmatrix}
    \rightarrow
    \begin{pmatrix}
      1 & 0 & 0 \\
      0 & 1 & 0 \\
      0 & 0 & 1
    \end{pmatrix}
  \end{align*}
  と変形できるので, $w_1$と$w_2$のいずれかに変形できることがわかった.
  \\
  (ii) $g$の$(1, 3)$成分と$(2, 3)$成分がともに$0$の場合.\\
  このとき$(3, 3)$成分は非零なので, 第$3$行を掃き出して (第3列はすでに掃き出されている),
  \begin{align*}
    \begin{pmatrix}
      ? & ? & 0 \\
      ? & ? & 0 \\
      0 & 0 & 1
    \end{pmatrix}
  \end{align*}
  とできる. 左上の$2 \times 2$行列についてはすべての基本変形が許されるので, これは$w_1$と同値である.

  以上により, $G$の任意の元は$w_1$または$w_2$と同値であることがわかった.
  $w_1$と$w_2$とが同値でないことは, $w_1 \in P$かつ$w_2 \notin P$であることから明らかである.
\end{ans}

\begin{ans}
  (1) $g$の第$i_n$行の$c\ (\in \mathbb{R})$倍を第$j\ (> i_n)$行に足すことは, $g$に$B$のある元を左から掛けることに等しい.
  また, $g$の第$n$列の$c\ (\in \mathbb{R})$倍を第$j\ (< n)$列に足すことは, $g$に$B$のある元を右から掛けることに等しい.
  したがって, $g$に対してこれらの操作 (行列の基本変形) を適当に繰り返すことによって条件を満たすような$h$に変形することを考えれば,
  ある$b_1, b_2 \in B$に対して$h = b_1gb_2$が成り立つ.\\
  (2) (1) において$n$について行った操作を$n-1, n-2,..., 1$について順に行い, さらに適当に対角行列を掛けると,
  $h = b_1gb_2$で, $h$の各列で$1$箇所だけが$1$で残りの成分はすべて$0$であるようなものがとれる.
  $h \in \mathrm{GL}_n(\mathbb{R})$なので, 各列で成分が$1$である行はすべて異なる.
  すなわち$h$は置換行列である.\\
  (3) $P_\sigma$の$(i, j)$成分が$\delta_{i\sigma(j)}$であることと,
  $j \neq n$ならば$b_{2, jn} = 0$であることから,
  \begin{align*}
    (b_1 P_\sigma b_2)_{in} &= \sum_{j, k}b_{ij}\delta_{j\sigma(k)}b_{2, kn} \\
    &= \sum_kb_{i\sigma(k)}b_{2, kn} \\
    &= b_{i\sigma(n)}b_{2, nn} \\
  \end{align*}
  ここで$i = \sigma(n)$とすると, (右辺) $\neq 0$であるが,
  これが$P_\tau$の$(\sigma(n), n)$成分 ($= \delta_{\sigma(n)\tau(n)}$) に等しいので,
  $\sigma(n) = \tau(n)$である.\\
  (4) (前半) (3) の式で$i \neq \sigma(n)$の場合を考えると,
  (右辺) $= 0$かつ$b_{2, nn} \neq 0$であるから$b_{1, i\sigma(n)} = 0$.\\
  (後半) 明らかに$P_\nu b_1 P_\nu^{-1}$は正則なので,
  その$(i, j)$成分が$i < j$のとき$0$であることを示せばよい.
  \begin{align*}
    (P_\nu b_1 P_\nu^{-1})_{ij} &= \sum_{k, l}P_{\nu, ik}b_{1, kl}(P_\nu^{-1})_{lj} \\
    &= \sum_{k, l}\delta_{i\nu(k)}b_{1, kl}\delta_{\nu(l)j} \\
    &= b_{1, \nu^{-1}(i)\nu^{-1}(j)}
  \end{align*}
  $\nu$の定め方から, $i < j$かつ$\nu^{-1}(i) > \nu^{-1}(j)$が成り立つのは$j = n$のときだけであり,
  このとき$\nu^{-1}(j) = \nu^{-1}(n) = \sigma(n)$.
  よって(4)より$j = n$のときも上式右辺は$0$である.\\
  (5) (3) の式で, $n$を$n-1$に置き換えたものを考えると,
  \begin{align*}
    (b_1 P_\sigma b_2)_{i(n-1)} &= \sum_{j, k}b_{1, ij}\delta_{j\sigma(k)}b_{2, k(n-1)} \\
    &= \sum_kb_{1, i\sigma(k)}b_{2, k(n-1)} \\
    &= b_{1, i\sigma(n-1)}b_{2, (n-1)(n-1)} + b_{1, i\sigma(n)}b_{2, n(n-1)} \\
  \end{align*}
  まず$i = \sigma(n-1)$とすると, (4) の前半より右辺第$2$項は$0$である.
  第$1$項$\neq 0$であるから右辺は全体として$\neq 0$であり,
  これが$P_\tau$の$(\sigma(n-1), n-1)$成分なので,
  (3)と同様に$\sigma(n-1) = \tau(n-1)$である.
  また, $i \neq \sigma(n-1), \sigma(n)$ならば, (4) の前半と同様にして$b_{1, i\sigma(n-1)} = 0$.
  以下同様に (3) の式で$n$を$n-2, n-3,..., 1$に置き換えたものを順に考えれば,
  $\sigma(n-2) = \tau(n-2),\ \sigma(n-3) = \tau(n-3),..., \sigma(1) = \tau(1)$が示せる.\\
  (メモ) (1)-(5) より両側剰余類$B\backslash G/B$の完全代表系はすべての置換行列からなる集合であることが分かった.\\
  (メモ2) (4) の後半を使っていないので, (5) は想定解答ではなさそうな気がする.
\end{ans}

\fakesubsection{2.8 正規部分群と剰余群}

\begin{ans}
  (1) 正規部分群でない. $g = (3\ 4)$, $h = (1\ 2)$とおくと$h \in H$で
  $ghg^{-1} = (3\ 4)(1\ 3)(3\ 4) = (1\ 4) \notin H$.\\
  (2) 正規部分群でない.
  $g = \begin{pmatrix}
    1 & 0 \\
    0 & 2
  \end{pmatrix}$, $h = \begin{pmatrix}
    \frac{1}{\sqrt{2}} & \frac{1}{\sqrt{2}} \\
    - \frac{1}{\sqrt{2}} & \frac{1}{\sqrt{2}}
  \end{pmatrix}$とおくと
  \begin{align*}
    ghg^{-1} &= \begin{pmatrix}
      1 & 0 \\
      0 & 2
    \end{pmatrix}\begin{pmatrix}
      \frac{1}{\sqrt{2}} & \frac{1}{\sqrt{2}} \\
      - \frac{1}{\sqrt{2}} & \frac{1}{\sqrt{2}}
    \end{pmatrix}\begin{pmatrix}
      1 & 0 \\
      0 & \frac{1}{2}
    \end{pmatrix} = \begin{pmatrix}
      \frac{1}{\sqrt{2}} & \frac{1}{2\sqrt{2}} \\
      - \sqrt{2} & \frac{1}{\sqrt{2}}
    \end{pmatrix} \\
    \transpose{(ghg^{-1})}(ghg^{-1}) &= \begin{pmatrix}
      \frac{1}{\sqrt{2}} & - \sqrt{2} \\
      \frac{1}{2\sqrt{2}} & \frac{1}{\sqrt{2}}
    \end{pmatrix}\begin{pmatrix}
      \frac{1}{\sqrt{2}} & \frac{1}{2\sqrt{2}} \\
      - \sqrt{2} & \frac{1}{\sqrt{2}}
    \end{pmatrix} \neq I_n
  \end{align*}
  したがって$ghg^{-1} \notin H$.\\
  (3) 正規部分群でない.
  $g = \begin{pmatrix}
    1 & 0 \\
    0 & i
  \end{pmatrix}$, $h = \begin{pmatrix}
    1 & 1 \\
    0 & 1
  \end{pmatrix}$とおくと,
  \[
    ghg^{-1} = \begin{pmatrix}
      1 & 0 \\
      0 & i
    \end{pmatrix}\begin{pmatrix}
      1 & 1 \\
      0 & 1
    \end{pmatrix}\begin{pmatrix}
      1 & 0 \\
      0 & -i
    \end{pmatrix} = \begin{pmatrix}
      1 & -i \\
      0 & 1
    \end{pmatrix} \notin \mathrm{GL}_2(\mathbb{R})
  \]
  (4) 正規部分群である. $(1\ 2)(3\ 4)(1\ 3)(2\ 4) = (1\ 4)(2\ 3)$より,
  $H$は$(1\ 2)(3\ 4)$と$(1\ 3)(2\ 4)$で生成される.
  また演習問題2.3.9より, $G$は$(1\ 2\ 3\ 4)$と$(1\ 2)$で生成される.
  命題2.8.7より, $h = (1\ 2)(3\ 4), (1\ 3)(2\ 4)$と$g = (1\ 2\ 3\ 4), (1\ 2)$の組み合わせについて
  $ghg^{-1} \in H$であることを見ればよい:
  \begin{align*}
    (1\ 2\ 3\ 4)(1\ 2)(3\ 4)(1\ 2\ 3\ 4)^{-1} = (1\ 4)(2\ 3) \in H\\
    (1\ 2)(1\ 2)(3\ 4)(1\ 2)^{-1} = (1\ 2)(3\ 4) \in H\\
    (1\ 2\ 3\ 4)(1\ 3)(2\ 4)(1\ 2\ 3\ 4)^{-1} = (1\ 3)(2\ 4) \in H \\
    (1\ 2)(1\ 3)(2\ 4)(1\ 2)^{-1} = (1\ 4)(2\ 3) \in H
  \end{align*}
  (5) 正規部分群である. $g = \begin{pmatrix}
    g_{11} & 0 \\
    g_{21} & g_{22}
  \end{pmatrix}$, $h = \begin{pmatrix}
    h_{11} & 0 \\
    h_{21} & h_{11}
  \end{pmatrix}$とおくと,
  $ghg^{-1}$の$(1, 1)$成分は$g_{11}h_{11}g_{11}^{-11} = h_{11}$,
  $(2, 2)$成分は$g_{22}h_{11}g_{22}^{-1} = h_{11}$であるから,
  $ghg^{-1} \in H$.
\end{ans}

\begin{ans}
  $h \in H$とする. $g \in H$ならば$ghg^{-1} \in H$は明らか.
  $g \notin H$ならば, $G = H \sqcup gH = H \sqcup Hg$より, $gH = Hg$すなわち$gHg^{-1} = H$.
  したがって任意の$g \in G$, $h \in H$に対して$ghg^{-1} \in H$.
\end{ans}

\begin{ans}
  ($N_1N_2$が部分群であること)
  $1 \in N_1N_2$.
  $N_1N_2$の任意の$2$つの元$h_1h_2, h_1^\prime h_2^\prime$ ($h_1, h_1^\prime \in N_1$かつ$h_2, h_2^\prime \in N_2$)について
  $h_1 h_2 h_1^\prime h_2^\prime = h_1 (h_2 h_1^\prime h_2^{-1}) h_2 h_2^\prime \in N_1N_2$.
  また$N_1N_2$の任意の元$h_1h_2$ ($h_1 \in N_1$, $h_2 \in N_2$)に対して$h_1h_2^{-1}h_1^{-1} \in N_2$より$(h_1h_2)^{-1} = h_2^{-1}h_1^{-1} \in N_1N_2$.\\
  (正規部分群であること) 任意の$g \in G$に対して$gN_1N_2g^{-1} = (gN_1g^{-1})(gN_2g^{-1}) \subset N_1N_2$.
\end{ans}

\begin{ans}
  $\mathfrak{S}_3$の位数は$6$なので, 部分群の位数は$1, 2, 3, 6$のいずれかである.
  位数$1$の部分群は$\{1\}$, 位数$6$の部分群は$\mathfrak{S}_3$である.
  $2$と$3$は素数なので, これらに対応する部分群は巡回群のみである.
  したがって, 部分群は
  $\{1\}$, $\mathfrak{S}_3$,
  $\gen{(1\ 2)}$, $\gen{(2\ 3)}$, $\gen{(1\ 3)}$,
  $\gen{(1\ 2\ 3)}$
  で尽くされる.
  これらの部分群のうち, $\{1\}$と$\mathfrak{S}_3$は明らかに正規部分群である.
  また$\gen{(1\ 2\ 3)}$は指数$2$の部分群であるから,
  演習問題2.8.2より正規部分群である.
  これら以外は正規部分群ではない. 例えば
  $(2\ 3)(1\ 2)(2\ 3)^{-1} = (1\ 3) \notin \gen{(1\ 2)}$
  より$\gen{(1\ 2)}$が正規部分群でないことが分かる. 他も同様.
\end{ans}

\begin{ans}
  四元数群を$G$と書くことにする.
  $G$の位数は$8$なので, 部分群の位数は$1, 2, 4, 8$のいずれかである.
  このうち位数が$1, 8$であるのは$\{1\}, G$である.
  それ以外の位数が$2$または$4$の部分群$H$について, もし$i \in H$ならば, $\gen{i} \subset H$であるが,
  $\order{\gen{i}} = 4$より$H = \gen{i}$である.
  $j, k$についても同様.
  $i, j, k \notin H$で$H \neq \{1\}$であるようなものは$H = \gen{-1}$のみである.
  以上により, $G$の部分群は$\{1\}, \gen{-1}, \gen{i}, \gen{j}, \gen{k}, G$である.
  これらの部分群のうち, 明らかに$\{1\}$と$G$は正規部分群である.
  また$\gen{i}, \gen{j}, \gen{k}$は指数$2$なので,
  演習問題2.8.2より正規部分群である.
  $\gen{-1}$が正規部分群であることも容易に確かめられる.
  以上により, $G$の部分群$\{1\}, \gen{-1}, \gen{i}, \gen{j}, \gen{k}, G$はすべて正規部分群である.
\end{ans}

\fakesubsection{2.9 群の直積}

\begin{ans}
  容易なので略.
\end{ans}

\begin{ans}
  まず$\phi_1: G_1 \rightarrow G_1$を定める. $g_1 \in G_1$に対して,
  $\phi(g_1, 1_{G_2}) = (g_1^\prime, g_2^\prime)$であるとする.
  $(g_1, 1_{G_2})$の位数は$n_1$の約数であり,
  $\phi(g_1, 1_{G_2}) = (g_1^\prime, g_2^\prime)$の位数は更にその約数であるから,
  $g_2^\prime$の位数は$n_1$の約数である.
  一方$G_2$の位数は$n_1$と互いに素であったから,
  $g_2^\prime$の位数は$1$, すなわち$g_2^\prime = 1_{G_2}$である.
  そこで$\phi_1(g_1) = g_1^\prime$として$\phi_1: G_1 \rightarrow G_1$を定めれば,
  $\phi_1$は$\phi(g_1, 1_{G_2}) = (\phi_1(g_1), 1_{G_2})$をみたす.
  このように定めた$\phi_1$が準同型であることは,
  $(\phi_1(gh), 1_{G_2})
  = \phi(gh, 1_{G_2})
  = \phi(g, 1_{G_2})\phi(h, 1_{G_2})
  = (\phi_1(g), 1_{G_2})(\phi_1(h), 1_{G_2})
  = (\phi_1(g)\phi_1(h), 1_{G_2})$
  からしたがう.
  同様にして$\phi(1_{G_1}, g_2) = (1_{G_1}, \phi_2(g_2))$をみたす
  準同型$\phi_2: G_2 \rightarrow G_2$を定めることができる.
  この$\phi_1, \phi_2$について,
  $\phi(g_1, g_2)
  = \phi(g_1, 1_{G_2})\phi(1_{G_1}, g_2)
  = (\phi_1(g_1), 1_{G_2})(1_{G_1}, \phi_2(g_2))
  = (\phi_1(g_1), \phi_2(g_2))$.
\end{ans}

\begin{ans}
  (1) $15 = 1 \cdot 8 + 7,\ 8 = 1 \cdot 7 + 1$より,
  $1 = 8 - 1 \cdot 7 = 8 - 1 \cdot (15 - 1 \cdot 8) = - 15 + 2 \cdot 8$.
  よって$(x, y) = (-1, 2)$が$15x + 8y = 1$の解の$1$つであるから,
  $15 \cdot (-1) \cdot 5 + 8 \cdot 2 \cdot 2 = -43$.
  (2) $35 = 1 \cdot 24 + 11,\ 24 = 2 \cdot 11 + 2,\ 11 = 5 \cdot 2 + 1$より,
  $1 = 11 - 5 \cdot 2
  = 11 - 5 \cdot (24 - 2 \cdot 11)
  = (-5) \cdot 24 + 11 \cdot 11
  = (-5) \cdot 24 + 11 \cdot (35 - 1 \cdot 24)
  = 11 \cdot 35 + (-16) \cdot 24$.
  よって$(x, y) = (11, -16)$が$35x + 24y = 1$の解の$1$つであるから,
  $35 \cdot 11 \cdot 5 + 24 \cdot (-16) \cdot 4 = 389$.
\end{ans}

\fakesubsection{2.10 準同型定理}

\begin{ans}
  $\phi: G \rightarrow H_2$を$\phi(re^{i\theta}) = r$と定義すると
  $\phi$は全射準同型であり, $\Ker{\phi} = H_1$である.
  したがって準同型定理より$G/H_1 \simeq H_2$.
  また$\psi: G \rightarrow H_1$を$\psi(re^{i\theta}) = e^{i\theta}$と定義すると
  $\psi$は全射準同型であり, $\Ker{\psi} = H_2$である.
  したがって準同型定理より$G/H_2 \simeq H_1$.
\end{ans}

\begin{ans}
  $\phi: \mathbb{R} \rightarrow \mathbb{R}/a\mathbb{Z}$を
  $\phi(x) = ax + a\mathbb{Z}$と定めると, $\phi$は全射準同型で
  $\Ker{\phi} = \mathbb{Z}$である. よって準同型定理より
  $\mathbb{R}/\mathbb{Z} \simeq \mathbb{R}/a\mathbb{Z}$.
\end{ans}

\begin{ans}
  $\phi: G \rightarrow \mathbb{R}^\times$を
  $\begin{pmatrix}
    a_{11} & 0 \\
    a_{21} & a_{22}
  \end{pmatrix} \mapsto \frac{a_{11}}{a_{22}}$
  と定義すると, $\phi$は全射準同型で$\Ker{\phi} = H$であるから,
  準同型定理より$G/H \simeq \mathbb{R}^\times$.
\end{ans}

\begin{ans}
  $G/H = \{H, x_1 + H, x_2 + H,..., x_{n-1} + H\}$であるとする.
  これが位数$n$の群をなすことから, $nx_i + H = H$すなわち$nx_i \in H (i = 1,..., n-1)$である.
  一方, 任意の$x \in G$はある$i$と$h \in H$について$x = x_i + h$と書けるから,
  $nx = nx_i + nh \in H$. よって$nG \subset H$.
\end{ans}

\begin{ans}
  一般に, $G$の指数$p$の部分群の数を求める.
  $H$を$G$の指数$p$の部分群とすると, 前問より$H$は$pG$を含む.
  そこで, 例題2.10.12と同様に, $H$は$G/pG$の指数$p$の部分群と1対1に対応する.
  $G/pG \simeq \Z{p} \times \Z{p}$であるから,
  $\Z{p} \times \Z{p}$の指数$p$の, すなわち位数$p$の部分群の数を求めればよい.
  この部分群は巡回部分群であり, $\langle(x, y)\rangle$ と書ける.
  $x = \overline{0}$であるようなものは, $\langle(\overline{0}, \overline{1})\rangle$のみである.
  $x \neq \overline{0}$の場合は, $x = \overline{1}$と仮定しても一般性を失わない.
  このとき$y$の取りうる値は$y = \overline{0}, \overline{1},..., \overline{p-1}$の$p$通りであるが,
  これらすべてについて, 生成される部分群が互いに異なることは明らかである.
  したがって, 指数$p$の部分群の数は$p + 1$個である.
\end{ans}

\begin{ans}
  中国剰余定理より
  $G \simeq \Z{45} \times \Z{3} \times \Z{8} \times \Z{2} \times \Z{7}$
  である. $\Z{45}$, $\Z{3}$, $\Z{7}$において$2$倍写像は全単射なので,
  $G/2G \simeq \Z{2} \times \Z{2}$である.
  したがって例題2.10.12と同様にして, $G$の指数$2$の部分群の個数は$3$である.
\end{ans}

\begin{ans}
  写像$f: G/H \rightarrow (G/N)/(H/N)$を,
  $aH \mapsto (aN)(H/N)$と定めたい.
  \begin{align*}
    &(aN)(H/N) = (bN)(H/N) \\
    &\Leftrightarrow (b^{-1}N)(aN) \in H/N \\
    &\Leftrightarrow b^{-1}aN \in H/N \\
    &\Leftrightarrow b^{-1}a \in H \\
  \end{align*}
  であることから, $f$がwell-definedかつ単射であることがわかる.
  また明らかに$f$は全射なので, $f$は全単射.
\end{ans}

\begin{ans}
  (1) $G = \Z{12}$とおく. $\abs{G} = 12$なので, 部分群の位数の候補は$1, 2, 3, 4, 6, 12$である.
  これらのうち, $1, 12$にはそれぞれ自明な部分群$\{0\}, G$が対応する.
  また, 中国剰余定理より$G \simeq \Z{3} \times \Z{4}$である.
  以下, $G$の部分群$H$の位数が$2, 3, 4, 6$の場合を考える.\\
  \underline{位数$2$の場合}:
  $H$は巡回群で, ある元$(\overline{a}, \overline{b}) \in \Z{3} \times \Z{4}$で生成される.
  $2\overline{a} = \overline{0}\ (\in \Z{3})$より$\overline{a} = \overline{0}$.
  よって$\overline{b} \in \Z{4}$の位数が$2$なので,
  $(\overline{a}, \overline{b}) = (\overline{0}, \overline{2})$.
  逆に, $\langle (\overline{0}, \overline{2}) \rangle$は位数$2$の部分群なので,
  これが位数$2$の唯一の部分群である.\\
  \underline{位数$3$の場合}:
  $H$は巡回群で, ある元$(\overline{a}, \overline{b}) \in \Z{3} \times \Z{4}$で生成される.
  $3\overline{b} = \overline{0}\ (\in \Z{4})$より$\overline{b} = \overline{0}$.
  よって$H$は$\Z{3}$の部分群と同型であるが, 位数の比較により, $\Z{3}$自体と同型である.
  すなわち位数$3$の部分群は$\langle (\overline{1}, \overline{0}) \rangle$のみである.\\
  \underline{位数$4$の場合}:
  $(\overline{a}, \overline{b}) \in H$とすると,
  $4\overline{a} = \overline{0}\ (\in \Z{3})$より$\overline{a} = 0$.
  あとは位数$3$の場合と同様の議論により, 位数$4$の部分群は$\langle (\overline{0}, \overline{1}) \rangle$のみである.\\
  \underline{位数$6$の場合}:
  $(\overline{a}, \overline{b}) \in H$とすると,
  $6\overline{b} = \overline{0}\ (\in \Z{4})$より$\overline{a} = \overline{0}, \overline{2}$.
  これをみたす$(\overline{a}, \overline{b})$は$6$組しかないから, これらが部分群$H$をなすはずである.
  実際, $\langle (\overline{1}, \overline{2}) \rangle$が位数$6$の部分群である.\\
  (2) $G = \Z{18}$とおく. $\abs{G} = 18$なので, 部分群の位数の候補は$1, 2, 3, 6, 9, 18$である.
  これらのうち, $1, 18$にはそれぞれ自明な部分群$\{0\}, G$が対応する.
  また, 中国剰余定理より$G \simeq \Z{2} \times \Z{9}$である.
  以下, $G$の部分群$H$の位数が$2, 3, 4, 6$の場合を考える.\\
  \underline{位数$2$の場合}:
  (1) の位数$3$の場合と同様で, $\langle (\overline{1}, \overline{0}) \rangle$のみ.\\
  \underline{位数$3$の場合}:
  (1) の位数$2$の場合と同様で, $\langle (\overline{0}, \overline{3}) \rangle$のみ.\\
  \underline{位数$6$の場合}:
  (1) の位数$6$の場合と同様で, $\langle (\overline{1}, \overline{3}) \rangle$のみ.\\
  \underline{位数$9$の場合}:
  (1) の位数$4$の場合と同様で, $\langle (\overline{0}, \overline{1}) \rangle$のみ.\\
  以上を一般化すると, $p, q$を素数として$\Z{p^2q}$の部分群を求める問題となる. (追記するかも.)
\end{ans}

\begin{ans}
  (1) $G$の元の位数の候補は$1, 2, 3, 6$である.
  もし位数$6$の元が存在すれば$G$は巡回群$\Z{6}$に同型であり,
  $\overline{2}$ (に同型で対応する元) が位数$3$である.
  そこで位数$6$の元が存在しない場合を考える.
  この場合に位数$3$の元が存在することを背理法で示そう.
  もし位数$3$の元が存在しなければ, 単位元以外の元はすべて位数$2$なので,
  演習問題2.4.8より$G$は可換である. よって$x$を単位元以外の元とすると
  $H = \langle x \rangle$は$G$の位数$2$の正規部分群であり, $G/H$は位数$3$の巡回群である.
  そこで$G/H$の生成元である剰余類の代表元を$g$とすると,
  $gHgH \neq H$より$(g)^2 \notin H$であるが,
  このことから$g$の位数が$3$以上であることになり, 矛盾.
  よって$G$に位数$6$の元が存在しない場合にも, 位数$3$の元が存在することが分かった.\\
  (2) $G$にもし位数$6$の元が存在すれば, (1) と同様に, $\overline{3}$が位数$2$である.
  そこで位数$6$の元が存在しない場合を考える.
  演習問題2.8.2より, $H$は$G$の正規部分群であることに注意する.
  位数$2$の巡回群$G/H$の生成元である剰余類の代表元を$g$とすると,
  $gHgHgH \neq H$より$g^3 \notin H$であるが,
  このことから$g$の位数は$3$ではない. $gH$が$G/H$の生成元なので, $g$の位数は$1$でもない.
  したがって$g$の位数は$2$である.\\
  (3) $x$を$G$の位数$3$の元, $y$を位数$2$の元とする.
  $H = \langle x \rangle$は (2) で見たように正規部分群であるが,
  $G$が可換なら$K = \langle y \rangle$も正規部分群である.
  また, 各元の位数を比較することにより$H \cap K = 1_G$である.
  $h_1, h_2 \in H$, $k_1, k_2 \in K$について, もし$h_1k_1 = h_2k_2$ならば$h_2^{-1}h_1 = k_2k_1^{-1} = 1_G$より
  $h_1 = h_2$かつ$k_1 = k_2$である. したがって$\abs{HK} = \abs{H}\abs{K} = 6$なので$HK = G$.
  命題2.9.2より$G \simeq H \times K \simeq \Z{3} \times \Z{2} \simeq \Z{6}$.\\
  (4) $G$が非可換であるとする. $x$を位数$3$の元, $H = \langle x \rangle$として, $y$を$G/H$の生成元の代表元とする.
  (2) でみたように, $y$は位数$2$の元である.
  また$G$の任意の元は, $H$の生成元$x$を用いて$y^kx^l\ (k = 0, 1,\ l = 0, 1, 2)$と書ける.
  もし$y$が$x$と可換であれば, $y^{k_1}x^{l_1}y^{k_2}x^{l_2} = y^{k_2}x^{l_2}y^{k_1}x^{l_1}$であるから,
  $G$は可換となり矛盾. したがって$y$と$x$は非可換である.
  また, $(x^2)^2 = x$より, $y$は$x^2$とも非可換である. (もし$y$が$x^2$と可換なら$x = (x^2)^2$とも可換となって矛盾.)
  そこで, $y, xyx^{-1}, x^2yx^{-2}$を考えると,
  $y$と$x$が非可換かつ$y$と$x^2$が非可換であることからこれらは互いに異なる元であり,
  また明らかに互いに共役である.
  さらに$y$が位数$2$であることからこれらは単位元とは異なる元であり, $(x^kyx^{-k})^2 = x^ky^2k^{-k} = 1_G$より位数は$2$である.
  $G$にはこれら$3$つの元の他には$H$の元 (位数は$1$か$3$) しかないので, 位数$2$の元はこの$3$つだけである.\\
  (5) $\rho(g)(i) = \rho(g)(j) \Rightarrow gx_ig^{-1} = gx_jg^{-1} \Rightarrow x_i = x_j$であるから,
  $\rho(g)$は単射. したがって$\rho(g) \in \mathfrak{S}_3$.
  また$x_{\rho(gh)(i)}
  = ghx_i(gh)^{-1}
  = gx_{\rho(h)(i)}g^{-1}
  = x_{\rho(g)(\rho(h)(i))}
  = x_{(\rho(g)\rho(h))(i)}$
  より, $\rho(gh) = \rho(g)\rho(h)$. すなわち$\rho$は準同型である.
  もし$\rho(g) = 1_{\mathfrak{S}_3}$ならば$gx_ig^{-1} = x_i\ (i = 1, 2, 3)$
  すなわち$g$と$x_i$は可換であるが, (4) で見たように$y$は$x, x^2$と非可換なので$g$は$x, x^2$とは異なる元である.
  また, $x_1, x_2, x_3$は$yH$の元なので$y, yx, yx^2$と書けるはずであるが,
  これらは$y$が$x, x^2$と非可換であることから互いに非可換である. よって$g$は$x_1, x_2, x_3$のいずれとも異なる.
  したがって$g = 1_G$であるから, $\rho$は単射. $G$と$\mathfrak{S}_3$の位数が等しいことから, $\rho$は同型.
\end{ans}

\fakesection{第3章 群を学ぶ理由}
\fakesection{第4章 群の作用}

\fakesubsection{4.1 群の作用}

\begin{ans}
  $x_2x_1 = x_2,\ x_2x_2 = x_1,\ x_2x_3 = x_4,\ x_2x_4 = x_3$
  より, $\rho(x_2) = (1\ 2)(3\ 4)$.
  同様にして, $\rho(x_3) = (1\ 3)(2\ 4)$, $\rho(x_4) = (1\ 4)(2\ 3)$.
\end{ans}

\begin{ans}
  \begin{align*}
    &(2\ 3)1 = (2\ 3)\\
    &(2\ 3)(1\ 2) = (1\ 3\ 2)\\
    &(2\ 3)(1\ 3) = (1\ 2\ 3)\\
    &(2\ 3)(2\ 3) = 1\\
    &(2\ 3)(1\ 2\ 3) = (1\ 3)\\
    &(2\ 3)(1\ 3\ 2) = (1\ 2)\\
  \end{align*}
  となるので$\rho((2\ 3)) = \begin{pmatrix*}
    1 & 2 & 3 & 4 & 5 & 6\\
    4 & 6 & 5 & 1 & 3 & 2
  \end{pmatrix*} = (1\ 4)(2\ 6)(3\ 5)$.
\end{ans}

\begin{ans}
  \begin{align*}
    &(1\ 3\ 2)x_1 = (1\ 3\ 2) \in x_3H\\
    &(1\ 3\ 2)x_2 = 1 \in x_1H\\
    &(1\ 3\ 2)x_3 = (1\ 2\ 3)
  \end{align*}
  となるので$\rho((1\ 3\ 2)) = (1\ 3\ 2)$.
\end{ans}

\begin{ans}
  \begin{align*}
    &\Ad{(1\ 2\ 3)}(1) = 1\\
    &\Ad{(1\ 2\ 3)}((1\ 2)) = (1\ 2\ 3)(1\ 2)(1\ 3\ 2) = (2\ 3)\\
    &\Ad{(1\ 2\ 3)}((1\ 3)) = (1\ 2\ 3)(1\ 3)(1\ 3\ 2) = (1\ 2)\\
    &\Ad{(1\ 2\ 3)}((2\ 3)) = (1\ 2\ 3)(2\ 3)(1\ 3\ 2) = (1\ 3)\\
    &\Ad{(1\ 2\ 3)}((1\ 2\ 3)) = (1\ 2\ 3)\\
    &\Ad{(1\ 2\ 3)}((1\ 3\ 2)) = (1\ 3\ 2)
  \end{align*}
  となるので$\rho((1\ 2\ 3)) = (2\ 4\ 3)$.
\end{ans}

\begin{ans}
  (1)
  \begin{align*}
    &ix_1 = i \cdot 1 = i = x_3\\
    &ix_2 = i \cdot (-1) = -i = x_4\\
    &ix_3 = i \cdot i = -1 = x_2\\
    &ix_4 = i \cdot (-i) = 1 = x_1\\
    &ix_5 = i \cdot j = k = x_7\\
    &ix_6 = i \cdot (-j) = -k = x_8\\
    &ix_7 = i \cdot k = -j = x_6\\
    &ix_8 = i \cdot (-k) = j = x_5
  \end{align*}
  となるので$\rho(i) = (1\ 3\ 2\ 4)(5\ 7\ 6\ 8)$.\\
  (2)
  \begin{align*}
    &kx_1 = k \cdot 1 = k = x_7\\
    &kx_2 = k \cdot (-1) = -k = x_8\\
    &kx_3 = k \cdot i = j = x_5\\
    &kx_4 = k \cdot (-i) = -j = x_6\\
    &kx_5 = k \cdot j = -i = x_4\\
    &kx_6 = k \cdot (-j) = i = x_3\\
    &kx_7 = k \cdot k = -1 = x_2\\
    &kx_8 = k \cdot (-k) = 1 = x_1
  \end{align*}
  となるので$\rho(k) = (1\ 7\ 2\ 8)(3\ 5\ 4\ 6)$.
\end{ans}

\begin{ans}
  (1) (本の解答例に比べるとかなり拙い解法.)
  \begin{align*}
    &yy^{-1} = 1\\
    &yxy^{-1} = x^3\\
    &yx^2y^{-1} = (yxy^{-1})^2 = (x^3)^2 = x^6\\
    &yx^3y^{-1} = (yxy^{-1})^3 = (x^3)^3 = x^9 = x^2\\
    &yx^4y^{-1} = (yxy^{-1})^4 = (x^3)^4 = x^{12} = x^5\\
    &yx^5y^{-1} = (yxy^{-1})^5 = (x^3)^5 = x^{15} = x\\
    &yx^6y^{-1} = (yxy^{-1})^6 = (x^3)^6 = x^{18} = x^4\\
  \end{align*}
  となるので, $\Ad{y}$の$\langle x \rangle$への作用は置換
  $\sigma = (1\ 3\ 2\ 6\ 4\ 5)$で表される. (ただしここでは$x^i = i$と考えている.)
  $\sigma^{100}(1) = \sigma^{6 \cdot 16 + 4}(1) = \sigma^{4}(1) = 4$なので, $y^{100}xy^{-100} = x^4$.\\
  (2)
  \begin{align*}
    &yy^{-1} = 1\\
    &yxy^{-1} = x^5\\
    &yx^2y^{-1} = (yxy^{-1})^2 = (x^5)^2 = x^{10} = x^3\\
    &yx^3y^{-1} = (yxy^{-1})^3 = (x^5)^3 = x^{15} = x\\
    &yx^4y^{-1} = (yxy^{-1})^4 = (x^5)^4 = x^{20} = x^6\\
    &yx^5y^{-1} = (yxy^{-1})^5 = (x^5)^5 = x^{25} = x^4\\
    &yx^6y^{-1} = (yxy^{-1})^6 = (x^5)^6 = x^{30} = x^2\\
  \end{align*}
  となるので, $\Ad{y}$の$\langle x \rangle$への作用は置換
  $\sigma = (1\ 5\ 4\ 6\ 2\ 3)$で表される. (ただしここでは$x^i = i$と考えている.)
  $\sigma^{1000}(1) = \sigma^{6 \cdot 166 + 4}(1) = \sigma^{4}(1) = 2$なので, $y^{1000}xy^{-1000} = x^2$.\\
\end{ans}

\begin{ans}
  $n = 1$の場合は問題の主張は成り立たないので, $n \ge 2$とする.
  $G$のある元$g$について
  $\norm{\bm{x}} = \norm{\bm{y}}$かつ
  $g\frac{\bm{x}}{\norm{\bm{x}}} = \frac{\bm{y}}{\norm{\bm{y}}}$
  ならば
  $g\bm{x} = \bm{y}$
  なので, はじめから$\norm{\bm{x}} = \norm{\bm{y}} = 1$であるとして一般性を失わない.
  また$\bm{x} = [1, 0,..., 0]$に対して$g_1\bm{x} = \bm{y}$かつ$g_2\bm{x} = \bm{y}^\prime$ならば$g_2g_1^{-1}\bm{y} = \bm{y}^\prime$なので,
  $\bm{x} = [1, 0,..., 0]$の場合を示せばよい.
  $\bm{y_1} = \bm{y}$とし, $\bm{y_2},..., \bm{y_n}$を, Gram-Schmidtの直交化により
  $\{\bm{y_1},.., \bm{y_n}\}$が$V$の正規直交基底であるものとしてとる.
  さらに$\bm{y_2}$の符号を必要ならば入れ替えることにして
  $g = (\bm{y_1},..., \bm{y_n})$と定めると, $g \in \SO{n}$かつ$g\bm{x} = \bm{y_1} = \bm{y}$である.
\end{ans}

\begin{ans}
  (1) $\sigma((2, 4)) = (\sigma(2), \sigma(4)) = (1, 4)$.\\
  (2) 任意の$\sigma \in G$に対して$\sigma((1, 1)) = (\sigma(1), \sigma(1))$なので,
  $(1, 1)$の軌道は$(i, i)$の形をしている. 逆に任意の$i \in X$に対して$\sigma = (1\ i)$と定めれば
  $\sigma((1, 1)) = (i, i)$なので, $(1, 1)$の軌道は$\{(i, i) \mid i \in X\}$である.
  他の軌道として$(1, 2)$の軌道を考えると, 任意の$\sigma \in G$に対して$\sigma(1) \neq \sigma(2)$なので,
  $(1, 2)$の軌道は$(i, j)$ ($i \neq j$) の形をしている. 逆に$i, j \in X, i \neq j$に対して適当に$\sigma \in G$を定めて
  $\sigma(1) = i$, $\sigma(2) = j$となるようにできて,$\sigma((1, 2)) = (i, j)$となるので,
  $(1, 2)$の軌道は$\{(i, j) \mid i, j \in X, i \neq j\}$である.
  以上の$2$つの軌道で$Y$の元は尽くされるので, これらが$Y$における$G$の軌道の全てである.\\
  (3) $\sigma((1, 1)) = (1, 1) \Longleftrightarrow \sigma(1) = 1$なので, $(1, 1)$の安定化群は$\mathfrak{S}_{n-1}$.
  また$\sigma((1, 2)) = (1, 2) \Longleftrightarrow \sigma(1) = 1 \land \sigma(2) = 2$なので, $(1, 2)$の安定化群は$\mathfrak{S}_{n-2}$.
  ($n = 2$のとき$\mathfrak{S}_0 = \{1\}$である.)
\end{ans}

\begin{ans}
  (1) $g\bm{x} = \bm{x}$なる$g \in G$は, $[1, 0]$を$[1, 0]$に, $[0, 1]$を$[1, 0]$と線形独立なベクトルに移す行列であるから,
  $G_{\bm{x}} = \biggl\{\begin{pmatrix*}
    1 & a_{12}\\
    0 & a_{22}
  \end{pmatrix*} \biggm\vert a_{22} \neq 0\biggr\}$.\\
  (2) 任意の$g \in G$に対して$g\bm{x} \neq [0, 0]$なので,
  $[0, 0]$は$g$の軌道に属さない.
  逆に, $[a_{11}, a_{21}] \neq [0, 0]$ならば
  $g = \begin{pmatrix*}
    a_{11} & -a_{21}\\
    a_{21} & a_{11}
  \end{pmatrix*}$とすると
  $g \in G$で$g\bm{x} = [a_{11}, a_{21}]$なので,
  $[0, 0]$以外の点はすべて$\bm{x}$の軌道に属する.
  したがって$G \cdot \bm{x} = \{y \mid y \in \mathbb{R}^2\setminus[0, 0]\}$.
\end{ans}

\begin{ans}
  (1) $1$を含む共役類は明らかに$\{1\}$である.
  また$-1$も他の元と可換なので, $-1$を含む共役類は$\{-1\}$.
  また$i$については$ij = j \cdot (-i)$, $ik = k \cdot (-i)$などから$i$を含む共役類は$\{\pm i\}$.
  同様に, 他の共役類は$\{\pm j\}$, $\{\pm k\}$である.\\
  (2) $1$はすべての元と可換なので中心化群は$G$.
  $i$と可換であるのは$\{\pm 1, \pm i\}$だけなので、これが$i$の中心化群である.
\end{ans}

\begin{ans}
  (1) $\sigma$は
  $A_1 \rightarrow A_2 \rightarrow A_3 \rightarrow A_4 \rightarrow A_5 \rightarrow A_6 \rightarrow A_7 \rightarrow A_8 \rightarrow A_1$
  と作用するので,
  $\sigma l_1 = l_2$, $\sigma l_2 = l_3$, $\sigma l_3 = l_4$, $\sigma l_4 = l_1$である.
  よって$\rho(\sigma) = (1\ 2\ 3\ 4)$.
  $\tau$は$A_1 \rightarrow A_1$,
  $A_2 \rightarrow A_8 \rightarrow A_2$,
  $A_3 \rightarrow A_7 \rightarrow A_3$,
  $A_4 \rightarrow A_6 \rightarrow A_4$,
  $A_5 \rightarrow A_5$と作用するので,
  $\sigma l_1 = l_1$, $\sigma l_2 = l_4$, $\sigma l_3 = l_3$, $\sigma l_4 = l_2$である.
  よって$\rho(\tau) = (2\ 4)$.\\
  (2) $D_8 \ni g$で$l_1$が不変であることは, $g$によって$A_1$が$A_1$または$A_5$に移ることと同値である.
  $A_1$が$A_1$に移すのは$1, \tau$であり, $A_1$を$A_5$に移すのは$\sigma^4, \tau\sigma^4$なので,
  $l_1$の安定化群は$\{1, \sigma^4, \tau, \tau\sigma^4\}$.
\end{ans}

\begin{ans}
  命題4.1.10の表記に倣って, $D_n = \{1, t, \cdots , t^{n-1}, r, rt, \cdots , rt^{n-1}\}$と書くことにする.\\
  (a)(1)
  \begin{align*}
    &t^lt^k(t^l)^{-1} = t^k\\
    &rt^lt^k(rt^l)^{-1} = rt^lt^kt^{-l}r = rt^kr = t^{-k}\\
    &t^lrt^k(t^l)^{-1} = rt^{-l}t^kt^{-l} = rt^{k-2l}\\
    &rt^lrt^k(rt^l)^{-1} = rt^lrt^kt^{-l}r = rt^lt^{-k+l} = rt^{-k+2l}
  \end{align*}
  より, $D_4$の共役類は
  $\{
    \{1\},
    \{t, t^3\},
    \{t^2\},
    \{r, rt^2\},
    \{rt, rt^3\}
  \}$
  である.\\
  (a)(2) $\{1\}$, $\{t^2\}$については, 明らかに中心化群は$D_4$である.
  $t$については, 上の式から$\{1, t, t^2, t^3\}$が中心化群である (代表元として$t^3$を選んでも同じ).
  $r$については, 上の$3$つ目の式で$l = 0, 2$, 上の$4$つ目の式で$l = 0, 2$の場合が対応するので,
  中心化群は$\{1, t^2, r, rt^2\}$である (代表元として$rt^2$を選んでも同じ).
  $rt$については, 上の$3$つ目の式で$l = 0, 2$, 上の$4$つ目の式で$l = 1, 3$の場合が対応するので,
  中心化群は$\{1, t^2, rt, rt^3\}$である (代表元として$rt^3$を選んでも同じ).\\
  (b)(1)
  上の$4$つの式から, $D_5$の共役類は
  $\{
    \{1\},
    \{t, t^4\},
    \{t^2, t^3\},
    \{r, rt, rt^2, rt^3, rt^4\}
  \}$
  である.\\
  (b)(2) $\{1\}$については, 明らかに中心化群は$D_5$である.
  $D_5$の元については任意の$k$に対して$t^k \neq t^{-k}$なので,
  $\{t, t^4\}, \{t^2, t^3\}$については,
  どの代表元を選んでも中心化群は$\{1, t, t^2, t^3, t^4\}$である.
  $rt^k$の形の元については, 上の$3$つ目の式が$r^k$に等しくなるのは$k - 2l \equiv k\ (\mathrm{mod}\ 5)$すなわち$l = 0$の場合のみ,
  上の$4$つ目の式が$r^k$に等しくなるのは$-k + 2l \equiv k\ (\mathrm{mod}\ 5)$すなわち$k + l \equiv 0\ (\mathrm{mod}\ 5)$の場合のみであるので,
  $r$の中心化群は$\{1, r\}$である. 代表元として他の$rt^k$をとると, $\{1, rt^{5-k}\}$が中心化群となる.\\
  (考察1) ある共役類が$1$つの元のみからなるならば, その元は$G$のすべての元と可換であるから, 中心化群は$G$となる.\\
  (考察2) 共役類が$2$つの元から成るならば, $2$つの元それぞれの中心化群は等しい.
  このことを以下で示そう.
  共役類$\{g_1, g_2\}$を考えているとすると,
  任意の$g \in Z_G(g_1)$について, $g$の共役による作用によって$g_2$は共役類の元$g_1, g_2$のいずれかに移るはずであるが, $g_1$に移ることはありえない.
  なぜなら, $g^{-1} \in Z_G(g_1)$であって$g^{-1}$の共役による作用で$g_1$が$g_2$に移ることはないからである.
  したがって$g \in Z_G(g_2)$なので$Z_G(g_1) \subset Z_G(g_2)$. 逆も同様に成り立つので$Z_G(g_1) = Z_G(g_2)$である.
\end{ans}

\begin{ans}
  (1)
  \begin{align*}
    \begin{pmatrix*}
      a & b \\
      c & d
    \end{pmatrix*} A = \begin{pmatrix*}
      2a & b \\
      2c & d
    \end{pmatrix*},\ A \begin{pmatrix*}
      a & b \\
      c & d
    \end{pmatrix*} = \begin{pmatrix*}
      2a & 2b \\
      c & d
    \end{pmatrix*}
  \end{align*}
  であるから, 両者が等しいことは$b = 0$かつ$c = 0$であることと同値である.
  よって$A$の中心化群は$\biggl\{\begin{pmatrix*}
    a & 0 \\
    0 & d \\
  \end{pmatrix*} \biggm\vert a, d \neq 0 \biggr\}$である.\\
  (2)
  \begin{align*}
    \begin{pmatrix*}
      a & b \\
      c & d
    \end{pmatrix*} A = \begin{pmatrix*}
      2a & a + 2b \\
      2c & c + 2d
    \end{pmatrix*},\ A \begin{pmatrix*}
      a & b \\
      c & d
    \end{pmatrix*} = \begin{pmatrix*}
      2a + c & 2b + d \\
      2c & 2d
    \end{pmatrix*}
  \end{align*}
  であるから, 両者が等しいことは$c = 0$かつ$a = d$であることと同値である.
  よって$A$の中心化群は$\biggl\{\begin{pmatrix*}
    a & b \\
    0 & a \\
  \end{pmatrix*} \biggm\vert a \neq 0 \biggr\}$である.
\end{ans}

\begin{ans}
  (1) 問題文に書いていないが$g \in G$とする.
  $c \neq 0$ならば$\mathrm{Im}(cz + d) \neq 0$より$cz + d \neq 0$である.
  $c = 0$ならば$g \in \mathrm{SL}_2(\mathbb{R})$より$d \neq 0$であるのでやはり$cz + d \neq 0$.
  また,
  \begin{align*}
    \frac{az + b}{cz + d}
    = \frac{(az + b)(c\bar{z} + d)}{(cz + d)(c\bar{z} + d)}
    = \frac{ac\abs{z}^2 + bd + adz + bc\bar{z}}{\abs{cz + d}^2}
  \end{align*}
  より, $\mathrm{Im}(gz) = \frac{\mathrm{Im}(z)(ad - bc)}{\abs{cz + d}^2} > 0$.
  よって$gz \in \mathbb{H}$.\\
  (2) $g = 1_G = \begin{pmatrix*}
    1 & 0 \\
    0 & 1
  \end{pmatrix*}$ならば$gz = z$.\\
  また, $G$の任意の$2$つの元を$g = \begin{pmatrix*}
    a & b \\
    c & d
  \end{pmatrix*}$, $h = \begin{pmatrix*}
    k & l \\
    m & n
  \end{pmatrix*}$とおくと,
  \begin{align*}
    g(hz) &= g\biggl(\frac{kz + l}{mz + n}\biggr)
    = \frac{a(kz + l) + b(mz + n)}{c(kz + l) + d(mz + n)}
    = \frac{(ak + bm)z + (al + bn)}{(ck + dm)z + (cl + dn)}\\
    (gh)z &= \begin{pmatrix*}
      ak + bm & al + bn \\
      ck + dm & cl + dn
    \end{pmatrix*}z = \frac{(ak + bm)z + (al + bn)}{(ck + dm)z + (cl + dn)}
  \end{align*}
  よって$g(hz) = (gh)z$.\\
  (3) $\mathbb{H}$の任意の元$x + yi$ ($y > 0$)に対して,
  $g = \begin{pmatrix*}
    \sqrt{y} & \frac{x}{\sqrt{y}} \\
    0 & \frac{1}{\sqrt{y}}
  \end{pmatrix*}$と定めると, $g \in G$であり,
  $gi = \frac{\sqrt{y} \sqrt{-1} + \frac{x}{\sqrt{y}}}{\frac{1}{\sqrt{y}}} = x + y\sqrt{-1}$.
  よって$G \cdot \sqrt{-1} = \mathbb{H}$.\\
  (4) $g \sqrt{-1} = \sqrt{-1}
  \Leftrightarrow \frac{a\sqrt{-1} + b}{c\sqrt{-1} + d} = \sqrt{-1}
  \Leftrightarrow (a - d)\sqrt{-1} + (b + c) = 0
  \Leftrightarrow g = \begin{pmatrix*}
    a & b \\
    -b & a
  \end{pmatrix*}$.
  また$\det g = 1$より, 安定化群は$\biggl\{\begin{pmatrix*}
    \cos\theta & \sin\theta \\
    -\sin\theta & \cos\theta
  \end{pmatrix*} \biggm\vert \theta \in [0, 2\pi) \biggr\}$.
\end{ans}

\begin{ans}
  (1) $(\rho(1_G)f)(h) = f(1_Gh) = f(h)$より, $\rho(1_G)f = f$.
  また, $(\rho(g_1g_2)f)(h) = f(g_1g_2h) = (\rho(g_1)f)(g_2h) = (\rho(g_2)(\rho(g_1)f))(h)$
  より, $\rho$は右作用である.\\
  (2) $(\rho(1_G)f)(h) = f(h1_G^{-1}) = f(h)$より, $\rho(1_G)f = f$.
  また, $(\rho(g_1g_2)f)(h) = f(hg_2^{-1}g_1^{-1}) = (\rho(g_1)f)(hg_2^{-1}) = (\rho(g_2)(\rho(g_1)f))(h)$
  より, $\rho$は右作用である.\\
  (3) $(\rho(1_G)f)(h) = f(1_G^{-1}h1_G) = f(h)$より, $\rho(1_G)f = f$.
  また, $(\rho(g_1g_2)f)(h) = f(g_2^{-1}g_1^{-1}hg_1g_2) = (\rho(g_2)f)(g_1^{-1}hg_1) = (\rho(g_1)(\rho(g_2)f))(h)$
  より, $\rho$は左作用である.\\
  (補足) 一般に, 左作用$\phi: G \times X \rightarrow X$があるとき,
  $\psi: G \times X \rightarrow X$を$\psi(g, x) = \phi(g^{-1}, x)$と定めると,
  \begin{align*}
    \psi(1_G, x) &= \phi(1_G, x) \\
    &= x \\
    \psi(g, \psi(h, x)) &= \phi(g^{-1}, \phi(h^{-1}, x)) \\
    &= \phi(g^{-1}h^{-1}, x) \\
    &= \psi(hg, x)
  \end{align*}
  より, $\psi$は右作用である. 左右を逆にしても同様のことが言える.
\end{ans}

\begin{ans}
  $G$が推移的に作用するとし, $G \cdot x = \{1, \cdots , n\}$であるとする. ($x$は実のところ何でもよい.)
  このとき命題4.1.24より, $\abs{G} = \abs{G \cdot x}\abs{G_x} = n\abs{G_x}$.
\end{ans}

\begin{ans}
  (1) $N$は巡回群であるから, $N$の生成元を$1$つ固定して,
  その元がどの生成元に移るかを定めることによって, $\mathrm{Aut}N$の元が定まる.
  逆に, $\mathrm{Aut}N$の元は$N$の生成元を生成元に移すものであるから, $\abs{\mathrm{Aut}N}$は生成元の個数, すなわち$16$に等しい.\\
  (2) $n \mapsto gng^{-1}$は$N$の自己同型を与えるが, この形の自己同型全体は, $\mathrm{Aut}N$の部分群をなすことに注意する.
  この部分群は$G$の内部自己同型 (inner automorphism) 全体のなす群において$N$上で等しいものを同一視したものであるから,
  以後$\mathrm{Inn}_{G, N}$と書くことにしよう.
  (1) より$\abs{\mathrm{Aut}N} = 16$であったから, $\abs{\mathrm{Inn}_{G, N}}$は$2$の冪である.
  よって任意に$n_0 \in N$をとると, $\mathrm{Inn}_{G, N}$の$N$への作用$(\phi, n) \mapsto \phi(n)$による$n_0$の軌道の濃度は,
  $\abs{\mathrm{Inn}_{G, N}}$の約数であり, これもまた$2$の冪である.
  一方, この作用は$G$の$N$への共役による作用とみなせるから, $n_0$の軌道の濃度は$\abs{G}$の約数でもあり, 奇数である.
  したがって$n_0$の軌道の濃度は$1$. すなわち任意の$g \in G$に対して$gn_0g^{-1} = n_0$
  このことは, $N$が$G$の中心に含まれることを意味する.\\
  ((2) の別解) 準同型$\varphi: G \rightarrow \mathrm{Aut}N$を$g \mapsto (n \mapsto gng^{-1})$で定める.
  $\mathrm{Im}\varphi$は$\mathrm{Aut}N$の部分群なので位数は$2$の冪である.
  一方で準同型定理より$\mathrm{Im}\varphi$の位数は$G$の位数の約数なので奇数である.
  したがって$\abs{\mathrm{Im}\varphi} = 1$なので, $\mathrm{Im}\varphi = \{\mathrm{Id}_N\}$.
  すなわち任意の$g \in G$, $n \in N$にたいして$gng^{-1} = n$が成り立つので, $N$は$G$の中心に含まれる.
\end{ans}

\begin{ans}
  (1) $3$は$8$の約数でない, (4) $1$の数が$8$の約数でない
\end{ans}

\fakesubsection{4.2 対称群の共役類}

\begin{ans}
  容易なので略.
\end{ans}

\begin{ans}
  (1) $n = 5$のヤング図形は以下の$7$つである.
  \begin{align*}
    \ytableausetup{smalltableaux}
    \ydiagram{5}\ \ydiagram{4, 1}\ \ydiagram{3, 2}\ \ydiagram{3, 1, 1}\ \ydiagram{2, 2, 1}\ \ydiagram{2, 1, 1, 1}\ \ydiagram{1, 1, 1, 1, 1}
  \end{align*}
  共役類の代表元としては, これらのヤング図形の長さ$2$以上の行に適当に数を当てはめたものをとればよい.
  たとえば, $(1\ 2\ 3\ 4\ 5)$, $(1\ 2\ 3\ 4)$, $(1\ 2\ 3)(4\ 5)$, $(1\ 2\ 3)$, $(1\ 2)(3\ 4)$, $(1\ 2)$, $1$
  が代表元である.\\
  (2) 共役類の中では$\mathrm{sgn}$は一致すること,
  また$A_5$の共役類は$\mathfrak{S}_5$の共役類と一致するか, またはその細分であることに注意する.
  このことから, $A_5$の共役類を求めるためには, $\mathfrak{S}_5$の共役類のうち$A_5$に含まれるもののそれぞれが, $A_5$の共役類として細分されているかどうか,
  もし細分されているならどのように細分されているかを調べればよい.
  (1) に挙げた代表元のうち, $A_5$に含まれるのは$(1\ 2\ 3\ 4\ 5)$, $(1\ 2\ 3)$, $(1\ 2)(3\ 4)$, $1$の$4$つである.

  \underline{$(1\ 2\ 3\ 4\ 5)$について.}
  $\sigma \in A_5$が$\mathfrak{S}_5$において$(1\ 2\ 3\ 4\ 5)$と共役であるとすると,
  ある$\tau \in \mathfrak{S}_5$について$\sigma = \tau(1\ 2\ 3\ 4\ 5)\tau^{-1}$.
  $\tau$が偶置換ならば$\sigma$は$A_5$において$(1\ 2\ 3\ 4\ 5)$と共役,
  $\tau$が奇置換ならば$\sigma = \tau(1\ 2)(2\ 1\ 3\ 4\ 5)(1\ 2)\tau^{-1}$より$\sigma$は$A_5$において$(2\ 1\ 3\ 4\ 5)$と共役である.
  すなわち$\sigma$は$(1\ 2\ 3\ 4\ 5)$と$(2\ 1\ 3\ 4\ 5)$のいずれかと共役.

  一方, $(1\ 2\ 3\ 4\ 5)$の$\mathfrak{S}_5$における共役類と, $A_5$における共役類が一致しないことが以下のようにして示せる.
  $A_5$の共役による作用について, $(1\ 2\ 3\ 4\ 5)$の安定化群を$H$とする.
  明らかに, $\langle(1\ 2\ 3\ 4\ 5)\rangle \subset H$である.
  よって, $A_5$における$(1\ 2\ 3\ 4\ 5)$の共役類のサイズは$\abs{A_5}/\abs{H} \le \abs{A_5}/\abs{\langle(1\ 2\ 3\ 4\ 5)\rangle} = 60 / 5 = 12$よりたかだか$12$である.
  一方, $\mathfrak{S}_5$における$(1\ 2\ 3\ 4\ 5)$の共役類のサイズは$4! = 24$であるから,
  これらの共役類は一致しない.
  (注意: $(2\ 1\ 3\ 4\ 5)$についても全く同じ議論によって$A_5$における共役類のサイズがたかだか$12$であるといえるので, 結局どちらの共役類のサイズも$12$に等しい.)

  以上により, $\mathfrak{S}_5$における$(1\ 2\ 3\ 4\ 5)$の共役類は,
  $A_5$において$(1\ 2\ 3\ 4\ 5)$を代表元とする共役類と$(2\ 1\ 3\ 4\ 5)$を代表元とする共役類とに細分される.

  \underline{$(1\ 2\ 3)$について.}
  $\sigma = \tau(1\ 2\ 3)\tau^{-1} \Leftrightarrow \sigma = \tau(4\ 5)(1\ 2\ 3)(4\ 5)\tau^{-1}$より,
  $(1\ 2\ 3)$の$\mathfrak{S}_5$における共役類と$A_5$における共役類は一致する.

  \underline{$(1\ 2)(3\ 4)$について.}
  $\sigma = \tau(1\ 2)(3\ 4)\tau^{-1} \Leftrightarrow \sigma = \tau(1\ 2)(1\ 2)(3\ 4)(1\ 2)\tau^{-1}$より,
  $(1\ 2)(3\ 4)$の$\mathfrak{S}_5$における共役類と$A_5$における共役類は一致する.

  \underline{$1$について.} 共役類が$1$元集合なので, そのまま$A_5$の共役類の代表元となっている.\\
\end{ans}

\begin{ans}
  (1) $\begin{pmatrix*}
    1 & 2 & 3 & 4 & 5 & 6 \\
    4 & 1 & 3 & 2 & 6 & 5
  \end{pmatrix*}$\\
  (2) $\{1, 2, 3\}$を$\{4, 1, 3\}$に移すような$\nu$については,
  $\nu(1)$と$\nu(4)$の値を決めれば$1$つに定まる. したがって$\nu$は$3 \cdot 3 = 9$通りある.
  $\{1, 2, 3\}$を$\{2, 6, 5\}$に移すような$\nu$についても同様. したがって全部で$18$通りである.
\end{ans}

\begin{ans}
  (1) 共役による作用での$(1\ 2)$の軌道 (共役類) は$(2, 1, 1)$型の置換からなり, これは全部で${}_4\mathrm{C}_2 = 6$個あるから,
  $Z_G(\sigma)$の位数, すなわち共役による作用での$(1\ 2)$の安定化群の位数は$\abs{G}/6 = 4$である.
  一方, 明らかに$(1\ 2), (3\ 4) \in Z_G(\sigma)$であるから,
  $\langle (1\ 2), (3\ 4) \rangle \subset Z_G(\sigma)$である.
  $(1\ 2)$と$(3\ 4)$は可換でいずれも位数$2$であって, $\langle (1\ 2), (3\ 4) \rangle$の任意の元は$(1\ 2)^i(3\ 4)^j (i, j = 0, 1)$と書けるので,
  $\abs{\langle (1\ 2), (3\ 4) \rangle} = 4$である.
  したがって$Z_G(\sigma) = \langle (1\ 2), (3\ 4) \rangle$.\\
  (2) 共役による作用での$(1\ 2)(3\ 4)$の軌道は$(2, 2)$型の置換からなり, これは${}_4\mathrm{C}_2/2 = 3$個ある.
  したがって$\abs{Z_G(\sigma)} = \abs{G}/3 = 8$.
  まず, 共役による作用で$(1\ 2)$をそれ自身に移すものとして$1, (1\ 2), (3\ 4), (1\ 2)(3\ 4)$があり,
  $(1\ 2)$を$(3\ 4)$に移すものとして$(1\ 3)(2\ 4), (1\ 4)(2\ 3), (1\ 3\ 2\ 4), (1\ 4\ 2\ 3)$がある.
  以上を合わせて$8$個なので, これらが$Z_G(\sigma)$をなす.
  まとめると$Z_G(\sigma) = \langle (1\ 2), (3\ 4), (1\ 3)(2\ 4) \rangle$.\\
  (3) 共役による作用での$(1\ 2\ 3)$の軌道は$(3, 1)$型の置換からなり, これは${}_4\mathrm{C}_3 \cdot 2 = 8$個ある.
  したがって$\abs{Z_G(\sigma)} = \abs{G}/8 = 3$.
  一方, 明らかに$\langle (1\ 2\ 3) \rangle \subset Z_G(\sigma)$であるが,
  $\abs{\langle (1\ 2\ 3) \rangle} = 3$なので$Z_G(\sigma) = \langle (1\ 2\ 3) \rangle$.\\
  (4) 共役による作用での$(1\ 2\ 3)$の軌道は$(3, 1, 1)$型の置換からなり, これは${}_5\mathrm{C}_3 \cdot 2 = 20$個ある.
  したがって$\abs{Z_G(\sigma)} = \abs{G}/20 = 6$.
  一方, 明らかに$\langle (1\ 2\ 3), (4\ 5) \rangle \subset Z_G(\sigma)$であるが,
  $\abs{\langle (1\ 2\ 3), (4\ 5) \rangle} = 6$なので,
  $Z_G(\sigma) = \langle (1\ 2\ 3), (4\ 5) \rangle$.\\
  (5) 共役による作用での$(1\ 2\ 3)(4\ 5\ 6)$の軌道は$(3, 3)$型の置換からなり, これは${}_6\mathrm{C}_3 \cdot 2 \cdot 2 / 2 = 40$個ある.
  したがって$\abs{Z_G(\sigma)} = \abs{G}/40 = 18$.
  まず, 明らかに$\langle (1\ 2\ 3), (4\ 5\ 6), (1\ 4)(2\ 5)(3\ 6) \rangle \subset Z_G(\sigma)$であり,
  $(1\ 2\ 3)$, $(4\ 5\ 6)$, $(1\ 4)(2\ 5)(3\ 6)$の位数はそれぞれ$3$, $3$, $2$である.
  そこで, $(1\ 2\ 3)^i(4\ 5\ 6)^j((1\ 4)(2\ 5)(3\ 6))^k$という形の元を考えることにする.
  $i, j, i^\prime, j^\prime = 0, 1, 2$, また$k, k^\prime = 0, 1$として,
  \begin{align*}
    (1\ 2\ 3)^i(4\ 5\ 6)^j((1\ 4)(2\ 5)(3\ 6))^k = (1\ 2\ 3)^{i^\prime}(4\ 5\ 6)^{j^\prime}((1\ 4)(2\ 5)(3\ 6))^{k^\prime}
  \end{align*}
  ならば,
  \begin{align*}
    (1\ 2\ 3)^{i-i^\prime}(4\ 5\ 6)^{j-j^\prime} = ((1\ 4)(2\ 5)(3\ 6))^{k^\prime - k}
  \end{align*}
  となるが, 左辺の位数は$3$の約数, 右辺の位数は$2$の約数なので, 両辺は$1$である.
  したがって$(i, j, k) = (i^\prime, j^\prime, k^\prime)$なので,
  $(1\ 2\ 3)^i(4\ 5\ 6)^j((1\ 4)(2\ 5)(3\ 6))^k$の形の元は互いに異なり, $3 \cdot 3 \cdot 2 = 18$個ある.
  よって$Z_G(\sigma) = \langle (1\ 2\ 3), (4\ 5\ 6), (1\ 4)(2\ 5)(3\ 6) \rangle$.\\
  (6) 共役による作用での$(1\ 2)(3\ 4)(5\ 6)$の軌道は$(2, 2, 2)$型の置換からなり, これは$5 \cdot 3 = 15$個ある ($1$の行き先と, 残りの$4$個から$1$つ選んだものの行き先とを決めると考えればよい).
  したがって$\abs{Z_G(\sigma)} = \abs{G}/15 = 48$.
  まず, 明らかに$\langle (1\ 2), (3\ 4), (5\ 6), (1\ 3)(2\ 4), (1\ 3\ 5)(2\ 4\ 6) \rangle \subset Z_G(\sigma)$であり,
  生成元の位数はそれぞれ$2$, $2$, $2$, $2$, $3$である.
  そこで, $(1\ 2)^i(3\ 4)^j(5\ 6)^k((1\ 3)(2\ 4))^l((1\ 3\ 5)(2\ 4\ 6))^m$という形の元を考えることにする.
  $i, j, k, l, i^\prime, j^\prime, k^\prime, l^\prime = 0, 1$, また$m, m^\prime = 0, 1, 2$として,
  \begin{align*}
    (1\ 2)^i&(3\ 4)^j(5\ 6)^k((1\ 3)(2\ 4))^l((1\ 3\ 5)(2\ 4\ 6))^m\\
    = &(1\ 2)^{i^\prime}(3\ 4)^{j^\prime}(5\ 6)^{k^\prime}((1\ 3)(2\ 4))^{l^\prime}((1\ 3\ 5)(2\ 4\ 6))^{m^\prime}
  \end{align*}
  ならば,
  \begin{align*}
    (1\ 2)^{i - i^\prime}(3\ 4)^{j - j^\prime}(5\ 6)^{k - k^\prime}
     = ((1\ 3)(2\ 4))^{l^\prime}((1\ 3\ 5)(2\ 4\ 6))^{m^\prime - m}((1\ 3)(2\ 4))^l
  \end{align*}
  となるが, 両辺の$1$, $3$, $5$の行き先の候補を比較すると, 共通するものはそれぞれ$1$, $3$, $5$しかないので,
  $(i, j, k) = (i^\prime, j^\prime, k^\prime)$である. したがって
  \begin{align*}
    ((1\ 3)(2\ 4))^{l + l^\prime} = ((1\ 3\ 5)(2\ 4\ 6))^{m^\prime - m}
  \end{align*}
  となるが, 左辺の位数は$2$の約数, 右辺の位数は$3$の約数なので, 両辺は$1$である.
  したがって$(l, m) = (l^\prime, m^\prime)$なので, 結局$(i, j, k, l, m) = (i^\prime, j^\prime, k^\prime, l^\prime, m^\prime)$.
  よって$(1\ 2)^i(3\ 4)^j(5\ 6)^k((1\ 3)(2\ 4))^l((1\ 3\ 5)(2\ 4\ 6))^m$という形の元は互いに異なり,
  $2 \cdot 2 \cdot 2 \cdot 2 \cdot 3 = 48$個ある.
  $\abs{Z_G(\sigma)} = 48$かつ
  $\langle (1\ 2), (3\ 4), (5\ 6), (1\ 3)(2\ 4), (1\ 3\ 5)(2\ 4\ 6) \rangle \subset Z_G(\sigma)$であったから,
  $Z_G(\sigma) = \langle (1\ 2), (3\ 4), (5\ 6), (1\ 3)(2\ 4), (1\ 3\ 5)(2\ 4\ 6) \rangle$.
\end{ans}

\begin{ans}
  (1) $Z_G(\sigma) \supset \langle \sigma \rangle$であることは明らか.
  一方, 共役による作用での$\sigma$の軌道は長さ$n$の巡回置換全体からなるので, 軌道の濃度は$(n - 1)!$である.
  したがって$\abs{Z_G(\sigma)} = \abs{\mathfrak{S}_n}/(n - 1)! = n$であるから, $Z_G(\sigma) = \langle \sigma \rangle$.\\
  (2) 明らかに任意の$\tau \in \langle \sigma \rangle \setminus \{1\}$に対して
  $(1\ 2)\tau(1\ 2) \neq \tau$であるから, $Z_G(\sigma) \cap Z_G((1\ 2)) = \{1\}$.
  よって$Z_G(\mathfrak{S}_n) = \{1\}$.
\end{ans}

\begin{ans}
  (フォーマルな証明を書くのが難しいので, ひとまずスケッチにとどめておく.)
  % TODO: ちゃんとした証明を書く
  $\tau \in Z_G(\sigma)$であるとき, $\sigma$を巡回置換の積で表したものについて,
  その積を構成する巡回置換たちのうち, 同じ長さのものの間に置換が引き起こされる.
  したがって, $\tau$に対して, $i = 1, 2,... t$ごとに$a_i$次の置換が定まることになる.
  この対応は$\mathfrak{S}_{a_1} \times \cdots \mathfrak{S}_{a_t}$への全射準同型を定める.
  その核は, $\tau$の作用を巡回置換を表す文字の配列に対する形式的なものとみなせば, ヤング図形の"行"ごとにサイクリックに要素をシフトするものであり,
  $(\mathbb{Z}/j_1\mathbb{Z})^{a_1} \times \cdots \times (\mathbb{Z}/j_t\mathbb{Z})^{a_t}$に同型である.
  したがって準同型定理により, 求めるべき同型が得られる.
\end{ans}

\begin{ans}
  (1) $i_1,..., i_l$がすべて異なる数ならば, $\tau \in Z_{\mathfrak{S}_n}(\sigma)$による共役$\tau\sigma\tau^{-1}$は
  $\sigma$を構成する巡回置換それぞれについて, 数字の列をシフトするものである.
  このシフトは列と同じ長さの巡回置換の累乗であり, 長さが奇数の巡回置換は偶置換であることから, $\tau$は偶置換である.
  したがって$Z_{\mathfrak{S}_n}(\sigma) \subset Z_{A_n}(\sigma)$. 逆の包含関係は自明である.
  またこのとき, $\sigma$の奇置換による共役は, $(1\ 2)\sigma(1\ 2)$の偶置換による共役に他ならない.
  したがって, $\sigma$の$\mathfrak{S}_n$における共役類は, $A_n$における2つの共役類$C(\sigma)$と$C((1\ 2)\sigma(1\ 2))$の和である.
  この2つの共役類が相異なるものでありしかも元の個数が等しいことは,
  $\abs{\mathfrak{S}_n}/\abs{Z_{\mathfrak{S}_n}(\sigma)} = 2\abs{A_n}/\abs{Z_{A_n}(\sigma)}$
  と命題4.1.24から分かる.\\
  (2) このとき$i_1,\cdots, i_l$について「偶数が1つある」か「同じ奇数がある」のいずれかが成り立つが,
  いずれの場合も, 奇置換$\tau_0$を$\tau_0\sigma\tau_0^{-1} = \sigma$が成り立つようにとることができる.
  (前者の場合は偶数長の巡回のシフト, 後者の場合は同じ奇数長の巡回を交換するものを考えればよい.)
  このとき, $\tau\sigma\tau^{-1} = \sigma \Longleftrightarrow \tau\tau_0\sigma\tau_0^{-1}\tau^{-1} = \sigma$
  より, $\tau \in Z_{A_n}(\sigma) \Longleftrightarrow \tau\tau_0 \in Z_{\mathfrak{S}_n}(\sigma) \setminus Z_{A_n}(\sigma)$.
  したがって, $Z_{\mathfrak{S}_n}(\sigma) = Z_{A_n}(\sigma) \cup Z_{A_n}(\sigma)\tau_0$.
  共役類が等しいことも, $\tau_0$の存在から分かる. \\
  (1) の別解: $\sigma$を構成する巡回置換ごとに安定化群を考えることで,
  演習問題4.2.2 (2) の解答と同様の議論により, $\sigma$の$\mathfrak{S}_n$における共役類が$A_n$における元の個数の等しい$2$つの共役類の和であることが先に分かる.
  このことから$Z_{\mathfrak{S}_n}(\sigma) = Z_{A_n}(\sigma)$も示される.
\end{ans}

\begin{ans}
  $\rho((1\ 2)) = (2\ 3)$, $\rho((2\ 3)) = (1\ 2)$であることがただちに見て取れる.
  これらの像は$\mathfrak{S}_3$を生成するので, $\rho$は全射. よって同型.
\end{ans}

\begin{ans}
  (1) 計算するだけなので略.\\
  (2) (1) の結果と$\rho(e) = e$を合わせると, $\rho$の像に含まれる$4$つの異なる元が得られる.
  一方, $\rho$の像の位数は$\abs{\mathfrak{S}_3} = 6$の約数であるから, $\rho$は全射である.\\
  (3) $\sigma \in \Ker{\rho}$について, $\sigma(1) = 1$ならば,
  $\rho(\sigma)(1) = 1$より$\sigma(2) = 2$,
  $\rho(\sigma)(2) = 2$より$\sigma(3) = 3$,
  $\rho(\sigma)(3) = 3$より$\sigma(4) = 4$,
  であることが必要である.
  よって$\sigma = e$であるか, または$\sigma(1) = 1$をみたす$\sigma \in \Ker{\rho}$は存在しない. (ただしこの場合は明らかに$\sigma = e$は条件をみたす.)
  同様に$\sigma(1)$の値で場合分けすることで, $\Ker{\rho}$の要素の候補として$\{e, x_1, x_2, x_3\}$が得られる.
  準同型定理より$\abs{\Ker{\rho}} = 4$なので, $\Ker{\rho} = \{e, x_1, x_2, x_3\}$である.
\end{ans}

\begin{ans}
  部分群の共役類は4.5節で定義している.
  共役類の代表元である部分群の生成元を巡回置換の積の形で示せば, 同じ共役類に属する他の部分群の生成元も, その巡回置換の要素の入れ替えによって容易に得られることに注意する.
  そこで, 各共役類の代表元の生成元を巡回置換の積の形で示すことで解答とする.\\
  共役な部分群の位数は等しく, $\mathfrak{S}_4$の部分群の位数は1, 2, 3, 4, 6, 8, 12, 24のいずれかなので, それぞれについて考察する.\\
  \underline{位数$1$:}
  $\{e\}$のみ. \\
  \underline{位数$2$:}
  部分群は位数$2$の元$g$で生成される.
  $\mathfrak{S}_4$の元で位数が$2$であるのは, 型としては$(2, 1, 1)$と$(2, 2)$のみである.
  よって, 部分群の共役類は$2$つあり, それぞれの代表元として$\gen{(1\ 2)}$と$\gen{(1\ 2)(3\ 4)}$を挙げることができる.\\
  \underline{位数$3$:} 部分群は位数$3$の元$g$で生成される.
  $\mathfrak{S}_4$の元で位数が$3$であるのは, 型としては$(3, 1)$のみである.
  よって, 部分群の共役類はただ$1$つで, 代表元として$\gen{(1\ 2\ 3)}$を挙げることができる.\\
  \underline{位数$4$:}
  (i) 部分群が位数$4$の元を含むならば, その部分群は位数$4$の元$g$で生成される.
  $\mathfrak{S}_4$の元で位数が$4$であるのは, 型としては$(4)$のみであり,
  この場合の代表元として$\gen{(1\ 2\ 3\ 4)}$を挙げることができる.
  (ii) 部分群が位数$4$の元を含まないとき, 単位元以外の$3$つの元は位数$2$の元である.
  (ii-a) 位数$2$の元で型$(2, 1, 1)$のものが$2$つ以上あるとし, そのうちの$1$つを$(1\ 2)$とする.
  もう$1$つの型$(2, 1, 1)$の元としては, 代表元を考えているので, $(1\ 3)$, $(3\ 4)$のみをチェックすればよいが,
  これらのうち$(1\ 2)$と組み合わせて位数$4$の部分群を生成するのは$(3\ 4)$のみである.
  したがって, この場合の代表元としては$\gen{(1\ 2), (3\ 4)}$を挙げることができる.
  (ii-b) 位数$2$の元で型$(2, 2)$のものが$2$つ以上あるとし, そのうちの$1$つを$(1\ 2)(3\ 4)$とする.
  もう$1$つの型$(2, 2)$の元としては, 代表元を考えているので, $(1\ 3)(2\ 4)$のみチェックすればよい.
  この$2$元は部分群$\{e, (1\ 2)(3\ 4), (1\ 3)(2\ 4), (1\ 4)(2\ 3)\}$を生成することが確かめられるので,
  この部分群$\gen{(1\ 2)(3\ 4), (1\ 3)(2\ 4)}$を代表元として挙げることができる.
  (この場合は部分群の共役類がこの$1$元のみからなる.)\\
  \underline{位数$6$:}
  演習問題2.10.9 (1) で示したように, この部分群には位数$3$の元が存在する.
  $\mathfrak{S}_4$の元で位数が$3$であるのは, 型としては$(3, 1)$のみであるから,
  この部分群に$(1\ 2\ 3)$が含まれるとしてよい. 再び演習問題2.10.9より, この部分群には位数$2$の元で同じ型のものが$3$つ存在する.
  これらの型$(2, 2)$だとすると上でみたことから位数$4$の部分群があることになり矛盾するので, これらの型は$(2, 1, 1)$である.
  そこでその元の$1$つとしては, (代表元を考えているので) $(1\ 2)$と$(1\ 4)$のみをチェックすればよい.
  これらのうち$(1\ 2\ 3)$と組み合わせて位数$6$の部分群を生成するのは, $(1\ 2)$のみである.
  ((1\ 4)(1 2 3) = (1\ 2\ 3\ 4)で位数$4$の元ができるので, $(1\ 4)$は除外される.)
  よって代表元としては$\gen{(1\ 2\ 3), (1\ 2)} (\simeq \mathfrak{S}_3)$を挙げることができる.\\
  \underline{位数$8$:}
  この部分群がもし型$(2, 1, 1)$の元を3つ以上含むとすると, その部分群として$\mathfrak{S}_3$と同型な群があることになり矛盾.
  したがって型$(2, 1, 1)$の元は$2$つ以下である.また型$(2, 2)$の元は$\mathfrak{S}_4$には3つしかない.
  したがって, 位数$8$の部分群は, 型$(4)$の元を$2$つ以上含むことになる.
  代表元を考えているので, $(1\ 2\ 3\ 4)$は必ず含まれるとし, これと別の$(1\ i\ j\ k)$の形の元を含むと考えてよい.
  この組み合わせは$5$通りあるが,
  \begin{align*}
    (1\ 2\ 3\ 4)(1\ 2\ 4\ 3) &= (1\ 3\ 2)\\
    (1\ 2\ 3\ 4)(1\ 3\ 2\ 4) &= (1\ 4\ 2)\\
    (1\ 2\ 3\ 4)(1\ 3\ 4\ 2) &= (1\ 4\ 3)\\
    (1\ 2\ 3\ 4)(1\ 4\ 2\ 3) &= (2\ 4\ 3)\\
    (1\ 2\ 3\ 4)(1\ 4\ 3\ 2) &= e\\
  \end{align*}
  により, 位数$3$の元を生成しないのは$(1\ 2\ 3\ 4)$と$(1\ 4\ 3\ 2) = (1\ 2\ 3\ 4)^{-1}$の組み合わせのみである.
  したがって型$(4)$の元はこの$2$つのみと考えてよい.
  型$(2, 1, 1)$の元について再び考えると, その共役の作用によって$(1\ 2\ 3\ 4)$が上で除外された型$(4)$の元に移ってはいけないことから,
  $(1\ 2), (1\ 4), (2\ 3), (3\ 4)$は除外される. したがって残りは$(1\ 3)$と$(2\ 4)$であるが,
  ここまでで残ったものをすべて書き出すと,
  $\{e, (1\ 2\ 3\ 4), (1\ 4\ 3\ 2), (1\ 3), (2\ 4), (1\ 2)(3\ 4), (1\ 3)(2\ 4), (1\ 4)(2\ 3) \}$
  は$8$つの元からなる. $\mathfrak{S_4}$の位数$8$の部分群 (と同型な群) として少なくとも$D_4$があるので,
  この$8$つの元からなる集合が$D_4$と同型な部分群ということになる.
  生成元を簡潔に書くと, $\gen{(1\ 2\ 3\ 4), (1\ 3)}$.\\
  \underline{位数$12$:}
  位数$12$の部分群は, 演習問題2.8.2より正規部分群である.
  よってこの部分群は共役類の和であり,
  そこに含まれる元と同じ型の元はすべて含まれていなければならない.
  型ごとに元の数を確認すると,
  \begin{itemize}
    \item 型$(1, 1, 1, 1)$: $1$つ
    \item 型$(2, 1, 1)$: $6$つ
    \item 型$(3, 1)$: $8$つ
    \item 型$(4)$: $6$つ
    \item 型$(2, 2)$: $3$つ
  \end{itemize}
  であり, 部分群に$e$が必ず含まれることから始めて考察すると,
  型$(1, 1, 1, 1)$, 型$(2, 2)$, 型$(3, 1)$の元によってこの部分群は構成されることになり, 群構造は$1$つに定まる.
  これはあきらかに$A_4$である.
  あまり意味はないがここまでと同様に生成元で簡潔に書くと, $\gen{(1\ 2\ 3), (1\ 2)(3\ 4)}$となる.\\
  \underline{位数$24$:} これは$\mathfrak{S}_4$自身.\\
  以上をまとめると, 共役類の代表元は,
  $\{e\}$, $\gen{(1\ 2)}$, $\gen{(1\ 2)(3\ 4)}$, $\gen{(1\ 2\ 3)}$,
  $\gen{(1\ 2\ 3\ 4)}$, $\gen{(1\ 2), (3\ 4)}$, $\gen{(1\ 2)(3\ 4), (1\ 3)(2\ 4)}$,
  $\gen{(1\ 2\ 3), (1\ 2)} (\simeq \mathfrak{S}_3)$,
  $\gen{(1\ 2\ 3\ 4), (1\ 3)} (\simeq D_4)$,
  $\gen{(1\ 2\ 3), (1\ 2)(3\ 4)} (= A_4)$,
  $\mathfrak{S}_4$.
  正規部分群となるのは, 位数$12$の場合に言及したとおり, 部分群が共役類の和となる場合である.
  これをみたすのは, $\{e\}$, $\gen{(1\ 2)(3\ 4), (1\ 3)(2\ 4)}$,
  $\gen{(1\ 2\ 3), (1\ 2)(3\ 4)} (= A_4)$, $\mathfrak{S}_4$.
\end{ans}

\fakesubsection{4.3 交換子群と可解群}

\begin{ans}
  計算するだけなので略.
\end{ans}

\begin{ans}
  (1) (積で閉じていること) $A, B \in N_i$を$A = (a_{jk})$, $B = (b_{jk})$とすると,
  $(AB)_{jk} = \sum_l a_{jl}b_{lk}0$.
  (a) $j > k$ならば, $j > l$または$l > k$がつねに成り立つので, $(AB)_{jk} = 0$.
  (b) $j = k$ならば, $j = k = l$のときのみ$a_{jl}b_{lk} = 1$でそれ以外は$j$$0$なので, $(AB)_{kk} = 1$.
  (c) $j + i > k > j$とする. このとき$a_{jl} = 0$となるのは$l \in [1, j) \cup (j, j + i) (\supset [1, j) \cup (j, k])$,
  $b_{lk} = 0$となるのは$l \in (k - i, k) \cup (k, n] (\supset [j, k) \cup (k, n])$のときである.
  両者の区間和は$[1, n]$を覆うので, つねに$a_{jl}b_{lk} = 0$, よって$(AB)_{jk} = 0$.
  以上により$AB \in N_i$.\\
  (逆元で閉じていること) $A \in N_i$とする. $B = I - A$とおくと,
  $B^{n} = 0$であることが計算で確かめられる.
  よって$A = I - B$の逆元は$I + B + B^2 + \cdots + B^{n-1}$.
  これは$N_i$の元である.\\
  (2) $A \in N_1$, $B \in N_i$とする.
  $C = I - A$, $D = I - B$とおくと,
  \[
    ABA^{-1}B^{-1} = (I - C)(I - D)(I + C + C^2 \cdots + C^{n-1})(I + D + D^2 + \cdots + D^{n-1})
  \]
  これは$I + CDX + DCY + ZD^2\ (X, Y, Z \in N_1)$と整理できるので, $N_{i+1}$の元である.\\
  (3) 一般に, $G$の部分群の列$G = G_0 \supset G_1 \supset \cdots \supset G_n = \{1\}$が
  $i = 0,..., n - 1$に対して$[G, G_i] \subset G_{i+1}$をみたすならば, $G$はべき零群であることを示す.
  この条件をみたすときとくに$[G, G_{i+1}] \subset G_{i+1}$が成り立つから,
  任意の$a \in G$, $b \in G_{i+1}$に対して$aba^{-1} \in G_{i+1}b = G_{i+1}$.
  よって$G_{i+1} \triangleleft G$.
  また$aG_{i+1} \in G/G_{i+1}$, $bG_{i+1} \in G_i/G_{i+1}$を任意にとると,
  $aG_{i+1} \cdot bG_{i+1} \cdot (aG_{i+1})^{-1} \cdot (bG_{i+1})^{-1} = (aba^{-1}b^{-1})G_{i+1} = G_{i+1}$.
  よって$aG_{i+1}$と$bG_{i+1}$は可換であるから, $G_i/G_{i+1}$は$G/G_{i+1}$の中心に含まれる.
  よって$G$はべき零群である. 以上の事実を$N_1$の部分群の列$N_1 \supset N_2 \supset \cdots N_{n} = \{1\}$に適用すればよい.
\end{ans}

\fakesubsection{4.4 p群}

\begin{ans}
  (アイディア: 演習問題4.3.2 (3) の解答で示したように, $p$群$G$に対して$[G, G_i] \subset G_{i+1}$をみたす減少列の存在を示せばよい.
  減少列の最後の$2$つの要素の候補として$Z(G) \supsetneq \{1\}$が簡単に思いつく.
  そこで, 減少列を逆向きに構成することを考える. 添字も逆向きにつけることにして, $G_0 = \{1\}$, $G_1 = Z(G)$からはじめて
  $[G, G_{i+1}] \subset G_{i}$をみたし、かつどこかで$G_n = G$となる増大列を構成すればよい.
  $G_{i}$に対して$G_{i+1}$をなるべく大きく取るようにすれば, 列$\{G_i\}$が狭義増大列となってやがて$G$に到達する可能性がありそうである.)\\

  \underline{補題}: $H \triangleleft G$とする. このとき,
  \[
    K = \{x \mid x \in G \land [G, x] \subset H \}
  \]
  と定めると, $K \supset H$かつ$K \triangleleft G$.\\

  \underline{証明}: $x, y \in K$とする. 任意の$g \in G$に対して
  \[
    gx^{-1}g^{-1}x = (gx^{-1}g^{-1})x(gx^{-1}g^{-1})^{-1}x^{-1} \cdot (gxg^{-1}x^{-1})^{-1} \in H
  \]
  より$x^{-1} \in K$. また,
  \[
    gxyg^{-1}y^{-1}x^{-1} = (gx)y(gx)^{-1}y^{-1} \cdot (yg)x(yg)^{-1}x^{-1} \in H
  \]
  より$xy \in K$. さらに$g_0 \in G$を任意に取ると,
  \[
    g(g_0xg_0^{-1})g^{-1}(g_0xg_0^{-1})^{-1} = (gg_0)x(gg_0)^{-1}x^{-1} \cdot (g_0xg_0^{-1}x^{-1})^{-1} \in H
  \]
  より$g_0xg_0^{-1} \in K$なので$K$は正規部分群である.
  $K$が$H$を含むことは, $H$が正規部分群であることから従う.\\

  \underline{例}: $H = \{1\}$のとき, $K = Z(G)$.\\

  補題を$G_0 = \{1\}$から始めて$G_i$に適用して$G_{i+1}$を得ることを繰り返して,
  増大列
  \[
    G_0 = \{1\} \subset G_1 \subset G_2 \subset \cdots \subset G_i \subset G_{i+1} \subset \cdots
  \]
  で$G_i \triangleleft G$かつ$[G, G_{i+1}] \subset G_{i}$をみたすものが得られる.
  もし$G_i \neq G$ならば, $G/G_i$は$p$群なので命題4.4.3より$Z(G/G_i) \neq \{1\}$.
  したがって$a \notin G_i$で$aG_i \in Z(G/G_i)$なるものがある. この$a$と任意の$g \in G$について,
  \[
    (gG_i)(aG_i)(g^{-1}G_i)(a^{-1}G_i) = (gG_i)(g^{-1}G_i)(aG_i)(a^{-1}G_i) = G_i
  \]
  であることから$gag^{-1}a^{-1} \in G_i$, よって$a \in G_{i+1}$.
  すなわち$G_i \subsetneq G_{i+1}$である.
  したがって上の増大列はある$n$について$G_n = G$となるまで真に増大し続けるので, $G$はべき零.
\end{ans}

\begin{ans}
  $H$を左剰余類$G/H$に左から作用させることを考える.
  この作用により$H \in G/H$は不動なので, $\order{G/H} = p$と命題4.1.24より$G/H$のすべての元が不動である.
  このことは, $\forall h \in H$, $\forall g \in G$にたいして$hgH = gH$すなわち$g^{-1}hg \in H$であることを意味する.
  したがって$H$は$G$の正規部分群である.
\end{ans}

\fakesubsection{4.5 シローの定理}

\begin{ans}
  (1) $\sigma(1) = (1\ 3\ 2)$, $\sigma((1\ 2)(3\ 4)) = (2\ 3\ 4)$より,
  $\sigma(y) = \{(1\ 3\ 2), (2\ 3\ 4)\}$.
  (2) $\tau(1) = 1$の場合と$\tau(1) = (1\ 2)(3\ 4)$の場合が考えられる.
  前者の場合は$\tau = 1$, 後者の場合は$\tau = (1\ 2)(3\ 4)$で,
  それぞれ$\tau((1\ 2)(3\ 4)) = (1\ 2)(3\ 4)$, $\tau((1\ 2)(3\ 4)) = 1$
  が成り立つので, $\{1, (1\ 2)(3\ 4)\}$が$y$の安定化群.
\end{ans}

\begin{ans}
  (1) 命題4.1.10でみたように, $D_4$の元は$\tau^i\sigma^j\ (i = 0, 1,\ j = 0, 1, 2, 3)$
  の形で書ける. これらのうち, 位数$2$の元が生成するものを考えればよい.
  具体的には, $\gen{\sigma^2}$, $\gen{\tau}$, $\gen{\tau\sigma}$, $\gen{\tau\sigma^2}$, $\gen{\tau\sigma^3}$.
  (2) 明らかに, 型の異なるものは同じ軌道に含まれない.
  したがって, $\{\gen{\tau}, \gen{\tau\sigma^2}\}$と
  $\{\gen{\sigma^2}, \gen{\tau\sigma}, \gen{\tau\sigma^3}\}$とは軌道が異なる.
  また, 共役による作用によって$\tau$の冪は不変なので,
  $\{\gen{\sigma^2}, \gen{\tau\sigma}, \gen{\tau\sigma^3}\}$はさらに
  $\{\gen{\sigma^2}\}$と$\{\gen{\tau\sigma}, \gen{\tau\sigma^3}\}$とに分かれる.
  $\tau = \sigma(\tau\sigma^2)\sigma{-1}$, $\tau\sigma = \sigma(\tau\sigma^3)\sigma^{-1}$なので,
  軌道はこれ以上は細分されない.
  (3) $\gen{\tau}$について.
  共役による作用を考えているので, 安定化群の元$\rho$について, $\rho(\tau) = \tau$が成り立つ.
  $\tau\tau\tau^{-1} = \tau$, $\sigma^2\tau\sigma^{-2} = \tau$なので,
  $\tau, \sigma^2 \in G_{\gen{\tau}}$. よって$\gen{\tau, \sigma^{2}} \subset G_{\gen{\tau}}$.
  また$\sigma\tau\sigma^{-1} = \tau\sigma^2$により$\sigma \notin G_{\gen{\tau}}$なので,
  $\abs{G_{\gen{\tau}}} \le 4$. よって$G_{\gen{\tau}} = \gen{\tau, \sigma^2}$.
  $\gen{\sigma^2}$について, 安定化群は明らかに$G$.
  $\gen{\tau\sigma}$の安定化群は, $\gen{\tau}$の安定化群と同様にして$\gen{\tau\sigma, \sigma^2}$であることがわかる.
  (注意: 演習問題4.1.12の解答に記した考察より, 安定化群は軌道の代表元の選び方によらない.)
\end{ans}

\begin{ans}
  (1) 途中までは例題4.5.8と同様.
  シロー$p$部分群の数を$s$, シロー$q$部分群の数を$t$とする.
  $s$, $t$の可能性はそれぞれ$1, q$および$1, p$である.
  もし$s = q$ならば$q \equiv 1 \mod p$なので, とくに$q > p$である.
  このとき$p \not\equiv 1 \mod q$より$t = 1$.
  よって$s, t$のいずれかは$1$なので, シロー$p$部分群とシロー$q$部分群の少なくとも一方が正規部分群である.
  したがって$G$は単純群でない.\\
  (2) このとき明らかに$q \not\equiv 1 \mod p$も成り立つので,
  例題4.5.8とまったく同様の議論により$G$は巡回群である.\\
  (3) $30$以下の素数$2, 3, 5, 7, 11, 13, 17, 19, 23, 29$のうちの異なる$2$つの積から$60$未満のものを挙げるとよい.
  (2)をみたす$q$の候補はただちに$3, 5, 7$まで絞り込めるので, それほど大変でもない.
\end{ans}

\begin{ans}
  (1)例題4.5.8と同様にして,
  $G$のシロー$2$部分群の数を$s$, シロー$5$部分群の数を$t$とする.
  $s$, $t$の可能性はそれぞれ$1, 5$および$1, 2, 4, 8$である.
  $2, 4, 8$はいずれも$\mod 5$で$1$と合同ではないので, $t = 1$.
  したがって, シロー$5$部分群が正規部分群であるので, $G$は単純群でない.\\
  (2) 同様にシロー$7$部分群が正規部分群であることが示される.
  (3) シロー$3$部分群が正規部分群であることが示される. (演習問題2.8.2を用いてもよい.)
\end{ans}

\begin{ans}
  シロー$7$部分群の個数は$1, 8$のいずれかである. $1$の場合はこれが正規部分群である.
  そこでシロー$7$部分群の個数が$8$の場合を考える.
  シロー$7$部分群が巡回群であることから, これらは互いに生成元を共通元として持たない.
  よって, $G$には位数$7$の元が$6 \cdot 8 = 48$個ある.
  したがってシロー$2$部分群に属しうる元の個数は$56 - 48 = 8$個のみなので, シロー$2$部分群の個数は$1$であり,
  この場合はシロー$2$部分群が正規部分群である.
\end{ans}

\begin{ans}
  (1) シロー$3$部分群の個数は$1, 10$, シロー$5$部分群の個数は$1, 6$のいずれかである.
  もしシロー$3$部分群の個数が$10$かつシロー$5$部分群の個数が$6$であれば,
  位数$3$の元が$20$個かつ位数$5$の元が$24$個存在することになるので矛盾.
  したがって, シロー$3$部分群とシロー$5$部分群のいずれかが正規部分群である.
  よって定理2.10.3より$HK$は部分群である.
  さらに$H \cap K = \{1_G\}$より$\order{HK} = 15$であることと例題4.5.8より
  $HK \simeq \mathbb{Z}/15\mathbb{Z}$.\\
  (2) シロー$2$部分群が存在することから, 位数$2$の元$g_2$が存在することがわかる.
  $g_2$による$G$への共役による作用を考える.
  $HK$は$G$の正規部分群なので, この作用は$HK$への作用 (自己同型$\phi$) を引き起こす.
  $HK$の位数$3$の元$g_3$, 位数$5$の元$g_5$をとって固定すると,
  $\phi(g_3)$は$g_3$か$g_3^{-1}$のいずれか,
  $\phi(g_5)$は$g_5$か$g_5^{-1}$のいずれかであることがわかる. ($\phi^2 = \mathrm{id}_{HK}$であることによる.)
  以下, 場合分けして考える.\\
  (2-i) $\phi(g_3) = g_3$かつ$\phi(g_5) = g_5$の場合.
  このとき, $g_2$は$HK$および$g_2 HK$の元と可換であることから, $\gen{g_2} \simeq \mathbb{Z}/2\mathbb{Z}$は$G$の正規部分群である.
  したがって, 例題4.5.8の後半と同様の議論により, $G \simeq \mathbb{Z}/2\mathbb{Z} \times \mathbb{Z}/15\mathbb{Z} \simeq \mathbb{Z}/30\mathbb{Z}$.\\
  (2-ii) $\phi(g_3) = g_3$かつ$\phi(g_5) = g_5^{-1}$の場合.
  このとき, $D_5$の関係式がみたされることから例題4.6.6と同様の議論により$\gen{g_2, g_5} \simeq D_5$である.
  明らかに$\gen{g_3} \cap D_5 = \{1_G\}$.
  また, $g_3$が$g_2$および$g_5$と可換であることから,
  $\gen{g_3}$の元と$\gen{g_2, g_5}$の元とは可換である.
  命題2.9.2の途中 ($hk = kh$) からの議論を適用することにより, $G \simeq \gen{g_3} \times \gen{g_2, g_5} \simeq \mathbb{Z}/3\mathbb{Z} \times D_5$.\\
  (2-iii) $\phi(g_3) = g_3^{-1}$かつ$\phi(g_5) = g_5$の場合.
  上と同様にして, $G \simeq \mathbb{Z}/5\mathbb{Z} \times D_3$.\\
  (2-iv) $\phi(g_3) = g_3^{-1}$かつ$\phi(g_5) = g_5^{-1}$の場合.
  このとき, $g_2$と$HK$の生成元は$D_{15}$の関係式をみたすので, 例題4.6.6と同様の議論により$G \simeq D_{15}$.\\
\end{ans}

\begin{ans}
  (1) シロー$p$部分群の個数の候補は$1, q$であるが, $p > q$のとき$q \not\equiv 1 \mod p$なので, シロー$p$部分群の個数は$1$.
  したがって$H$は正規部分群である. \\
  (2) シロー$q$部分群の個数の候補は$1, p, p^2$であるが、$K$が正規部分群でなく、また$p \not\equiv 1 \mod q$なので,
  シロー$q$部分群の個数は$p^2$である. \\
  (3) (2)が成り立つとき, シロー$q$部分群は巡回群なので, その共役は互いに単位元のみを共通元にもつ.
  したがって$G$の位数$q$の元の個数は$p^2(q - 1)$個である. よってシロー$p$部分群の元となりうるものは$p^2$個しかないことから,
  このとき$H$が正規部分群である.
\end{ans}

\begin{ans}
  (1) シロー$2$部分群の個数の候補は$1, 3$である. もし$1$ならシロー$2$部分群が正規部分群であるので,
  $G$は単純群ではない. $3$ならば, シロー$2$部分群のなす集合への共役による作用に対応して
  置換表現$G \rightarrow S_3$があり, その核は$G$の自明でない正規部分群である.
  したがって, いずれにしても$G$は単純群ではない.\\
  (2) シロー$3$部分群について (1) と同様の考察をおこなう.\\
  (3) シロー$2$部分群について (1) と同様の考察をおこなう.
\end{ans}

\begin{ans}
  (第2版第1刷には問題文の誤植があるので, 正誤表を参照.)
  ここまでの演習問題と, $p$群が非可換単純群でないことよりしたがう.
\end{ans}

\fakesubsection{4.6 生成元と関係式}

\begin{ans}
  $x^2$と$y$で生成される部分群を$H$とする.
  $(x^2)^{-1} = x^2$と
  \[
    x^2y = x(xy) = x(y^2x) = (xy)yx = (y^2x)yx = y^2(xy)x = y^2(y^2x)x = yx^2
  \]
  より, $H$の元は$y^ix^j\ (i = 0, 1, 2, j = 0, 2)$の形に書くことができる.
  例題4.6.7で示されたことから, これらは互いに異なる元であり, $\order{H} = 6$.
  また生成元が互いに可換なので$H$は可換群であり, 演習問題2.10.9より$H \simeq \Z{6}$.
\end{ans}

\begin{ans}
  $K = \genrel{x, y}{x^n = y^2 = 1,\ yxy = x^{-1}}$とおく.
  $D_n$は$K$と同じ関係式をみたす$2$元で生成されるので,
  定理4.6.5より全射準同型$\phi: K \rightarrow D_n$がある.
  例題4.6.6と同様にして$\order{K} \le 2n$であり, $\order{D_n} = 2n$なので,
  $\phi$は同型である.
\end{ans}

\begin{ans}
  \[
    y = x(xyx)x = xy^2x = (xyx)^2 = (y^2)^2 = y^4
  \]
  より, $y^3 = 1$. $y$の位数は$3$と$5$の公約数で$1$なので, $y = 1$.
\end{ans}

\begin{ans}
  (1) 計算するだけなので略.\\
  (2) $n$に関する数学的帰納法で証明する.\\
  (2-i) $n = 3$の場合.
  関係式より, $H_3$の元は, $x_1$と$x_2$が交互に現れる形をとる.
  また,
  \begin{align*}
    &x_1x_2x_1x_2 = x_1(x_2x_1x_2) = x_1(x_1x_2x_1) = x_2x_1 \\
    &x_2x_1x_2x_1 = x_2(x_1x_2x_1) = x_2(x_2x_1x_2) = x_1x_2
  \end{align*}
  より, 語の長さは$3$以下としてよい.
  したがって, $H_3$の元は$1$, $x_1$, $x_2$, $x_1x_2$, $x_2x_1$, $x_1x_2x_1 (=x_2x_1x_2)$のいずれかであるので,
  $\order{H_3} \le 6$.\\
  (2-ii) $n = k - 1$の場合を仮定して, $n = k$の場合にも成立することを示す.
  $H_{k}$の中で$x_1,..., x_{k-2}$で生成された部分群を$K$とする.
  $K$の生成元は$H_{k-1}$の関係式をみたすので,
  定理4.6.5より, $H_{k-1}$から$K$への全射準同型がある.
  帰納法の仮定と合わせて$\order{K} \le \order{H_{k-1}} \le (k-1)!$.

  $\tau_i = x_ix_{i+1} \cdots x_{k-1}$ ($i = 1,...,k$, ただし$\tau_k = 1$)
  とおいて, $H_k/K$の元 (左剰余類) のうち$\tau_i$たちを代表元に取るものの集合
  \[
    \{ \tau_iK \mid i = 1,...,k \}
  \]
  を考える. これらの剰余類に対して, $H_k$の生成元$x_j\ (j = 1,...k-1)$の左からの積による作用は次のようになる.
  \begin{align*}
    x_j\tau_iK = \begin{cases*}
      \tau_ix_jK = \tau_iK\ &(j \le i - 2) \\
      \tau_{i-1}K\ &(j = i - 1) \\
      \tau_{i+1}K\ &(j = i) \\
      \tau_ix_{j-1}K = \tau_iK\ &(j \ge i + 1)
    \end{cases*}
  \end{align*}
  ただし, $j \ge i + 1$の場合はとくに$x_{i+1}x_ix_{i+1}x_{i+2} = x_ix_{i+1}x_ix_{i+2} = x_ix_{i+1}x_{i+2}x_i$
  であることを用いている.
  これにより, $\{ \tau_iK \mid i = 1,...,k \}$は$H_k$の左からの積による作用で閉じていることがわかる.
  $1$つの剰余類をとるとその軌道は$H_k/K$全体と一致するので,
  $\{ \tau_iK \mid i = 1,...,k \} = H_k/K$.
  よって$\order{H_k/K} \le k$.
  上でみたように$\order{K} \le (k-1)!$であったから, $\order{H_k} \le k!$.

  以上により, $n$についての数学的帰納法で$\order{H_n} \le n!$が示せた.\\
  (3) (1) と定理4.6.5より, 全射準同型$H_n \rightarrow \mathfrak{S}_n$がある.
  これと (2) より, この全射準同型は同型である.
\end{ans}

\begin{ans}
  (1) $x^{-1} = x^6$, $y^{-1} = y^2$, $yx = x^2y$であることによる.\\
  (2) $\sigma = (1\ 2\ 3\ 4\ 5\ 6\ 7)$, $\tau = (2\ 3\ 5)(4\ 7\ 6)$とおけばよい.\\
  (3) (1) より $\order{G} \le 21$. また定理4.6.5より全射準同型$\phi: G \rightarrow \gen{\sigma, \tau}$があるが,
  $\phi^{-1}(\sigma)$の位数は$7$の倍数, $\phi^{-1}(\tau)$の位数は$3$の倍数なので,$\order{G}$は21の倍数である.
  よって$\order{G} = 21$.
\end{ans}

\begin{ans}
  前問と同様で,
  \begin{align*}
    \sigma &= (1\ 2\ 3\ 4\ 5\ 6\ 7\ 8\ 9\ 10\ 11\ 12\ 13) \\
    \tau &= (2\ 4\ 10)(3\ 7\ 6)(5\ 13\ 11)(8\ 9\ 12)
  \end{align*}
  で考える.
\end{ans}

\begin{ans}
  与えられた生成元と関係式で定義される群を$G$とする.
  $x^4 = y^4 = 1$と$yx = x^{-1}y$より,
  $G$の任意の元は$x^iy^j\ (i = 0, 1, 2, 3,\ j = 0, 1, 2, 3)$の形で書ける.
  この形で書いた元についてさらに$y^2$を$x^2$で置き換えることにより,
  任意の元は$x^iy^j\ (i = 0, 1, 2, 3,\ j = 0, 1)$の形で書ける.
  したがって$\order{G} \le 8$.
  一方, 四元数群は$G$と同じ生成元と関係式をもつから, $G$から四元数群への全射準同型がある.
  四元数群の位数は$8$なので, これは同型である.
\end{ans}

\begin{ans}
  (1) $\sigma\tau\nu = 1$は計算によって容易に確かめられる.
  $1$, $\sigma$, $\tau$, $\tau^2$, $\nu$, $\nu^2$, $\nu\tau\nu^{-1} = (1\ 3\ 4)$の$7$つはすべて異なる元であるから,
  $\order{\gen{\sigma, \tau, \nu}} = 12$. よって, $\gen{\sigma, \tau, \nu} = A_4$.
  (演習問題4.2.10の解答でも言及したように$\gen{\sigma, \tau} = \gen{\sigma, \nu} = A_4$であり,
  また明らかに$\gen{\tau, \nu} = A_4$でもある.)\\
  (2) $Sy = \{ y, y^2, y^3 = 1, y^2zy \}$であるが,
  $yz = x^{-1} = x =  z^2y^2$より,
  \begin{align*}
    y^2zy = y(yz)y = y(z^2y^2)y = yz^2 = (yz)z = z^2y^2z.
  \end{align*}
  よって$Sy \subset HS$より$HSy \subset HS$である. $Sz = \{ z, yz, y^2z, y^2z^2 \}$については,
  上で見たように$yz = z^2y^2$であることと,
  \begin{align*}
    y^2z^2 = z^3y^2z^2 = z(z^2y^2)z^2 = z(yz)z^2 = zy
  \end{align*}
  による.\\
  (3) (2) より, $G$の中で$y$と$z$で生成される部分群は$HS$に含まれる.
  この部分群は$G$と一致するので, $G = HS$である.
  $\order{H} \le 3$, $\order{S} \le 4$なので, $\order{G} \le 12$.\\
  (4) $A_4$は$G$と同じ生成元と関係式をもつので, $G$から$A_4$への全射準同型がある.
  $\order{A_4} = 12$より, これは同型.
\end{ans}

\begin{ans}
  (1) $\sigma\tau\nu = 1$は計算により確かめられる.
  演習問題2.3.9 (2) より, $\mathfrak{S}_4$は$\sigma$と$\nu$で生成される.\\
  (2) 前問と同様に$HSy, HSz \subset HS$を確かめればよい.
  とくに$y^2zy$, $y^2z^2y^2$, $yz$, $y^2z^2yz$が$HS$の元であることをみる.
  $yz = x^{-1} = x = z^3y^2$ (また$zy = y^2z^3$)  に注意すると,
  $yz$については明らかに$HS$の元であり,\\
  \begin{align*}
    &y^2zy = y(z^3y^2)y = yz^3 = z^3y^2z^2 \\
    &y^2z^2y^2 = y^2z^6y^2 = (y^2z^3)(z^3y^2) = (zy)(yz) = zy^2z \\
    &y^2z^2yz = y(yz)zyz = y(z^3y^2)zyz = (yz)(z^2y^2zyz) = (z^3y^2)(z^2y^2zyz) \\
    &= z^3y^2z^2y^2(zy)z = z^3y^2z^2y^2(y^2z^3)z = z^3y^2z^2y
  \end{align*}
  これにより, $HSy, HSz \subset HS$であることが確かめられたので,
  前問と同様に$G = HS$である.\\
  (4) 前問と同様に$\order{G} \le \order{H} \cdot \order{S} = 24$であり,
  一方$\mathfrak{S}_4$は$G$と同じ生成元と関係式をもつので, $G$から$\mathfrak{S}_4$への全射準同型がある.
  $\order{\mathfrak{S}_4} = 24$なので, これは同型.
\end{ans}

\begin{ans}
  (1) $\sigma\tau\nu = 1$は計算により確かめられる.
  $\order{\gen{\sigma, \tau, \nu}}$は$2$, $3$, $5$の公倍数なので, $30$または$60$.
  さらに, $\mu := \nu\tau\nu^{-1} = (2\ 1\ 4)$とおくと
  \begin{align*}
    \mu\sigma\mu^{-1} &= (1\ 4)(2\ 3) \\
    \mu^2\sigma\mu^{-2} &= (1\ 3)(2\ 4)
  \end{align*}
  より, $\gen{\sigma, \tau, \nu}$に位数$4$の部分群$\{1, (1\ 2)(3\ 4), (1\ 3)(2\ 4), (1\ 4)(2\ 3)\}$があることがわかる.
  したがって$\order{\gen{\sigma, \tau, \nu}}$は$4$の倍数であり, $60$であるから, $\gen{\sigma, \tau, \nu} = A_5$.\\
  (2) 演習問題4.6.8のヒントの方法で表を書くと次のようになる.\\
  \begin{tabular}{|c|ccc|ccccc|cccc|}
    \hline
    & y & y & y & z & z & z & z & z & y & z & y & z \\
    \hline
    1 & 2 & 3 & 1 & 1 & 1 & 1 & 1 & 1 & 2 & 3 & 1 & 1 \\
    2 & 3 & 1 & 2 & 3 & 4 & 5 & 6 & 2 & 3 & 4 & 6 & 2 \\
    3 & 1 & 2 & 3 & 4 & 5  & 6 & 2 & 3 & 1 & 1 & 2 & 3 \\
    4 & 6 & 7 & 4 &  5 & 6 & 2 & 3 & 4 & 6 & 2 & 3 & 4 \\
    5 & 8 & 9 & 5 & 6 & 2 & 3 & 4 & 5 & 8 & 7 & 4 & 5 \\
    6 & 7 & 4 & 6 & 2 & 3 & 4 & 5 & 6 & 7 & 9 & 5 & 6 \\
    7 & 4 & 6 & 7 & 9 & 10 & 11 & 8 & 7 & 4 & 5 & 8 & 7 \\
    8 & 9 & 5 & 8 & 7 & 9 & 10 & 11 & 8 & 9 & 10 & 11 & 8 \\
    9 & 5 & 8 & 9 & 10 & 11 & 8 & 7 & 9 & 5 & 6 & 7 & 9 \\
    10 & 11 & 12 & 10 & 11 & 8 & 7 & 9 & 10 & 11 & 8 & 9 & 10 \\
    11 & 12 & 10 & 11 & 8 & 7 & 9 & 10 & 11 & 12 & 12 & 10 & 11 \\
    12 & 10 & 11 & 12 & 12 & 12 & 12 & 12 & 12 & 10 & 11 & 12 & 12 \\
    \hline
  \end{tabular}\\
  したがって, $S = \{
    1,\ %1
    y,\ %2 = 1y
    y^2,\ %3 = 2y
    y^2z,\ %4 = 3z
    y^2z^2,\ %5 = 4z
    y^2z^3,\ %6 = 5z
    y^2z^3y,\ %7 = 6y
    y^2z^2y,\ %8 = 5y
    y^2z^2y^2,\ %9 = 8y
    y^2z^2y^2z,\\ %10 = 9z
    y^2z^2y^2z^2,\ %11 = 10z
    y^2z^2y^2z^2y\ %12 = 11y
    \}$
  とおくことにする. $H$を$A_5$のなかで$\nu$で生成された部分群とし, $Sy, Sz \subset HS$を示したい.
  まず$Sy \subset HS$を示すために問題となるのは, $S$の元のうち
  $y^2$, $y^2z$, $y^2z^3y$, $y^2z^2y^2$, $y^2z^2y^2z$, $y^2z^2y^2z^2y$ % 3, 4, 7, 9, 10, 12
  である. それぞれについて, 右から$y$をかけたものを考えると
  \begin{align*}
    &(y^2)y = 1 \\ % 3y = 1
    &(y^2z)y = y(yz)y = y(z^4y^2)y = (yz)z^3 = z^4(y^2z^3) \\ % 4y = 6
    &(y^2z^3y)y = z(z^4y^2z^3y^2) = z(yzz^3y^2) = z(y(z^4y^2)) = z(y(yz)) = z(y^2z) \\ % 7y = 4
    &(y^2z^2y^2)y = y^2z^2 \\ % 9y = 5
    &(y^2z^2y^2z)y = y(yz)zy^2zy = y(z^4y^2)zy^2zy = (yz)z^3y^2zy^2zy = z^4y^2z^3y^2zy^2zy \\
    &= z^4y^2z^2(zy)yzy^2zy = z^4y^2z^2(y^2z^4)yzy^2zy = z^4y^2z^2y^2z^3(zyzy)yzy \\
    &= z^4(y^2z^2y^2z^2) \\ % 10y = 11
    &(y^2z^2y^2z^2y)y = (y^2z^4)z^4(z^4y^2)z^2y^2 = (zy)z^4(yz)z^2y^2  = zy(z^4y^2)(y^2z^4)(z^4y^2)\\
    &= zy(yz)(zy)(yz) = z(y^2z^2y^2z) % 12y = 10
  \end{align*}
  よって$Sy \subset HS$である. $Sz \subset HS$については$S$の元のうち
  $y$, $y^2z^3$, $y^2z^3y$, $y^2z^2y$, $y^2z^2y^2z^2$, $y^2z^2y^2z^2y$  % 2, 6, 7, 8, 11, 12
  を調べればよい. 以下のようになる.
  \begin{align*}
    &yz = z^4y^2 \\ % 2z = 3
    &(y^2z^3)z = zy \\ % 6z = 2
    &(y^2z^3y)z = y^2z^2(zy)z = y^2z^2(y^2z^4)z = y^2z^2y^2 \\ % 7z = 9
    &(y^2z^2y)z = y^2z^2z^4y^2 = y(yz)y^2 = y(z^4y^2)y^2 = (yz)z^3y = (z^4y^2)z^3y = z^4(y^2z^3y) \\ % 8z = 7
    &(y^2z^2y^2z^2)z = (y^2z^4)z^3y^2z^3 = zyz^3y^2z^3 = zyz^4(z^4y^2)z^3 = zyz^4(yz)z^3 = zy(z^4y^2)y^2z^4 \\
    &= zy(yz)y^2z^4 = zy^2z^2(z^4y^2)z^4 = zy^2z^2(yz)z^4 = z(y^2z^2y)\\ % 11z = 8FIXME:
    &y^2z^2y^2z^2yz = z^4(zy)yz^2y^2z^2yz = z^4(y^2z^4)yz^2y^2z^2yz = z^4y^2z^4yz(y^2z^4)yz^2yz \\
    &= z^4y^2z^3(zyzy)yz^4yz^2yz = z^4y^2z^3yz^4yz^2yz = z^4y^2z^3yz^4(yz)zyz\\
    &= z^4y^2z^3yz^4(z^4y^2)zyz = z^4y^2z^3yz^3y = z^4y^2z^2(zy)z^3y = z^4y^2z^2(y^2z^4)z^3y \\
    &= z^4(y^2z^2y^2z^2y) % 12z = 12
  \end{align*}
  よって$Sz \subset HS$. あとは前問と同じ.
\end{ans}

\begin{ans}
  (1) $g \in Z(G)$は, $g$が生成元$x$および$y$と可換であることと同値である.
  $g = y^ix^j$とおくと, まず$g$が$x$と可換であることから,
  \begin{align*}
    &(y^ix^j)x = x(y^ix^j) \\
    &\Rightarrow y^ix = xy^i \\
    &\Rightarrow y^ix = y^{2i}x \\
    &\Rightarrow y^i = 1
  \end{align*}
  したがって中心の元は$g = x^j$とかける. さらにこれが$y$と可換でなければならないが,
  \begin{align*}
    &xy = y^2x \neq yx \\
    &x^2y = y^4x^2 = yx^2
  \end{align*}
  より, $Z(G) = \{1, x^2\}$.\\
  (2) (1) より$1$つの元からなる共役類は$\{1\}$と$\{x^2\}$のみである.
  $x(y^ix^j)x^{-1} = (y^{2i}x)x^jx^{-1} = y^{2i}x^j$,
  $y^{-1}(y^ix^j)y = y^{-1}y^i(y^{2^j}x^j) = y^{2^j+i-1}x^j$
  より, 共役類のなかで$x$の冪は等しい.
  このことからまず共役類の元の個数が$3$以下であり, 類等式の可能性としては
  \begin{align*}
    &12 = 1 + 1 + 2 + 2 + 2 + 2 + 2 \\
    &12 = 1 + 1 + 2 + 2 + 3 + 3
  \end{align*}
  のみであることがわかる.
  さらに, $x$の冪が等しい$3$元集合 (中心の元を含まない) がこれ以上分割されることはありえないので,
  可能な類等式は
  \[
    12 = 1 + 1 + 2 + 2 + 3 + 3
  \]
  のみであり, 中心以外の元については$x$の冪によって共役類に分かれる.\\
  (3) $(hgh^{-1})^n = 1 \Leftrightarrow g^n = 1$であるから, 共役類のなかで位数は等しい.
  よって代表元の位数のみ確認すればよい.
  代表元のうち, $y$の位数は明らかに$3$であり, $1$, $x$, $x^2$, $x^3$の位数は明らかに$3$ではない.
  残りの$yx^2$については
  \begin{align*}
    &(yx^2)^2 = yx^2yx^2 = y(y^4x^2)x^2 = y^2 \\
    &(yx^2)^3 = y^2(yx^2) = x^2
  \end{align*}
  となるので, これも位数は$3$ではない.
  したがって位数が$3$となるのは$y$, $y^2$の$2$つである.
\end{ans}

\begin{ans}
  (1) 演習問題4.2.6より, $\order{Z(\sigma)} = 2 \cdot (n - 2)!$.
  また, $\phi(\sigma)$が互換$a$個の積でかけるとすると$\order{Z(\phi(\sigma))} = 2^a \cdot a! \cdot (n - 2a)!$.
  $\phi$は自己同型なので, これらは等しい:
  \[
    2 \cdot (n - 2)! = 2^a \cdot a! \cdot (n - 2a)!
  \]
  $a (> 1)$を固定したとき(左辺)/(右辺)が$n$について単調増加であることに注意すると,
  $a = 2, 3$で等式をみたすのは$(n, a) = (6, 3)$のときのみであることが確かめられる.
  そこで$a > 3$として, (左辺) > (右辺)となることを示す.
  単調増加性より, $n = 2a$としてよい. すなわち
  \[
    2 \cdot (2a - 2)! > 2^a \cdot a!
  \]
  を示せばよいが, これは$a > 3$のとき
  \begin{align*}
    2 \cdot (2a - 2)! &= 2 \cdot (2a - 2) \cdots (a + 1) \cdot a! \\
    &> 2 \cdot 4^{a - 2} \cdot a! \\
    &= 2^{2a-3} \cdot a! \\
    &> 2^a \cdot a!
  \end{align*}
  であることからわかる.

  (2) (1) と同様に,
  \[
    3 \cdot (n - 3)! = 3^a \cdot a! \cdot (n - 3a)!
  \]
  を考える.
  これも$a (> 1)$を固定すると(左辺)/(右辺)が$n$について単調増加であることに注意すると,
  $a = 2$で等式が成り立つのは$n = 6$のときのみであることが確かめられる.
  そこで$a > 2$として(左辺) > (右辺)を示す.
  単調増加性より, $n = 3a$としてよい. すなわち
  \[
    3 \cdot (3a - 3)! > 3^a \cdot a!
  \]
  を示せばよいが, これは$a > 2$のとき
  \begin{align*}
    3 \cdot (3a - 3)! &= 3 \cdot (3a - 3) \cdots (a + 1) \cdot a! \\
    &> 3 \cdot 3^{2a - 3} \cdot a! \\
    &= 3^{2a - 2} \cdot a! \\
    &> 3^a \cdot a!
  \end{align*}
  であることからわかる.

  (3) $n = 2$のときは自明.
  $n = 3$のときは, 互換$\phi((1\ 2))$, $\phi((1\ 3))$の共通する数字を$a_1$とすればよい.
  $n \ge 4$のとき$1$, $i$, $j$, $k$をすべて異なる数字として,
  $\phi((1\ i))$と$\phi((1\ j))$と$\phi((1\ k))$の$3$つが共通する数字を持つことを示す.
  まず, $\phi((1\ i))\phi((1\ j)) = \phi((1\ j\ i))$が長さ$3$の巡回置換であることから,
  $\phi((1\ i))$と$\phi((1\ j))$は共通する数字を持つ互換である.
  同じことが任意のペアについて言える.
  そこで
  \begin{align*}
    \phi((1\ i)) &= (a\ b) \\
    \phi((1\ j)) &= (a\ b^\prime)
  \end{align*}
  とおいて, $\phi((1\ k))$が数字$a$を含まないと仮定すると, $\phi((1\ k)) = (b\ b^\prime)$でなければならない.
  ところが$\{ (1\ i), (1\ j), (1\ k) \}$が$\mathfrak{S}_4$と同型な部分群を生成する一方で,
  $\{ (a\ b), (a\ b^\prime), (b\ b^\prime) \}$は$\mathfrak{S}_3$と同型な部分群を生成するので,
  $\phi$が同型であることと矛盾する.
  したがって, $\phi((1\ i))$と$\phi((1\ j))$と$\phi((1\ k))$は共通する数字$a$を持つ.
  このことから, $i = 2,\cdots, n$について, $\phi((1\ i))$は共通する数字を持つことがわかる.
  これを$a_1$とおけばよい.\\
  (4) (3)より, $\phi$は$(1\ i)$の形の元に対して置換
  \[
    \sigma = \begin{pmatrix}
      1 & 2 & \cdots & n \\
      a_1 & a_2 & \cdots & a_n
    \end{pmatrix}
  \]
  による共役として作用する. $(1\ i)$の形の元たちは$\mathfrak{S}_n$を生成するから,
  $\phi$は内部自己同型$g \mapsto \sigma g \sigma^{-1}$である.\\
  (注意): $(1\ i)$の形の元たちが$\mathfrak{S}_n$を生成することは,
  $(i\ j) = (1\ i)(1\ j)(1\ i)$と命題2.1.14 (1) からわかる.
\end{ans}

\begin{ans}
  与えられた式の右辺を順に$x_1,..., x_5$に割り当てたとき演習問題4.6.4の関係式をみたすことが, 計算によって確かめられる.
  よって右辺によって生成される部分群$K$について, 定理4.6.5より全射準同型$H_6 \rightarrow K$がある.
  この全射準同型によって$x_1 \mapsto (1\ 2)(3\ 4)(5\ 6),..., x_5 \mapsto (1\ 4)(2\ 3)(5\ 6)$となっている.
  また, 演習4.6.4で存在が示された同型$\mathfrak{S}_6 \rightarrow H_6$は
  $(1\ 2) \mapsto x_1,..., (5\ 6) \mapsto x_5$をみたすものであったから,
  これらの合成により, 全射準同型$\phi: \mathfrak{S}_6 \rightarrow K$で与えられた式をみたすものが矛盾なく定義できる.
  あとは$K (= \mathrm{Im}\phi) = \mathfrak{S}_6$であることを示せばよい.

  (アイディア: 有限群の自己同型群は有限群なので, $\phi$が(外部)自己同型ならばある$i$について$\phi^i$が内部自己同型 (恒等写像を含む) になり,
  $\phi^i((1\ 2)),..., \phi^i((5\ 6))$は$\mathfrak{S}_6$を生成する互換となるはずである.
  実は$\mathrm{Aut}(\mathfrak{S}_6)/\mathrm{Inn}(\mathfrak{S}_6) \simeq \Z{2}$であることが知られていて,
  $\phi^2$は内部自己同型である.)

  計算によって
  \begin{align*}
    \phi((1\ 2)(3\ 4)(5\ 6)) &= (5\ 6) \\
    \phi((1\ 4)(2\ 5)(3\ 6)) &= (3\ 5) \\
    \phi((1\ 3)(2\ 4)(5\ 6)) &= (2\ 3) \\
    \phi((1\ 2)(3\ 6)(4\ 5)) &= (1\ 2) \\
    \phi((1\ 4)(2\ 3)(5\ 6)) &= (1\ 4)
  \end{align*}
  であることがわかる. これらの互換は
  \begin{align*}
    \tau = \begin{pmatrix}
      6 & 5 & 3 & 2 & 1 & 4 \\
      1 & 2 & 3 & 4 & 5 & 6
    \end{pmatrix}
  \end{align*}
  による内部自己同型$g \mapsto \tau g \tau^{-1}$によって$\mathfrak{S}_6$の既知の生成元
  $(1\ 2),..., (5\ 6)$にうつるので, $\mathfrak{S}_6$を生成する.
  よって$\mathrm{Im}\phi = \mathfrak{S}_6$.
\end{ans}

\fakesubsection{4.7 位数12の群の分類}


\begin{ans}
  $H$を$G$のシロー$2$部分群,
  $K$を$G$のシロー$p$部分群とする.
  $H \cap K = \{1\}$より,
  $\order{HK} = 2p$. よって$HK = G$である.
  もし$H$が$G$の正規部分群であれば,
  $HK = G$はアーベル群であることになるので, $H$は正規部分群ではない.
  したがって$H$の共役は$p$個あり, $G$に位数$2$の元は (少なくとも) $p$個ある.
  一方, $G (= HK)$の任意の元は$H$の生成元$r$と$K$の生成元$t$を用いて$r^it^j$($0 \le i \le 1$, $0 \le j \le p - 1$) と書けるが,
  これらのうち位数$2$となりうるのは$i = 1$のもののみである.
  したがって$rt$は位数$2$なので, $rtr = t^{-1}$.
  また, 明らかに$t^p = r^2 = 1$. したがって全射準同型
  \[
    L := \genrel{x, y}{x^p = y^2 = 1, yxy^{-1} = x^{-1}} \rightarrow G
  \]
  が存在する. 演習問題4.6.2より$L \simeq D_p$であり,
  $\order{G} = \order{D_p} = 2p$なので, $G \simeq D_p$.
\end{ans}

\begin{ans}
  $H$を$G$のシロー$7$部分群, $K$を$G$のシロー$3$部分群とする.
  $H \cap K = \{1\}$より$\order{HK} = 21$で, $HK = G$である.
  $H$の生成元を$a$, $K$の生成元を$b$とすると$G = \gen{a, b}$であり,
  $a^7 = b^3 = 1$である.
  また, $H$の共役の数の候補は$1$, $3$であるが, このうち$7$で割ったあまりが$1$であるのは$1$のみである.
  よって$H$は正規部分群で, $bab^{-1} \in H$.
  そこで$bab^{-1} = a^i$ ($i = 0, \cdots$ , 6) とおくと, $b^3 = 1$より$a = a^{i^3}$.
  これが成り立つのは$i = 1, 2, 4$であるが, $G$は非可換なので$i \neq 1$.
  もし$i = 4$ならば$b^2$を$b$と置きなおすことにすれば,
  $a^7 = b^3 = 1$と$bab^{-1} = a^2$が成り立つので, 全射準同型
  \[
    \genrel{x, y}{x^7 = y^3 = 1, yxy^{-1} = x^2} \rightarrow G
  \]
  が存在する. 位数が等しいことから, これは同型である.
\end{ans}

\begin{ans}
  前問と同様. $H$をシロー$13$部分群, $K$をシロー$3$部分群とすると, $HK = G$で,
  $H$の生成元$a$, $K$の生成元$b$により$G = \gen{a, b}$である.
  また$H$は正規部分群.
  $bab^{-1} = a^i$をみたすのは$i = 3, 9$だが, $i = 9$のときは$b^2$を$b$と置きなおせばよい.
\end{ans}

\begin{ans}
  (1) $G$が非可換であることによる.\\
  (2) $G$が非可換であることと, 演習問題2.4.8による.\\
  (3) $\gen{x}$は$G$の正規部分群なので, $yxy^{-1} = x^i$と書ける.
  また, $G = \gen{x, y}$なので, $G$の非可換性から$yxy^{-1} \neq x$.
  さらに$yxy^{-1}$と$x$の位数が等しいことから, $yxy^{-1} = x^{-1}$である.

  上の (3) における$y$の位数は$2$または$4$である.
  $y$の位数が$2$の場合は, $D_4$と同じ関係式が成り立つことと, 位数が等しいことから$G \simeq D_4$.
  $y$の位数が$4$の場合, 剰余類により$G = \gen{x} \cup y\gen{x}$と分解できることから, $y^2 \in \gen{x}$である.
  $y^2$の位数はこの場合$2$なので, $x^2 = y^2$である. したがって演習問題4.6.7の関係式が成り立ち, 位数が等しいことから$G$は四元数群と同型.
\end{ans}

\begin{ans}
  位数$18$の群$G$は以下のいずれかに同型であることを示す:\\
  (1) $\Z{18}$ ($\simeq \Z{2} \times \Z{9}$)\\
  (2) $\Z{2} \times \Z{3} \times \Z{3}$\\
  (3) $D_9$\\
  (4) $\mathfrak{S}_3 \times \Z{3}$\\
  %(4) $\genrel{x, y, z}{x^2 = y^3 = z^3 = 1, xy = y^2x, xz = zx, yz = zy}$ ($\simeq \mathfrak{S}_3 \times \Z{3}$)\\
  (5) $\genrel{x, y, z}{x^2 = y^3 = z^3 = 1, xy = y^2x, xz = z^2x, yz = zy}$\\

  (1)-(4)が位数18であることは明らか.
  (5)について, 任意の元が$z^iy^jx^k$と書けることから位数$18$以下である.
  さらに, $\mathfrak{S}_6$の部分群
  \[
    \gen{(1\ 2)(4\ 5), (1\ 2\ 3), (4\ 5\ 6)}
  \]
  の生成元が同じ関係式をみたすことから, (5)の群からこの部分群への全射準同型がある.
  $\order{\gen{(1\ 2)(4\ 5)}} = 2$, $\order{\gen{(1\ 2\ 3), (4\ 5\ 6)}} = 9$
  より, (5)の群の位数は$18$の倍数である. よって位数は$18$.

  後で用いるので, (4)も(5)のように生成元と関係式であらわしておくことにする.
  \[
    \genrel{x, y, z}{x^2 = y^3 = z^3 = 1, xy = y^2x, xz = zx, yz = zy}
  \]
  を考えると, (5)と同様にこの群の位数は$18$以下であることがわかる.
  \[
    \mathfrak{S}_3 \times \Z{3} = \gen{((1\ 2), \bar{0}), ((1\ 2\ 3), \bar{0}), (1_{\mathfrak{S}_3}, \bar{1})}
  \]
  の生成元が同じ関係式をみたすことと, 位数の比較により
  \[
    \mathfrak{S}_3 \times \Z{3} \simeq \genrel{x, y, z}{x^2 = y^3 = z^3 = 1, xy = y^2x, xz = zx, yz = zy}
  \]
  である.

  (1)-(5)が互いに同型でないことは, 以下で排他的な条件のもとで$G$が(1)-(5)のそれぞれと同型であることによって示される.

  まず, $G$の共通の性質についてみることにする.

  $H$を$G$のシロー$2$部分群, $K$をシロー$3$部分群とする.
  このとき$K$は指数$2$なので正規部分群であり, したがって$HK$は$G$の部分群である.
  $\order{HK}$は$\order{H} = 2$と$\order{K} = 9$の公倍数なので, $\order{HK} \ge 18$.
  よって$HK = G$. さらに$H \cap K = \{1_G\}$である.

  次に$K$の構造について調べる.
  $K$は位数$9 = 3^2$なので, 命題4.4.4よりアーベル群である.
  $K$に位数$9$の元があれば, $K \simeq \Z{9}$である.
  そうでなければ, $K$の単位元以外の元は位数$3$である.
  そこで位数$3$の元$a$, $b$で$b \notin \gen{a}$となるものをとると,
  $K$がアーベル群であることから$\gen{a}$, $\gen{b}$はともに正規部分群で,
  $\gen{a} \cap \gen{b} = \{ 1 \}$,
  また$\order{\gen{a}\gen{b}} \ge \order{\gen{a} \cup \{b\}} = 4$より
  $\gen{a}\gen{b} = K$である.
  よって命題2.9.2より$K \simeq \gen{a} \times \gen{b} \simeq \Z{3} \times \Z{3}$.
  これで$K \simeq \Z{9}$または$K \simeq \Z{3} \times \Z{3}$であることがわかった.

  以下, 場合分けして$G$が(1)-(5)のいずれかと同型であることを示す.\\

  \underline{場合1: $H$が$G$の正規部分群である場合.}
  このとき, 命題2.9.2より$G \simeq H \times K$である.したがって,
  $K$の構造によって
  \begin{align*}
    &(1)\ G \simeq \Z{2} \times \Z{9}\\
    &(2)\ G \simeq \Z{2} \times \Z{3} \times \Z{3}
  \end{align*}
  のいずれかである.\\

  \underline{場合2: $H$が$G$の正規部分群でない場合.}
  このとき, $h \in H$に対して$\phi(h) \in \mathrm{Aut}K$を
  $\phi(h)(k) = hkh^{-1}$と定める.
  もし$\phi(H)$が自明であるとすると, $H$が正規部分群であることになり, 矛盾.
  よって$\phi(H) \simeq \Z{2}$である.\\

  \underline{場合2-1: $K \simeq \Z{9}$の場合.}
  $H = \gen{a}$, $K = \gen{b}$とする.
  $\phi(H)$が非自明なので, $aba^{-1} \neq b$.
  $aba^{-1} = b^i$とおくと,
  $b = ab^ia^{-1} = (aba^{-1})^i = b^{i^2}$.
  これがなりたつ$i (\neq 1)$は$i = 8$のみであることが, 計算により確かめられる.
  $G = HK$は$a$, $b$で生成されて$D_9$と同じ関係式をみたすので,
  全射準同型$D_9 \rightarrow G$がある. 位数の比較により, $G \simeq D_9$.\\

  \underline{場合2-2: $K \simeq \Z{3} \times \Z{3}$の場合.}
  $H = \gen{a}$とする.
  さらにここでは簡単のため, 以下のような表記を用いることにする:
  \begin{itemize}
    \item $K$における群演算は和として記号'+'で表記する.
    \item $\mathrm{Aut}K$の単位元を$I$と書く.
    \item $\mathrm{Aut}K$の元の和を, 値の和で定義する: $(A + B)(k) := A(k) + B(k)$
    \item $K$を$\Z{3} \times \Z{3}$とみて両方の成分を$\bar{2}$倍する写像 ($\in \mathrm{Aut}K$) も$\bar{2}$と書く.
    \item $\phi(H)$の非単位元を$P$と書く. ($P \neq I$, $P^2 = I$.)
  \end{itemize}
  $I \pm P$は準同型であるから, $\mathrm{Im}(I \pm P)$は$K$の部分群.
  さらに
  \[
    \bar{2}I = (I + P) + (I - P)
  \]
  より, $\mathrm{Im}(I + P) \cup \mathrm{Im}(I - P)$が$K$を生成する.
  また,
  \[
    (I \pm P)(I \mp P) = I - P^2 = 0
  \]
  より, $\mathrm{Im}(I \pm P) \subset \Ker{I \mp P}$.
  このことから, $x \in \mathrm{Im}(I + P) \cap \mathrm{Im}(I - P)$ならば,
  \[
    x = (I + P)(x) + (I - P)(x) = 0 + 0 = 0
  \]
  すなわち$\mathrm{Im}(I + P) \cap \mathrm{Im}(I - P) = \{ 0 \}$であることもわかる.

  $\mathrm{Im}(I \pm P)$は$\{ 0 \}$, $\Z{3}$, $\Z{3} \times \Z{3}$のいずれかと同型である.
  以下, $\mathrm{Im}(I + P)$がいずれと同型であるかによって場合分けして考える.\\

  \underline{場合2-2-1: $\mathrm{Im}(I + P) \simeq \{ 0 \}$の場合.}
  このとき$K = \gen{b, c}$とすると, $(I + P)(b) = 0$より$P(b) = -b$.
  同様に$P(c) = -c$.
  $K$の群演算を積の形に戻すと, $aba^{-1} = b^{-1}$かつ$aca^{-1} = c^{-1}$である.
  $bc = cb$は明らか.
  したがって$G = HK = \gen{a, b, c}$は(5)の群の関係式をみたすので, 全射準同型
  \[
    \genrel{x, y, z}{x^2 = y^3 = z^3 = 1, xy = y^2x, xz = z^2x, yz = zy} \rightarrow G
  \]
  がある. 位数の比較により, これは同型である.\\

  \underline{場合2-2-2: $\mathrm{Im}(I + P) \simeq \Z{3}$の場合.}
  $\mathrm{Im}(I + P) \cap \mathrm{Im}(I - P) = \{ 0 \}$であることと,
  $\mathrm{Im}(I + P) \cup \mathrm{Im}(I - P)$が$K$を生成することから, $\mathrm{Im}(I - P) \simeq \Z{3}$.
  そこで, $\mathrm{Im}(I - P)$の生成元$b$, $\mathrm{Im}(I + P)$の生成元$c$とすると, $K = \gen{b, c}$であり,
  $b \in \Ker{I + P}$より$P(b) = -b$, $c \in \Ker{I - P}$より$P(c) = c$.
  $K$の群演算を積の形に戻すと, $aba^{-1} = b^{-1}$かつ$aca^{-1} = c$である.
  $bc = cb$は明らか.
  したがって$G = HK = \gen{a, b, c}$は(4)の群の関係式をみたすので, 全射準同型
  \[
    \genrel{x, y, z}{x^2 = y^3 = z^3 = 1, xy = y^2x, xz = zx, yz = zy} \rightarrow G
  \]
  がある. 位数の比較により, これは同型である.\\

  \underline{場合2-2-3: $\mathrm{Im}(I + P) = K$の場合.}
  このとき$K = \Ker{I - P}$, すなわち$P = I$となり, 矛盾.\\

  以上により, $G$は(1)-(5)のいずれかと同型であることが示された.\\

  (場合2-2に関する注意: 線形代数における一般論として, 標数$\neq 2$の体を係数とする正方行列$P$について,
  $P^2 = I$であるならば$P$は対角化可能である.
  ここではその事実の証明を念頭におきつつ, 体$\Z{3}$を係数とする$2$次正方行列の場合の証明を群論的に記述した.)
\end{ans}


\end{document}
