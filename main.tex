\documentclass{amsart}

\usepackage{amsmath, amssymb}
\usepackage{commath}
\usepackage{bm}
\usepackage{mathtools}

\theoremstyle{definition}
\newtheorem{ans}{}
\numberwithin{ans}{subsection}

\newcommand{\fakesection}[1]{%
  \par\refstepcounter{section}% Increase section counter
  \sectionmark{#1}% Add section mark (header)
  \addcontentsline{toc}{section}{\protect\numberline{\thesection}#1}% Add section to ToC
  % Add more content here, if needed.
}
\newcommand{\fakesubsection}[1]{%
  \par\refstepcounter{subsection}% Increase subsection counter
  \subsectionmark{#1}% Add subsection mark (header)
  \addcontentsline{toc}{subsection}{\protect\numberline{\thesubsection}#1}% Add subsection to ToC
  % Add more content here, if needed.
}

\DeclareMathOperator{\id}{id}
\DeclareMathOperator{\tr}{tr}

\newcommand{\transpose}[1]{{\prescript{t}{}{#1}}}
\newcommand{\Sp}[1]{\mathrm{Sp}(#1)}
\newcommand{\U}[1]{\mathrm{U}(#1)}

\DeclarePairedDelimiter{\gen}{\langle}{\rangle}
\DeclarePairedDelimiter{\order}{\lvert}{\rvert}


\begin{document}

\fakesection{第1章 集合論}
\fakesubsection{1.1 集合と論理の復習}

\begin{ans}
  $f: g,\ A: X,\ B: X$
\end{ans}

\begin{ans}
  (1) $f(S) = \{3, 4\}$
  (2) $f^{-1}(S_1) = \emptyset,\ f^{-1}(S_2) = \{1, 3, 4, 5\}$
  (3) $f(a) = 2 (\in B)$であるような$a \in A$が存在しないので, 全射ではない.
  (4) $f(1) = f(4)$なので, 単射ではない.
\end{ans}

\begin{ans}
  写像$f: A \rightarrow B$について考える.
  全射: $B$のすべての要素が集合$A$から来ている.\\
  単射: $A$の異なる要素は$B$の異なる要素に行く.
\end{ans}

\begin{ans}
  全射: $f(x) = x,\ x\sin x,\ x^3 - x$
  単射: $f(x) = x,\ e^x,\ \arctan x$
\end{ans}

\begin{ans}
  (逆写像$\Rightarrow$全単射) $f \circ g = \id_B$が全射なので, 次問より, $f$は全射.
  同様に$g \circ f = \id_A$が単射なので, $f$は単射.\\
  (全単射$\Rightarrow$逆写像)
  $b \in B$を任意に取ると, $f$の全射性から$f(a) = b$なる$a \in A$がある.
  しかも$f$の単射性により, このような$a$は一意である.
  そこで$g: B \rightarrow A$を$g(b) = a$と定めれば,
  $f \circ g(b) = f(a) = b$, $g \circ f(a) = g(b) = a$
  が成り立つので, $f \circ g = \id_B$, $g \circ f = \id_A$.
\end{ans}

\begin{ans}
  (1) $c \in C$を任意に取る.
  $g$の全射性から$g(b) = c$となる$b \in B$があり,
  さらに$f$の全射性から$f(a) = b$となる$a \in A$がある.
  すなわち$g \circ f(a) = c$となる$a \in A$がある.
  (2) $g \circ f(a) = g \circ f(a^\prime)$であるとする.
  このとき$g$の単射性から$f(a) = f(a^\prime)$,
  さらに$f$の単射性から$a = a^\prime$.
  (3) $g \circ f$が全射なので, 任意の$c \in C$に対して
  $g \circ f(a) = c$となる$a \in A$がある.
  すなわち$g(f(a)) = c$となる$f(a) \in B$がある.
  (4) $f(a) = f(a^\prime)$であるとする.
  このとき$g \circ f(a) = g \circ f(a^\prime)$であり,
  $g \circ f$の単射性より$a = a^\prime$.
\end{ans}

\begin{ans}
  (全射$\Rightarrow \forall S\ f(f^{-1}(S)) = S$).
  まず$f(f^{-1}(S)) \subset S$を示す.
  $b \in f(f^{-1}(S))$ならば,
  $f(a) = b,\ a \in f^{-1}(S)$となる$a$が存在する.
  逆像の定義から$f(a) \in S$なので, $b \in S$.
  (注意: ここでは$f$の全射性は用いなかった.)
  次に$f(f^{-1}(S)) \supset S$を示す.
  $b \in S$であるとすると, $f$の全射性から$f(a) = b$となる$a \in A$がある.
  $a \in f^{-1}(S)$なので, $b \in f(f^{-1}(S))$.
  \\
  ($\forall S\ f(f^{-1}(S)) = S \Rightarrow$全射).
  任意の$b \in B$に対して$\{b\} \subset f(f^{-1}(\{b\})) \subset f(A)$.
\end{ans}

\begin{ans}
  前問で示したように, $f(f^{-1}(S)) \subset S$なので,
  $f^{-1}(f(f^{-1}(S))) \subset f^{-1}(S)$.
  逆に$f^{-1}(f(f^{-1}(S))) \supset f^{-1}(S)$は次のようにして示される.
  $a \in f^{-1}(S)$ならば, $f(a) = b,\ b \in S$となる$b$が存在する.
  ふたたび$a \in f^{-1}(S)$より, $b \in f(f^{-1}(S))$.
  さらに$f(a) = b$より, $a \in f^{-1}(f(f^{-1}(S)))$.
\end{ans}

\begin{ans}
  (1) $x = 4.1$
  (2) $A = \mathbb{N},\ B = \emptyset$
  (3) $f: \{0, 1, 2\} \rightarrow \{0, 1\}$を
  $f(0) = f(2) = 0,\ f(1) = 1$と定めて,
  $S_0 = \{0, 1\},\ S_1 = \{1, 2\}$
  とすると, $f(S_0 \cap S_1) = f(\{1\}) = \{1\}$,
  $f(S_0) \cap f(S_1) = \{0, 1\} \cap \{0, 1\} = \{0, 1\}$.
\end{ans}

\begin{ans}
  (1) (d).
  $x = 4$が$A \Rightarrow B$の反例,
  $x = 1$が$B \Rightarrow A$の反例である.
  (2) (a).
  $B \Rightarrow A$は真だが,
  $x = 4$が$A \Rightarrow B$の反例である.
  (3) (b). $A \Rightarrow B$は真だが,
  $X = \mathbb{N},\ Y = \emptyset$が$B \Rightarrow A$の反例である
\end{ans}

\begin{ans}
  (1) $A$が成り立たないか$B$が成り立たない, かつ$C$が成り立たない
  (2) $A$が成り立ち, かつ$B$も$C$も成り立たない
  (3) $A$か$B$の一方のみが成り立つ
  (4) 自然数$n$があって, 任意の実数$x$に対して$x \le 0$または$\frac{1}{n} \le x$が成り立つ
  (5) ある$\varepsilon > 0$について, 任意の$\delta > 0$に対して$x, y \in [0, 1]$で
  $\abs{x - y} < \delta$かつ$\abs{f(x) - f(y)} \ge \varepsilon$
  を満たすものがある.
\end{ans}

\begin{ans}
  (1) $4 + 5 = 9 \ge 3$なので, 関係$R$がある.
  (2) $1 + (-1) = 0 < 3$なので, 関係$R$はない.
\end{ans}

\begin{ans}
  (1) $X = \mathbb{R},\ R = \{(x, y) \mid y = x^2\}$\\
  (2) $X = \mathbb{Z},\ R = \{(x, y) \mid x\text{は}y\text{で割り切れる.}\}$,
  この$R$は順序である.\\
  (3) $X = \mathbb{R}^n\setminus\{\bm{0}\},\ R = \{(\bm{x}, \bm{y}) \mid \exists \alpha \in \mathbb{R}\setminus\{0\},\ \bm{y} = \alpha \bm{x} \}$,
  この$R$は同値関係 (p47) である.
\end{ans}

\fakesubsection{1.2 well-definedと自然な対象}

\begin{ans}
  線形写像$f:V \rightarrow V$に対してトレース$\tr(f)$を
  $f$の行列表示$M = (m_{ij}) \in M_n(\mathbb{R})$を$1$つとって
  $\tr(f) = \tr(M) = \sum_{i = 1}^n m_{ii}$と定義したいとする.
  このとき, $\tr(f)$がwell-definedであること,
  すなわち行列表示のとり方によらず$\tr(f)$が定まるかどうかが問題となる.
  これがwell-definedであることは, 行列のトレースについて$\tr(AB) = \tr(BA)$が成り立つことを用いて,
  別の基底での$f$の行列表示$P^{-1}MP$について
  $\tr(P^{-1}MP) = \tr(MPP^{-1}) = \tr(M)$
  であることからしたがう.
\end{ans}

\begin{ans}
  解答例にないものでは, 双対空間$V^\ast$, テンソル積$V \otimes V$など.
\end{ans}

\fakesubsection{1.3 選択公理とツォルンの補題}

\begin{ans}
  (1) $X$の全順序部分集合を任意に取り,
  適当に添え字をつけて$\{(S_\lambda, f_\lambda)\}_{\lambda \in \Lambda}$
  と書くことにする.
  (上界の候補として) $S = \bigcup_{\lambda \in \Lambda} S_\lambda$と定めて,
  写像$f: S \rightarrow B$を,
  任意の$x \in S$に対して$x \in S_\lambda$なる$\lambda$を$1$つとって$f(x) = f_\lambda(x)$
  として定義したい. まずこれがwell-definedであることを見るために,
  別の$\lambda^\prime$に対して$x \in S_\lambda^\prime$であるとする.
  このとき$\{(S_\lambda, f_\lambda)\}_{\lambda \in \Lambda}$が全順序であることから,
  「$S_\lambda \subset S_{\lambda^\prime}$かつ$f_{\lambda^\prime}$は$f_\lambda$の拡張」(またはここで$\lambda$と$\lambda^\prime$を入れ替えたもの)
  が成り立つ. よって$f_\lambda(x) = f_{\lambda^\prime}(x)$なので,
  $f$は$\lambda$のとり方によらず, well-definedである.
  つぎに$f$が単射であることをみる.
  $f(x) = f(y)$とすると, ある$\lambda$, $\lambda^{\prime}$に対して
  $f_\lambda(x) = f_{\lambda^\prime}(y)$である.
  $\{(S_\lambda, f_\lambda)\}_{\lambda \in \Lambda}$が全順序であることから
  一方は他方の拡張であり, $f_\lambda(x) = f_\lambda(y)$または$f_{\lambda^\prime}(x) = f_{\lambda^\prime}(y)$.
  いずれの場合も$f_\lambda, f_{\lambda^\prime}$が単射であることから$x = y$.
  したがって$f$は単射である.
  さらに$(S, f)$が$\{(S_\lambda, f_\lambda)\}_{\lambda \in \Lambda}$の上界であることをみる.
  任意の$\lambda \in \Lambda$に対して
  $S_\lambda \subset S$であり, また$x \in S_\lambda$ならば$f(x) = f_\lambda(x)$が成り立つ.
  よって$(S_\lambda, f_\lambda) \le (S, f)$なので, $(S, f)$は上界である.
  以上により, $X$の任意の全順序部分集合が上界を持つことが示せたから,
  ツォルンの補題が適用でき, $X$に極大元が存在することがわかる.\\
  (2) $S_0 \neq A$かつ$f_0(S_0) \neq B$であると仮定して矛盾を導く.
  $a \in A \setminus S_0$と$b \in B \setminus f_0(S_0)$をとり,
  $f: S_0 \cup \{a\} \rightarrow B$を,
  $S_0$上では$f = f_0$, $f(a) = b$と定めると,
  $(S_0 \cup \{a\}, f) \in X$かつ$(S_0, f_0) \leq (S_0 \cup \{a\}, f)$かつ$(S_0, f_0) \neq (S_0 \cup \{a\}, f)$となって,
  $(S_0, f_0)$が極大元であることと矛盾する.
  したがって$S_0 = A$または$f_0(S_0) = B$.
\end{ans}

\fakesection{第2章 群の基本}
\fakesubsection{2.1 群の定義}

\begin{ans}
  $1$が単位元だが$0$の逆元が存在しないので群ではない.
  (単位元の存在に加えて結合法則が成り立つのでモノイドである.)
\end{ans}

\begin{ans}
  $0$が単位元なので,
  $a \in \mathbb{R}$の逆元が$b \in \mathbb{R}$であるならば,
  $a + b + ab = 0$より$(1 + a)b = -a$が成り立つはずである.
  ところが$a = -1$のときこの等式はどんな$b$に対しても成り立たないから,
  $-1$の逆元は存在しない.
  したがってこの演算では群にならない.
  (結合法則は成り立つのでモノイドである.)
\end{ans}

\begin{ans}
  略.
\end{ans}

\begin{ans}
  $((ab)c)d = (a(bc))d = a((bc)d)$.
\end{ans}

\begin{ans}
  両辺に左から$a^{-1}b^{-1}$, 右から$d^{-1}$を掛けると
  $c^{-1} = a^{-1}b^{-1}ab$.
  この逆元をとって$c = b^{-1}a^{-1}ba$.
\end{ans}

\begin{ans}
  (1) $\begin{pmatrix}
    4 & 1 & 2 & 3 \\
    1 & 2 & 3 & 4
  \end{pmatrix} =
  \begin{pmatrix}
    1 & 2 & 3 & 4 \\
    2 & 3 & 4 & 1
  \end{pmatrix} = (1\ 2\ 3\ 4)$ \\
  (2) $(2\ 4)^{-1}(1\ 3)^{-1} = (2\ 4)(1\ 3)$ \\
  (3) $\begin{pmatrix}
    1 & 3 & 4 & 2 \\
    4 & 2 & 3 & 1
  \end{pmatrix}\begin{pmatrix}
    1 & 2 & 3 & 4 \\
    1 & 3 & 4 & 2
  \end{pmatrix} = \begin{pmatrix}
    1 & 2 & 3 & 4 \\
    4 & 2 & 3 & 1
  \end{pmatrix} = (1\ 4)$ \\
  (4) $(2\ 4)(1\ 3)(1\ 3) = (2\ 4)$ \\
  (5) $\begin{pmatrix}
    1 & 2 & 3 & 4 \\
    1 & 3 & 4 & 2
  \end{pmatrix}\begin{pmatrix}
    1 & 2 & 3 & 4 \\
    4 & 1 & 2 & 3
  \end{pmatrix}\begin{pmatrix}
    1 & 3 & 4 & 2 \\
    1 & 2 & 3 & 4
  \end{pmatrix} = \begin{pmatrix}
    4 & 3 & 1 & 2 \\
    2 & 4 & 1 & 3
  \end{pmatrix}\begin{pmatrix}
    1 & 4 & 2 & 3 \\
    4 & 3 & 1 & 2
  \end{pmatrix}\\
  \begin{pmatrix}
    1 & 2 & 3 & 4 \\
    1 & 4 & 2 & 3
  \end{pmatrix} = \begin{pmatrix}
    1 & 2 & 3 & 4 \\
    2 & 4 & 1 & 3
  \end{pmatrix} = (1\ 2\ 4\ 3)$\\
  (6) $(2\ 4)(1\ 3)(1\ 3)(1\ 3)(2\ 4) = (2\ 4)(1\ 3)(2\ 4) = (1\ 3)(2\ 4)(2\ 4) = (1\ 3)$
\end{ans}

\fakesubsection{2.2 環・体の定義}

\begin{ans}
  (1) $9 \equiv 2 \pmod{7}$より, $\overline{2}$.
  (2) $-3 \equiv 4 \pmod{7}$より, $\overline{4}$.
  (3) $20 \equiv 6 \pmod{7}$より, $\overline{6}$.
  (4) $125 \equiv 6 \pmod{7}$より, $\overline{6}$.
  (5) $4^{32} \equiv 4^{30} \cdot 4^2 \equiv 64^{10} \cdot 16 \equiv 1^{10} \cdot 16 \equiv 2 \pmod{7}$より, $\overline{2}$.
\end{ans}

\begin{ans}
  (1) $34 \cdot 21 \equiv (-5) \cdot 21 \equiv -105 \equiv 12 \pmod{39}$,
  $12 \cdot 33 \equiv 12 \cdot (-6) \equiv -72 \equiv 6 \pmod{39}$より, $\overline{6}$
  (2) $18 \cdot 13 = 6 \cdot 39 \equiv 0 \pmod{39}$より, $\overline{0}$.
  (3) $16^2 \equiv 256 \equiv 22 \pmod{39}$, $16^4 = 22^2 \equiv 16 \pmod{39}$,
  $16^8 \equiv 16^2 \equiv 22 \pmod{39}$より, $\overline{22}$.
  (4) (3) に続けて$16^{16} \equiv 22^2 \equiv 16 \pmod{39}$, $16^{32} \equiv 16^2 \equiv 22 \pmod{39}$より,
  $16^{34} = 16^2 \cdot 16^{32} \equiv 22 \cdot 22 \equiv 16 \pmod{39}$. よって$\overline{16}$.
\end{ans}

\fakesubsection{2.3 部分群と生成元}

\begin{ans}
  命題2.3.2の条件と同値であることを確かめればよい.
  $H$が部分群であるとすると, 条件 (3) より$x^{-1} \in H$.
  さらに条件 (2) より$x^{-1}y \in H$.
  逆に, 群$G$の空でない部分集合$H$について
  「$x, y \in H \Rightarrow x^{-1}y \in H$」
  が成り立っているとする. $H$は空でないから, ある元$x_0 \in H$が存在する.
  よって, $1_G = x_0^{-1}x_0 \in H$. すなわち条件 (1) が成り立つ.
  $x \in H$ならば, $x, 1_G \in H$について$x^{-1} = x^{-1}1_G \in H$. すなわち条件 (2) が成り立つ.
  $x, y \in H$ならば, $x^{-1} \in H$なので$xy = (x^{-1})^{-1}y \in H$. すなわち条件 (3) が成り立つ.
\end{ans}

\begin{ans}
  $\transpose{I_{2n}}J_nI_{2n} = J_n$より, $I_{2n} \in \Sp{2n}$.
  $h_1, h_2 \in \Sp{2n}$ならば,
  $\transpose{(h_1h_2)}J_n(h_1h_2) = \transpose{h_2}(\transpose{h_1}J_nh_1)h_2 = \transpose{h_2}J_nh_2 = J_n$より
  $h_1h_2 \in \Sp{2n}$.
  また, $h \in \Sp{2n}$ならば,
  $\transpose{(h^{-1})}J_nh^{-1} = \transpose{(h^{-1})}(\transpose{h}J_nh)h^{-1} = J_n$.
  よって$h^{-1} \in \Sp{2n}$.
\end{ans}

\begin{ans}
  $\overline{AB} = \overline{A}\,\overline{B}$であることに注意する.
  (このことから, $\overline{g}^{-1} = \overline{g^{-1}}$であることも分かる.)
  $\transpose{\overline{I_n}I_n} = I_nI_n = I_n$より$I_n \in \U{n}$.
  $h_1, h_2 \in \U{n}$ならば,
  $\transpose{\overline{h_1h_2}}h_1h_2
  = \transpose{\overline{h_2}}\,\transpose{\overline{h_1}}h_1h_2
  = \transpose{\overline{h_2}}I_nh_2 = I_n$より$h_1h_2 \in \U{n}$.
  また, $h \in \U{n}$ならば,
  $\transpose{\overline{h^{-1}}}h^{-1}
  = \transpose{\overline{h}^{-1}}(\transpose{\overline{h}}h)h^{-1}
  = I_nI_n = I_n$.
  よって$h^{-1} \in \U{n}$.
\end{ans}

\begin{ans}
  (1) 明らかに$I_n \in B$.
  $b = (b_{ij}),\ b^\prime = (b^\prime_{ij})$をともにBの元とすると,
  $bb^\prime$の$(i, j)$成分は$\sum_{k = 1}^nb_{ik}b^\prime_{kj}$である.
  もし$i < j$ならば, どんな$k$に対しても$i < k$または$k < j$が成り立つので,
  各項が$0$となり$bb^\prime$の$(i, j)$成分は$0$である. すなわち$bb^\prime \in B$.
  $b \in B$とすると, $b$に
  (i) 行を$\lambda (\neq 0)$倍する
  (ii) 第$i$行に第$j$行の$\lambda$倍を足す ($i > j$),
  という$2$つの操作 (基本変形の一部) を繰り返すことにより, 単位行列にできる.
  これらの操作は$b$に$B$のある元を左から掛けることに対応しており, それらの積が$b^{-1}$に他ならない.
  $B$が積で閉じていることはすでに示していたから, $b^{-1} \in B$.\\
  (2) $n = 1$のとき$B = G = \mathbb{R}^\times$は可換群である.
  $n \ge 2$のとき, $B$は可換群ではない. たとえば$n = 2$では
  $\begin{pmatrix}
    1 & 0 \\
    1 & 1
  \end{pmatrix}\begin{pmatrix}
    1 & 0 \\
    0 & 2
  \end{pmatrix} \neq \begin{pmatrix}
    1 & 0 \\
    0 & 2
  \end{pmatrix}\begin{pmatrix}
    1 & 0 \\
    1 & 1
  \end{pmatrix}$.
  一般の$n \ge 2$の場合にも, たとえば単位行列の定数倍でない対角行列と,
  左下の成分がすべて$1$である行列との積は非可換である.
\end{ans}

\begin{ans}
  $1 \in \mathbb{R}_>$,
  また$x, y \in \mathbb{R}_>$ならば$xy \in \mathbb{R}_>$,
  また$x \in \mathbb{R}_>$ならば$x^{-1} = \frac{1}{x} \in {R}_>$.
\end{ans}

\begin{ans}
  $0 \notin \mathbb{R}_>$なので, 部分群ではない.
\end{ans}

\begin{ans}
  $H$の元$h$は, 絶対値が$1$なので$h = e^{i\theta}$と書ける.
  さらに, $h^n = 1$より$\theta = \frac{2\pi m}{n}\ (m \in \mathbb{Z})$である.
  $e^\frac{2\pi im}{n} = e^\frac{2\pi i m^\prime}{n}$が成り立つのは,
  $m^\prime = m + kn\ (k \in \mathbb{Z})$と書けるとき, またそのときに限るので,
  $H = \{1 = e^0, e^\frac{2\pi i}{n}, e^\frac{2\pi i \cdot 2}{n},..., e^\frac{2\pi i(n - 1)}{n}\}$.
  よって$H$の位数は$n$.
  また, $\gen{e^\frac{2\pi i}{n}} = H$なので, $H$は巡回群.
\end{ans}

\begin{ans}
  (1) $\mathfrak{S}_3$は可換群ではないから巡回群ではない.
  (別解: $\mathfrak{S}_3$には位数$2$の元が$3$つあるが, 巡回群$\mathbb{Z}/6\mathbb{Z}$には$1$つしかないから, 両者は一致しない.) \\
  (2) $\mathbb{Q}$が巡回群であると仮定して, 生成元が$r \neq 0$であるとする.
  このとき, $\mathbb{Q}$の任意の元は$nr\ (n \in \mathbb{Z})$という形で表せるはずだが, $\frac{r}{2} \in \mathbb{Q}$はそうではないので矛盾. \\
  (3) $\mathbb{Q}$の場合と同様. (別解: 巡回群の濃度はたかだか可算だが, $\mathbb{R}$の濃度は非可算なので一致しない.) \\
  (4) $\mathbb{Q}^\times$が巡回群であると仮定して, 生成元が$r$であるとする.
  明らかに$r \neq \pm 1$である. $-r \in \mathbb{Q}^\times$なので,
  $r^n = -r$, すなわち$r^{n-1} = -1$となるような$n \in \mathbb{Z}$が存在するはずである.
  ところがこの式を満たすような$r$は$-1$しかないので矛盾.
  \\
  (5) $\mathbb{Z} \times \mathbb{Z}$が巡回群であると仮定して, 生成元が$(x, y)$であるとする.
  $(nx, ny)\ (n \in \mathbb{Z})$が$\mathbb{Z} \times \mathbb{Z}$のすべての元を渉っている必要があるので,
  $x, y$それぞれが$\mathbb{Z}$の生成元でなれけばならない. よって少なくとも$x \neq 0$である.
  一方, $(x, y + 1) \in \mathbb{Z} \times \mathbb{Z}$を考えると,
  $(x, y + 1) = (nx, ny)$を満たす$n \in \mathbb{Z}$が存在するはずだが,
  $x = nx$かつ$x \neq 0$より$n = 1$. 一方第$2$成分は$y + 1 \neq y$となり矛盾.
\end{ans}

\begin{ans}
  (1) 数学的帰納法により示す. $n = 1$のときは明らか.
  $n = k$のときに正しいとして, $n = k + 1$の場合を示す.
  任意の$\sigma \in \mathfrak{S}_{k + 1}$をとり.
  $\sigma(1) = i$であるとする.
  $\tau = (1\ 2)(2\ 3)\cdots(i-1\ i)\sigma$とおけば$\tau(1) = 1$であるから,
  $n = k$の場合に帰着できる. よって$n = k + 1$の場合にも正しい. \\
  (2) $(1\ 2\ \cdots\ n)^{-(i-1)}(1\ 2)(1\ 2\ \cdots\ n)^{i-1} = (i\ i + 1)$なので,
  (1) よりこれらも$\mathfrak{S}_n$を生成する.
\end{ans}

\fakesubsection{2.4 元の位数}

\begin{ans}
  (1) 12, 144 (2) yes
\end{ans}

\begin{ans}
  (1)
  $395 = 1 \cdot 265 + 130$,
  $265 = 2 \cdot 130 + 5$,
  $130 = 26 \cdot 5$.
  よって$d = 5$.\\
  (2)
  $5
  = 265 - 2 \cdot 130
  = 265 - 2 \cdot (395 - 1 \cdot 265)
  = -2 \cdot 395 + 3 \cdot 265$.
  よって$(x, y) = (-2, 3)$が$1$つの解である.
\end{ans}

\begin{ans}
  (1) 位数が少ないので単純に総当りで調べる.
  $\overline{2}^{-1} = \overline{4}$,
  $\overline{3}^{-1} = \overline{5}$,
  $\overline{4}^{-1} = \overline{2}$,
  $\overline{5}^{-1} = \overline{3}$,
  $\overline{6}^{-1} = \overline{6}$.\\
  (2) $284x + 3y = 1$の$1$つの解をユークリッドの互除法によって求めると,
  $(x, y) = (-1, 95)$となる. よって$\overline{3}^{-1} = \overline{95}$.
\end{ans}

\begin{ans}
  $1,..., p^n$のうち$p^n$と互いに素でないのは$p$の倍数で, $p^{n-1}$個ある.
  よって位数は$p^n - p^{n-1}$.
\end{ans}

\begin{ans}
  $x^{35d} = 1$であることと$35d$が$60$の倍数であることは同値.
  さらにこのことは$7d$が$12$の倍数であることと同値.
  $7$と$12$は互いに素なので, これは$d$が$12$の倍数であることと同値である.
  よって位数は$12$.
\end{ans}

\begin{ans}
  $x^{nm} = 1$であることと$nm$が$d$の倍数であることは同値.
  さらにこのことは$nm/\gcd(n, d)$が$d/\gcd(n, d)$の倍数であることと同値.
  $n/\gcd(n, d)$と$d/\gcd(n, d)$は互いに素なので, これは$m$が$d/\gcd(n, d)$の倍数であることと同値である.
  よって位数は$d/\gcd(n, d)$.
\end{ans}

\begin{ans}
  位数$d$の群$G$を生成する元の位数は$d$である.
  前問より$\overline{n}$の位数は$d/\gcd(d, n)$であるから,
  $d$と互いに素であるような$n$についての$\overline{n}$を挙げればよい.
  たとえば$\mathbb{Z}/15\mathbb{Z}$では
  $\overline{1}, \overline{2}, \overline{4}, \overline{7}, \overline{8}, \overline{11}, \overline{13}, \overline{14}$.
  他も同様.
\end{ans}

\begin{ans}
  任意の$g, h \in G$に対し,
  $gh = (gh)^{-1} = h^{-1}g^{-1} = hg$.
\end{ans}

\begin{ans}
  (1) $g^2 = \begin{pmatrix}
    -1 & 0 \\
    0 & -1
  \end{pmatrix},\
  g^4 = \begin{pmatrix}
    1 & 0 \\
    0 & 1
  \end{pmatrix}$.
  よって$g$の位数は$4$の約数であり, かつ$1$, $2$ではないから, 位数は$4$.
  また$h$については,
  $h^2 = \begin{pmatrix}
    0 & 1 \\
    -1 & -1
  \end{pmatrix},\
  h^3 = \begin{pmatrix}
    -1 & 0 \\
    0 & -1
  \end{pmatrix},\
  h^6 = \begin{pmatrix}
    1 & 0 \\
    0 & 1
  \end{pmatrix}$.
  よって位数は$6$の約数であり, かつ$1$, $2$, $3$ではないから, 位数は$6$.\\
  (2) $gh = \begin{pmatrix}
    1 & 0 \\
    1 & 1
  \end{pmatrix} = \begin{pmatrix}
    1 & 0 \\
    0 & 1
  \end{pmatrix} + \begin{pmatrix}
    0 & 0 \\
    1 & 0
  \end{pmatrix}$
  と分解できるが, $\begin{pmatrix}
    0 & 0 \\
    1 & 0
  \end{pmatrix}^2 = 0$であることに注意すると, 任意の正の整数$n$に対して
  $(gh)^n = \begin{pmatrix}
    1 & 0 \\
    0 & 1
  \end{pmatrix} + n \cdot \begin{pmatrix}
    0 & 0 \\
    1 & 0
  \end{pmatrix} \neq \begin{pmatrix}
    1 & 0 \\
    0 & 1
  \end{pmatrix}$.
\end{ans}

\begin{ans}
  (1) $a^n = 1, b^m = 1$ならば, $(ab)^{nm} = a^{nm}b^{nm} = 1$.
  (2) $1$の位数は$0$なので$1 \in H$.
  また$x$の位数は$x^{-1}$の位数に等しいので, $x \in H$ならば$x^{-1} \in H$.
\end{ans}

\fakesubsection{2.5 準同型と同型}

\begin{ans}
  (1) 「…」が成り立つとする.
  $i_1 = m, i_2 = 0$とすると, $x^{i_1} = x^{i_2} = 1$.
  よって$y^{i_1} = y^{i_2}$, すなわち$y^m = 1$.
  命題2.4.18より, $m$は$n$の倍数である.
  逆に, $m$が$n$の倍数であるとする.
  $x^{i_1} = x^{i_2}$であるような任意の$i_1, i_2 \in \mathbb{Z}$に対して,
  $i_1 - i_2$は命題2.4.18より$m$の倍数であり, 仮定からこれは$n$の倍数でもある.
  よって$y^{i_1 - i_2} = 1$, すなわち$y^{i_1} = y^{i_2}$であるから, 「…」が成り立つ.
  以上により, 「…」が成り立つための必要十分条件は, $m$が$n$の倍数であることである.\\
  (2) $m$, $n$が (1) の性質を満たすことから, $\phi(x^i) = y^i$と定めると$\phi$はwell-definedである.
  $\phi(x^ix^j) = \phi(x^{i+j}) = y^{i+j} = y^iy^j = \phi(x^i)\phi(x^j)$
  より, $\phi$は準同型.
\end{ans}

\begin{ans}
  $\phi_n(gh) = (gh)^n = g^nh^n = \phi_n(gh)$.
\end{ans}

\begin{ans}
  (1) $g$の位数を$n$とすると,$\phi(g)^n = \phi(g^n) = \phi(1_G) = 1_H$.
  よって$\phi(g)$の位数は$n$の約数である.
  (2) $\phi(g)$の位数を$m$とすると, $\phi(g^m) = \phi(g)^m = 1_H$.
  $\phi$が単射なので, $g^m = 1_G$. よって$g$の位数は$m$の約数である.
  (1) と合わせて, $g$の位数と$\phi(g)$の位数は等しい.
\end{ans}

\begin{ans}
  $\mathbb{Z}/4\mathbb{Z}$には位数$4$の元があるが,
  $\mathbb{Z}/2\mathbb{Z} \times \mathbb{Z}/2\mathbb{Z}$にはないので,
  これらは同型ではない.
\end{ans}

\begin{ans}
  $n \ge 0$の場合を数学的帰納法により示す.
  $n = 0$の場合は明らかに真である.
  $n = k$で真であるとすると,
  $(xyx^{-1})^{k+1} = (xyx^{-1})^k(xyx^{-1}) = (xy^kx^{-1})(xyx^{-1}) = xy^{k+1}x^{-1}$より,
  $n = k + 1$についても真である.
  $n < 0$の場合は, $(xyx^{-1})^n = ((xyx^{-1})^{-n})^{-1} = (xy^{-n}x^{-1})^{-1} = xy^nx^{-1}$.
\end{ans}

\begin{ans}
  $\begin{pmatrix}
    a & b \\
    c & d
  \end{pmatrix}$を$\mathrm{GL}_2(\mathbb{R})$または$\mathrm{GL}_2(\mathbb{C})$の元として,
  \[
    \begin{pmatrix}
      1 & 1 \\
      0 & 1
    \end{pmatrix} = \begin{pmatrix}
      a & b \\
      c & d
    \end{pmatrix} \begin{pmatrix}
      1 & 0 \\
      1 & 1
    \end{pmatrix} \cdot \frac{1}{ad - bc} \begin{pmatrix}
      d & -b \\
      -c & a
    \end{pmatrix} = \frac{1}{ad - bc} \begin{pmatrix}
      a + b & b \\
      c + d & d
    \end{pmatrix} \begin{pmatrix}
      d & -b \\
      -c & a
    \end{pmatrix}
  \]
  であるとする. $(2, 1)$成分を比較すると$0 = d^2$であるから, $d = 0$.
  よって
  \[
    \begin{pmatrix}
      1 & 1 \\
      0 & 1
    \end{pmatrix} = - \frac{1}{bc} \begin{pmatrix}
      a + b & b \\
      c & 0
    \end{pmatrix} \begin{pmatrix}
      0 & -b \\
      -c & a
    \end{pmatrix} = - \frac{1}{bc} \begin{pmatrix}
      -bc & -b^2 \\
      0 & -bc
    \end{pmatrix} = \begin{pmatrix}
      1 & \frac{b}{c} \\
      0 & 1
    \end{pmatrix}
  \]
  $(1, 2)$成分を比較して$b = c$である.
  逆に, $d = 0$, $b = c$, $ad - bc \neq 0$となるように$a, b, c, d$を定めれば,
  \[
    \begin{pmatrix}
      1 & 1 \\
      0 & 1
    \end{pmatrix} = \begin{pmatrix}
      a & b \\
      c & d
    \end{pmatrix} \begin{pmatrix}
      1 & 0 \\
      1 & 1
    \end{pmatrix} \begin{pmatrix}
      a & b \\
      c & d
    \end{pmatrix}^{-1}
  \]
  が成り立つことが確かめられる.
  したがって$A$, $B$は$\mathrm{GL}_2(\mathbb{R})$では共役である.
  $d = 0$, $b = c$, $ad - bc \neq 0$という条件のもとで, もし$b \in \mathbb{R}$ならば$
  \begin{vmatrix}
    a & b \\
    c & d
  \end{vmatrix} = - b^2 < 0
  $なので, $A$, $B$は$\mathrm{SL}_2(\mathbb{R})$では共役でない.
  一方$b = i$とすれば$
  \begin{vmatrix}
    a & b \\
    c & d
  \end{vmatrix} = 1
  $なので, $\mathrm{SL}_2(\mathbb{C})$では共役である.
\end{ans}

\begin{ans}
  $G = \mathbb{Z}/15\mathbb{Z}$の場合のみを考える. (他も同様である.)
  自己準同型$\phi: G \rightarrow G$は,
  $1$つの生成元での値$\phi(\overline{1})$を与えれば一意に定まる.
  そうして定めた$\phi$が同型であるための必要十分条件は, $\phi(\overline{1})$がまた$G$の生成元であることである.
  $G$の生成元は演習問題2.4.7で求めたとおり,
  $\overline{1}, \overline{2}, \overline{4}, \overline{7}, \overline{8}, \overline{11}, \overline{13}, \overline{14}$
  の$8$通りなので, $\mathrm{Aut}(G)$の位数は$8$.
  また, $\overline{1} \mapsto \overline{k}$なる自己同型は$x \mapsto \overline{k}x$と書けるので,
  明らかに$\mathrm{Aut}(G)$はアーベル群である.
  有限アーベル群の基本定理 (定理4.8.1) より,
  $\mathrm{Aut}(G) \cong \mathbb{Z}/8\mathbb{Z}$
  または
  $\mathrm{Aut}(G) \cong \mathbb{Z}/2\mathbb{Z} \times \mathbb{Z}/4\mathbb{Z}$
  または
  $\mathrm{Aut}(G) \cong \mathbb{Z}/2\mathbb{Z} \times \mathbb{Z}/2\mathbb{Z} \times \mathbb{Z}/2\mathbb{Z}$
  であるが, $\mathrm{Aut}(G)$に単一の生成元がないこと,
  および$\overline{1} \mapsto \overline{2}$なる自己同型の位数が$4$であることが計算により確かめられるので,
  $\mathrm{Aut}(G) \cong \mathbb{Z}/2\mathbb{Z} \times \mathbb{Z}/4\mathbb{Z}$.
\end{ans}

\begin{ans}
  (1) $b(ab)b^{-1} = ba$.
  (2) $(ab)^n = 1$ならば$(ba)^n = (b(ab)b^{-1})^n = b(ab)^nb^{-1} = 1$.
  同様に$(ba)^n = 1$ならば$(ab)^n = 1$なので, $ab$と$ba$の位数は等しい.
\end{ans}

\begin{ans}
  $G$は互換$(1\ 2)$と$(2\ 3)$で生成されるので,
  $\mathrm{Aut}(G)$の元は$(1\ 2), (2\ 3)$の行き先を定めることで決定できる.
  これらは位数$2$の元なので, 同型写像による行き先は同じく位数$2$の異なる元でなければならず, 候補は$3$つある.
  したがって, $\mathrm{Aut}(G)$の位数はたかだか${}_3 \mathrm{P}_2 = 6$である.
  $G$の位数も$6$なので, $\phi$が単射 ($\mathrm{Ker}(\phi) = \{1\}$) であることを確かめれば, $\phi$が同型であることが分かる.
  $\sigma \in \mathrm{Ker}(\phi)$ならば, 任意の$\tau \in G$に対して
  $\sigma \tau \sigma^{-1} = \tau$, すなわち$\sigma$と$\tau$は可換である.
  互換$(1\ 2)$と可換な元は$1$と$(1\ 2)$のみであることが計算で確かめられ,
  また$(1\ 3)$についても同様なので, $G$のすべての元と可換であるのは$1$のみである.
  よって$\sigma = 1$.
\end{ans}

\fakesubsection{2.6 同値関係と剰余類}

\begin{ans}
  同値関係ではない.
  $\{(x, x) \mid x \in \mathbb{R}\} \subset R$なので, 反射律は満たされている.
  また
  $(a, b) \in \{(x, x) \mid x \in \mathbb{R}\}$ならば$(b, a) \in \{(x, x) \mid x \in \mathbb{R}\}$,
  $(a, b) \in \{(x, 2x) \mid x \in \mathbb{R}\}$ならば$(b, a) \in \{(2x, x) \mid x \in \mathbb{R}\}$,
  $(a, b) \in \{(2x, x) \mid x \in \mathbb{R}\}$ならば$(b, a) \in \{(x, 2x) \mid x \in \mathbb{R}\}$
  なので, 対称律も満たされている.
  しかし, 推移律は満たされていない.
  たとえば$(1, 2), (2, 4) \in R$であるが$(1, 4) \notin R$.
\end{ans}

\begin{ans}
  (反射律) $a = 1a1^{-1}$.
  (対称律) $a = gbg^{-1}$ならば$b = g^{-1}a(g^{-1})^{-1}$.
  (推移律) $a = gbg^{-1}$, $b = hch^{-1}$ならば$a = g(hch^{-1})g^{-1} = (gh)c(gh)^{-1}$.
\end{ans}

\begin{ans}
  系2.6.21よりしたがう.
\end{ans}

\begin{ans}
  $H \cap K$は$H$, $K$の部分群なので$\abs{H \cap K}$は$\abs{H}$と$\abs{K}$の公約数である.
  よって$\abs{H \cap K} = 1$なので$H \cap K = \{1_G\}$.
\end{ans}

\fakesubsection{2.7 両側剰余類}

\begin{ans}
  $\sigma \in G$とする. もし$\sigma(4) = 4$なら, $\sigma \in H$より
  $\sigma \in H1_GH$である.
  もし$\sigma(4) = i (i \neq 4)$なら, $\sigma (i\ 3)(3\ 4) \in H$すなわち$\sigma \in H(3\ 4)(i\ 3)$より
  $\sigma \in H(3\ 4)H$である.
  以上により, 任意の$\sigma \in G$に対して$\sigma \in H1_GH$または$\sigma \in H(3\ 4)H$が成り立つ.
  $(3\ 4) \notin H1_GH$より, $\{1_G, (3\ 4)\}$が$1$つの完全代表系である.
\end{ans}

\begin{ans}
  (1) $g$の第$i_n$行の$c\ (\in \mathbb{R})$倍を第$j\ (> i_n)$行に足すことは, $g$に$B$のある元を左から掛けることに等しい.
  また, $g$の第$n$列の$c\ (\in \mathbb{R})$倍を第$j\ (< n)$列に足すことは, $g$に$B$のある元を右から掛けることに等しい.
  したがって, $g$に対してこれらの操作 (行列の基本変形) を適当に繰り返すことによって条件を満たすような$h$に変形することを考えれば,
  ある$b_1, b_2 \in B$に対して$h = b_1gb_2$が成り立つ.\\
  (2) (1) において$n$について行った操作を$n-1, n-2,..., 1$について順に行い, さらに適当に対角行列を掛けると,
  $h = b_1gb_2$で, $h$の各列で$1$箇所だけが$1$で残りの成分はすべて$0$であるようなものがとれる.
  $h \in \mathrm{GL}_n(\mathbb{R})$なので, 各列で成分が$1$である行はすべて異なる.
  すなわち$h$は置換行列である.\\
  (3) $P_\sigma$の$(i, j)$成分が$\delta_{i\sigma(j)}$であることと,
  $j \neq n$ならば$b_{2, jn} = 0$であることから,
  \begin{align*}
    (b_1 P_\sigma b_2)_{in} &= \sum_{j, k}b_{ij}\delta_{j\sigma(k)}b_{2, kn} \\
    &= \sum_kb_{i\sigma(k)}b_{2, kn} \\
    &= b_{i\sigma(n)}b_{2, nn} \\
  \end{align*}
  ここで$i = \sigma(n)$とすると, (右辺) $\neq 0$であるが,
  これが$P_\tau$の$(\sigma(n), n)$成分 ($= \delta_{\sigma(n)\tau(n)}$) に等しいので,
  $\sigma(n) = \tau(n)$である.\\
  (4) (前半) (3) の式で$i \neq \sigma(n)$の場合を考えると,
  (右辺) $= 0$かつ$b_{2, nn} \neq 0$であるから$b_{1, i\sigma(n)} = 0$.\\
  (後半) 明らかに$P_\nu b_1 P_\nu^{-1}$は正則なので,
  その$(i, j)$成分が$i < j$のとき$0$であることを示せばよい.
  \begin{align*}
    (P_\nu b_1 P_\nu^{-1})_{ij} &= \sum_{k, l}P_{\nu, ik}b_{1, kl}(P_\nu^{-1})_{lj} \\
    &= \sum_{k, l}\delta_{i\nu(k)}b_{1, kl}\delta_{\nu(l)j} \\
    &= b_{1, \nu^{-1}(i)\nu^{-1}(j)}
  \end{align*}
  $\nu$の定め方から, $i < j$かつ$\nu^{-1}(i) > \nu^{-1}(j)$が成り立つのは$j = n$のときだけであり,
  このとき$\nu^{-1}(j) = \nu^{-1}(n) = \sigma(n)$.
  よって(4)より$j = n$のときも上式右辺は$0$である.\\
  (5) (3) の式で, $n$を$n-1$に置き換えたものを考えると,
  \begin{align*}
    (b_1 P_\sigma b_2)_{i(n-1)} &= \sum_{j, k}b_{1, ij}\delta_{j\sigma(k)}b_{2, k(n-1)} \\
    &= \sum_kb_{1, i\sigma(k)}b_{2, k(n-1)} \\
    &= b_{1, i\sigma(n-1)}b_{2, (n-1)(n-1)} + b_{1, i\sigma(n)}b_{2, n(n-1)} \\
  \end{align*}
  まず$i = \sigma(n-1)$とすると, (4) の前半より右辺第$2$項は$0$である.
  第$1$項$\neq 0$であるから右辺は全体として$\neq 0$であり,
  これが$P_\tau$の$(\sigma(n-1), n-1)$成分なので,
  (3)と同様に$\sigma(n-1) = \tau(n-1)$である.
  また, $i \neq \sigma(n-1), \sigma(n)$ならば, (4) の前半と同様にして$b_{1, i\sigma(n-1)} = 0$.
  以下同様に (3) の式で$n$を$n-2, n-3,..., 1$に置き換えたものを順に考えれば,
  $\sigma(n-2) = \tau(n-2),\ \sigma(n-3) = \tau(n-3),..., \sigma(1) = \tau(1)$が示せる.\\
  (メモ) (1)-(5) より両側剰余類$B\backslash G/B$の完全代表系はすべての置換行列からなる集合であることが分かった.\\
  (メモ2) (4) の後半を使っていないので, (5) は想定解答ではなさそうな気がする.
\end{ans}

\fakesubsection{2.8 正規部分群と剰余群}

\begin{ans}
  (1) 正規部分群でない. $g = (3\ 4)$, $h = (1\ 2)$とおくと$h \in H$で
  $ghg^{-1} = (3\ 4)(1\ 3)(3\ 4) = (1\ 4) \notin H$.\\
  (2) 正規部分群でない.
  $g = \begin{pmatrix}
    1 & 0 \\
    0 & 2
  \end{pmatrix}$, $h = \begin{pmatrix}
    \frac{1}{\sqrt{2}} & \frac{1}{\sqrt{2}} \\
    - \frac{1}{\sqrt{2}} & \frac{1}{\sqrt{2}}
  \end{pmatrix}$とおくと
  \begin{align*}
    ghg^{-1} &= \begin{pmatrix}
      1 & 0 \\
      0 & 2
    \end{pmatrix}\begin{pmatrix}
      \frac{1}{\sqrt{2}} & \frac{1}{\sqrt{2}} \\
      - \frac{1}{\sqrt{2}} & \frac{1}{\sqrt{2}}
    \end{pmatrix}\begin{pmatrix}
      1 & 0 \\
      0 & \frac{1}{2}
    \end{pmatrix} = \begin{pmatrix}
      \frac{1}{\sqrt{2}} & \frac{1}{2\sqrt{2}} \\
      - \sqrt{2} & \frac{1}{\sqrt{2}}
    \end{pmatrix} \\
    \transpose{(ghg^{-1})}(ghg^{-1}) &= \begin{pmatrix}
      \frac{1}{\sqrt{2}} & - \sqrt{2} \\
      \frac{1}{2\sqrt{2}} & \frac{1}{\sqrt{2}}
    \end{pmatrix}\begin{pmatrix}
      \frac{1}{\sqrt{2}} & \frac{1}{2\sqrt{2}} \\
      - \sqrt{2} & \frac{1}{\sqrt{2}}
    \end{pmatrix} \neq I_n
  \end{align*}
  したがって$ghg^{-1} \notin H$.\\
  (3) 正規部分群でない.
  $g = \begin{pmatrix}
    1 & 0 \\
    0 & i
  \end{pmatrix}$, $h = \begin{pmatrix}
    1 & 1 \\
    0 & 1
  \end{pmatrix}$とおくと,
  \[
    ghg^{-1} = \begin{pmatrix}
      1 & 0 \\
      0 & i
    \end{pmatrix}\begin{pmatrix}
      1 & 1 \\
      0 & 1
    \end{pmatrix}\begin{pmatrix}
      1 & 0 \\
      0 & -i
    \end{pmatrix} = \begin{pmatrix}
      1 & -i \\
      0 & 1
    \end{pmatrix} \notin \mathrm{GL}_2(\mathbb{R})
  \]
  (4) 正規部分群である. $(1\ 2)(3\ 4)(1\ 3)(2\ 4) = (1\ 4)(2\ 3)$より,
  $H$は$(1\ 2)(3\ 4)$と$(1\ 3)(2\ 4)$で生成される.
  また演習問題2.3.9より, $G$は$(1\ 2\ 3\ 4)$と$(1\ 2)$で生成される.
  命題2.8.7より, $h = (1\ 2)(3\ 4), (1\ 3)(2\ 4)$と$g = (1\ 2\ 3\ 4), (1\ 2)$の組み合わせについて
  $ghg^{-1} \in H$であることを見ればよい:
  \begin{align*}
    (1\ 2\ 3\ 4)(1\ 2)(3\ 4)(1\ 2\ 3\ 4)^{-1} = (1\ 4)(2\ 3) \in H\\
    (1\ 2)(1\ 2)(3\ 4)(1\ 2)^{-1} = (1\ 2)(3\ 4) \in H\\
    (1\ 2\ 3\ 4)(1\ 3)(2\ 4)(1\ 2\ 3\ 4)^{-1} = (1\ 3)(2\ 4) \in H \\
    (1\ 2)(1\ 3)(2\ 4)(1\ 2)^{-1} = (1\ 4)(2\ 3) \in H
  \end{align*}
  (5) 正規部分群である. $g = \begin{pmatrix}
    g_{11} & 0 \\
    g_{21} & g_{22}
  \end{pmatrix}$, $h = \begin{pmatrix}
    h_{11} & 0 \\
    h_{21} & h_{11}
  \end{pmatrix}$とおくと,
  $ghg^{-1}$の$(1, 1)$成分は$g_{11}h_{11}g_{11}^{-11} = h_{11}$,
  $(2, 2)$成分は$g_{22}h_{11}g_{22}^{-1} = h_{11}$であるから,
  $ghg^{-1} \in H$.
\end{ans}

\begin{ans}
  $h \in H$とする. $g \in H$ならば$ghg^{-1} \in H$は明らか.
  $g \notin H$ならば, $G = H \sqcup gH = H \sqcup Hg$より, $gH = Hg$すなわち$gHg^{-1} = H$.
  したがって任意の$g \in G$, $h \in H$に対して$ghg^{-1} \in H$.
\end{ans}

\begin{ans}
  ($N_1N_2$が部分群であること)
  $1 \in N_1N_2$.
  $N_1N_2$の任意の$2$つの元$h_1h_2, h_1^\prime h_2^\prime$ ($h_1, h_1^\prime \in N_1$かつ$h_2, h_2^\prime \in N_2$)について
  $h_1 h_2 h_1^\prime h_2^\prime = h_1 (h_2 h_1^\prime h_2^{-1}) h_2 h_2^\prime \in N_1N_2$.
  また$N_1N_2$の任意の元$h_1h_2$ ($h_1 \in N_1$, $h_2 \in N_2$)に対して$h_1h_2^{-1}h_1^{-1} \in N_2$より$(h_1h_2)^{-1} = h_2^{-1}h_1^{-1} \in N_1N_2$.\\
  (正規部分群であること) 任意の$g \in G$に対して$gN_1N_2g^{-1} = (gN_1g^{-1})(gN_2g^{-1}) \subset N_1N_2$.
\end{ans}

\begin{ans}
  $\mathfrak{S}_3$の位数は$6$なので, 部分群の位数は$1, 2, 3, 6$のいずれかである.
  位数$1$の部分群は$\{1\}$, 位数$6$の部分群は$\mathfrak{S}_3$である.
  $2$と$3$は素数なので, これらに対応する部分群は巡回群のみである.
  したがって, 部分群は
  $\{1\}$, $\mathfrak{S}_3$,
  $\gen{(1\ 2)}$, $\gen{(2\ 3)}$, $\gen{(1\ 3)}$,
  $\gen{(1\ 2\ 3)}$
  で尽くされる.
  これらの部分群のうち, $\{1\}$と$\mathfrak{S}_3$は明らかに正規部分群である.
  また$\gen{(1\ 2\ 3)}$は指数$2$の部分群であるから,
  演習問題2.8.2より正規部分群である.
  これら以外は正規部分群ではない. 例えば
  $(2\ 3)(1\ 2)(2\ 3)^{-1} = (1\ 3) \notin \gen{(1\ 2)}$
  より$\gen{(1\ 2)}$が正規部分群でないことが分かる. 他も同様.
\end{ans}

\begin{ans}
  四元数群を$G$と書くことにする.
  $G$の位数は$8$なので, 部分群の位数は$1, 2, 4, 8$のいずれかである.
  このうち位数が$1, 8$であるのは$\{1\}, G$である.
  それ以外の位数が$2$または$4$の部分群$H$について, もし$i \in H$ならば, $\gen{i} \subset H$であるが,
  $\order{\gen{i}} = 4$より$H = \gen{i}$である.
  $j, k$についても同様.
  $i, j, k \notin H$で$H \neq \{1\}$であるようなものは$H = \gen{-1}$のみである.
  以上により, $G$の部分群は$\{1\}, \gen{-1}, \gen{i}, \gen{j}, \gen{k}, G$である.
  これらの部分群のうち, 明らかに$\{1\}$と$G$は正規部分群である.
  また$\gen{i}, \gen{j}, \gen{k}$は指数$2$なので,
  演習問題2.8.2より正規部分群である.
  $\gen{-1}$が正規部分群であることも容易に確かめられる.
  以上により, $G$の部分群$\{1\}, \gen{-1}, \gen{i}, \gen{j}, \gen{k}, G$はすべて正規部分群である.
\end{ans}

\fakesubsection{2.9 群の直積}

\begin{ans}
  容易なので略.
\end{ans}

\begin{ans}
  まず$\phi_1: G_1 \rightarrow G_1$を定める. $g_1 \in G_1$に対して,
  $\phi(g_1, 1_{G_2}) = (g_1^\prime, g_2^\prime)$であるとする.
  $(g_1, 1_{G_2})$の位数は$n_1$の約数であり,
  $\phi(g_1, 1_{G_2}) = (g_1^\prime, g_2^\prime)$の位数は更にその約数であるから,
  $g_2^\prime$の位数は$n_1$の約数である.
  一方$G_2$の位数は$n_1$と互いに素であったから,
  $g_2^\prime$の位数は$1$, すなわち$g_2^\prime = 1_{G_2}$である.
  そこで$\phi_1(g_1) = g_1^\prime$として$\phi_1: G_1 \rightarrow G_1$を定めれば,
  $\phi_1$は$\phi(g_1, 1_{G_2}) = (\phi_1(g_1), 1_{G_2})$をみたす.
  このように定めた$\phi_1$が準同型であることは,
  $(\phi_1(gh), 1_{G_2})
  = \phi(gh, 1_{G_2})
  = \phi(g, 1_{G_2})\phi(h, 1_{G_2})
  = (\phi_1(g), 1_{G_2})(\phi_1(h), 1_{G_2})
  = (\phi_1(g)\phi_1(h), 1_{G_2})$
  からしたがう.
  同様にして$\phi(1_{G_1}, g_2) = (1_{G_1}, \phi_2(g_2))$をみたす
  準同型$\phi_2: G_2 \rightarrow G_2$を定めることができる.
  この$\phi_1, \phi_2$について,
  $\phi(g_1, g_2)
  = \phi(g_1, 1_{G_2})\phi(1_{G_1}, g_2)
  = (\phi_1(g_1), 1_{G_2})(1_{G_1}, \phi_2(g_2))
  = (\phi_1(g_1), \phi_2(g_2))$.
\end{ans}

\begin{ans}
  (1) $15 = 1 \cdot 8 + 7,\ 8 = 1 \cdot 7 + 1$より,
  $1 = 8 - 1 \cdot 7 = 8 - 1 \cdot (15 - 1 \cdot 8) = - 15 + 2 \cdot 8$.
  よって$(x, y) = (-1, 2)$が$15x + 8y = 1$の解の$1$つであるから,
  $15 \cdot (-1) \cdot 5 + 8 \cdot 2 \cdot 2 = -43$.
  (2) $35 = 1 \cdot 24 + 11,\ 24 = 2 \cdot 11 + 2,\ 11 = 5 \cdot 2 + 1$より,
  $1 = 11 - 5 \cdot 2
  = 11 - 5 \cdot (24 - 2 \cdot 11)
  = (-5) \cdot 24 + 11 \cdot 11
  = (-5) \cdot 24 + 11 \cdot (35 - 1 \cdot 24)
  = 11 \cdot 35 + (-16) \cdot 24$.
  よって$(x, y) = (11, -16)$が$35x + 24y = 1$の解の$1$つであるから,
  $35 \cdot 11 \cdot 5 + 24 \cdot (-16) \cdot 4 = 389$.
\end{ans}

\fakesubsection{2.10 準同型定理}

\begin{ans}
  $\phi: G \rightarrow H_2$を$\phi(re^{i\theta}) = r$と定義すると
  $\phi$は全射準同型であり, $\mathrm{Ker}\phi = H_1$である.
  したがって準同型定理より$G/H_1 \cong H_2$.
  また$\psi: G \rightarrow H_1$を$\psi(re^{i\theta}) = e^{i\theta}$と定義すると
  $\psi$は全射準同型であり, $\mathrm{Ker}\psi = H_2$である.
  したがって準同型定理より$G/H_2 \cong H_1$.
\end{ans}

\begin{ans}
  $\phi: \mathbb{R} \rightarrow \mathbb{R}/a\mathbb{Z}$を
  $\phi(x) = ax + a\mathbb{Z}$と定めると, $\phi$は全射準同型で
  $\mathrm{Ker}\phi = \mathbb{Z}$である. よって準同型定理より
  $\mathbb{R}/\mathbb{Z} \cong \mathbb{R}/a\mathbb{Z}$.
\end{ans}

\end{document}
