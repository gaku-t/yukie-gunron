\documentclass{amsart}

\usepackage{amsmath, amssymb}

\theoremstyle{definition}
\newtheorem{ans}{}
\numberwithin{ans}{subsection}

\newcommand{\fakesection}[1]{%
  \par\refstepcounter{section}% Increase section counter
  \sectionmark{#1}% Add section mark (header)
  \addcontentsline{toc}{section}{\protect\numberline{\thesection}#1}% Add section to ToC
  % Add more content here, if needed.
}
\newcommand{\fakesubsection}[1]{%
  \par\refstepcounter{subsection}% Increase subsection counter
  \subsectionmark{#1}% Add subsection mark (header)
  \addcontentsline{toc}{subsection}{\protect\numberline{\thesubsection}#1}% Add subsection to ToC
  % Add more content here, if needed.
}

\DeclareMathOperator{\id}{id}


\begin{document}

\fakesection{第1章 集合論}
\fakesubsection{1.1 集合と論理の復習}

\begin{ans}
  $f: g,\ A: X,\ B: X$
\end{ans}

\begin{ans}
  (1) $f(S) = \{3, 4\}$
  (2) $f^{-1}(S_1) = \emptyset,\ f^{-1}(S_2) = \{1, 3, 4, 5\}$
  (3) $f(a) = 2 (\in B)$であるような$a \in A$が存在しないので, 全射ではない.
  (4) $f(1) = f(4)$なので, 単射ではない.
\end{ans}

\begin{ans}
  写像$f: A \rightarrow B$について考える.
  全射: $B$のすべての要素が集合$A$から来ている.\\
  単射: $A$の異なる要素は$B$の異なる要素に行く.
\end{ans}

\begin{ans}
  全射: $f(x) = x,\ x\sin x,\ x^3 - x$
  単射: $f(x) = x,\ e^x,\ \arctan x$
\end{ans}

\begin{ans}
  (逆写像$\Rightarrow$全単射) $f \circ g = \id_B$が全射なので, 次問より, $f$は全射.
  同様に$g \circ f = \id_A$が単射なので, $f$は単射.\\
  (全単射$\Rightarrow$逆写像)
  $b \in B$を任意に取ると, $f$の全射性から$f(a) = b$なる$a \in A$がある.
  しかも$f$の単射性により, このような$a$は一意である.
  そこで$g: B \rightarrow A$を$g(b) = a$と定めれば,
  $f \circ g(b) = f(a) = b$, $g \circ f(a) = g(b) = a$
  が成り立つので, $f \circ g = \id_B$, $g \circ f = \id_A$.
\end{ans}

\fakesubsection{1.2 well-definedと自然な対象}

\begin{ans}
  test
\end{ans}

\fakesubsection{1.3 選択公理とツォルンの補題}

\begin{ans}
  test
\end{ans}

\end{document}
