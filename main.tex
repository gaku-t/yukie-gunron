\documentclass{amsart}

\usepackage{amsmath, amssymb}
\usepackage{commath}
\usepackage{bm}

\theoremstyle{definition}
\newtheorem{ans}{}
\numberwithin{ans}{subsection}

\newcommand{\fakesection}[1]{%
  \par\refstepcounter{section}% Increase section counter
  \sectionmark{#1}% Add section mark (header)
  \addcontentsline{toc}{section}{\protect\numberline{\thesection}#1}% Add section to ToC
  % Add more content here, if needed.
}
\newcommand{\fakesubsection}[1]{%
  \par\refstepcounter{subsection}% Increase subsection counter
  \subsectionmark{#1}% Add subsection mark (header)
  \addcontentsline{toc}{subsection}{\protect\numberline{\thesubsection}#1}% Add subsection to ToC
  % Add more content here, if needed.
}

\DeclareMathOperator{\id}{id}
\DeclareMathOperator{\tr}{tr}


\begin{document}

\fakesection{第1章 集合論}
\fakesubsection{1.1 集合と論理の復習}

\begin{ans}
  $f: g,\ A: X,\ B: X$
\end{ans}

\begin{ans}
  (1) $f(S) = \{3, 4\}$
  (2) $f^{-1}(S_1) = \emptyset,\ f^{-1}(S_2) = \{1, 3, 4, 5\}$
  (3) $f(a) = 2 (\in B)$であるような$a \in A$が存在しないので, 全射ではない.
  (4) $f(1) = f(4)$なので, 単射ではない.
\end{ans}

\begin{ans}
  写像$f: A \rightarrow B$について考える.
  全射: $B$のすべての要素が集合$A$から来ている.\\
  単射: $A$の異なる要素は$B$の異なる要素に行く.
\end{ans}

\begin{ans}
  全射: $f(x) = x,\ x\sin x,\ x^3 - x$
  単射: $f(x) = x,\ e^x,\ \arctan x$
\end{ans}

\begin{ans}
  (逆写像$\Rightarrow$全単射) $f \circ g = \id_B$が全射なので, 次問より, $f$は全射.
  同様に$g \circ f = \id_A$が単射なので, $f$は単射.\\
  (全単射$\Rightarrow$逆写像)
  $b \in B$を任意に取ると, $f$の全射性から$f(a) = b$なる$a \in A$がある.
  しかも$f$の単射性により, このような$a$は一意である.
  そこで$g: B \rightarrow A$を$g(b) = a$と定めれば,
  $f \circ g(b) = f(a) = b$, $g \circ f(a) = g(b) = a$
  が成り立つので, $f \circ g = \id_B$, $g \circ f = \id_A$.
\end{ans}

\begin{ans}
  (1) $c \in C$を任意に取る.
  $g$の全射性から$g(b) = c$となる$b \in B$があり,
  さらに$f$の全射性から$f(a) = b$となる$a \in A$がある.
  すなわち$g \circ f(a) = c$となる$a \in A$がある.
  (2) $g \circ f(a) = g \circ f(a^\prime)$であるとする.
  このとき$g$の単射性から$f(a) = f(a^\prime)$,
  さらに$f$の単射性から$a = a^\prime$.
  (3) $g \circ f$が全射なので, 任意の$c \in C$に対して
  $g \circ f(a) = c$となる$a \in A$がある.
  すなわち$g(f(a)) = c$となる$f(a) \in B$がある.
  (4) $f(a) = f(a^\prime)$であるとする.
  このとき$g \circ f(a) = g \circ f(a^\prime)$であり,
  $g \circ f$の単射性より$a = a^\prime$.
\end{ans}

\begin{ans}
  (全射$\Rightarrow \forall S\ f(f^{-1}(S)) = S$).
  まず$f(f^{-1}(S)) \subset S$を示す.
  $b \in f(f^{-1}(S))$ならば,
  $f(a) = b,\ a \in f^{-1}(S)$となる$a$が存在する.
  逆像の定義から$f(a) \in S$なので, $b \in S$.
  (注意: ここでは$f$の全射性は用いなかった.)
  次に$f(f^{-1}(S)) \supset S$を示す.
  $b \in S$であるとすると, $f$の全射性から$f(a) = b$となる$a \in A$がある.
  $a \in f^{-1}(S)$なので, $b \in f(f^{-1}(S))$.
  \\
  ($\forall S\ f(f^{-1}(S)) = S \Rightarrow$全射).
  任意の$b \in B$に対して$\{b\} \subset f(f^{-1}(\{b\})) \subset f(A)$.
\end{ans}

\begin{ans}
  前問で示したように, $f(f^{-1}(S)) \subset S$なので,
  $f^{-1}(f(f^{-1}(S))) \subset f^{-1}(S)$.
  逆に$f^{-1}(f(f^{-1}(S))) \supset f^{-1}(S)$は次のようにして示される.
  $a \in f^{-1}(S)$ならば, $f(a) = b,\ b \in S$となる$b$が存在する.
  ふたたび$a \in f^{-1}(S)$より, $b \in f(f^{-1}(S))$.
  さらに$f(a) = b$より, $a \in f^{-1}(f(f^{-1}(S)))$.
\end{ans}

\begin{ans}
  (1) $x = 4.1$
  (2) $A = \mathbb{N},\ B = \emptyset$
  (3) $f: \{0, 1, 2\} \rightarrow \{0, 1\}$を
  $f(0) = f(2) = 0,\ f(1) = 1$と定めて,
  $S_0 = \{0, 1\},\ S_1 = \{1, 2\}$
  とすると, $f(S_0 \cap S_1) = f(\{1\}) = \{1\}$,
  $f(S_0) \cap f(S_1) = \{0, 1\} \cap \{0, 1\} = \{0, 1\}$.
\end{ans}

\begin{ans}
  (1) (d).
  $x = 4$が$A \Rightarrow B$の反例,
  $x = 1$が$B \Rightarrow A$の反例である.
  (2) (a).
  $B \Rightarrow A$は真だが,
  $x = 4$が$A \Rightarrow B$の反例である.
  (3) (b). $A \Rightarrow B$は真だが,
  $X = \mathbb{N},\ Y = \emptyset$が$B \Rightarrow A$の反例である
\end{ans}

\begin{ans}
  (1) $A$が成り立たないか$B$が成り立たない, かつ$C$が成り立たない
  (2) $A$が成り立ち, かつ$B$も$C$も成り立たない
  (3) $A$か$B$の一方のみが成り立つ
  (4) 自然数$n$があって, 任意の実数$x$に対して$x \le 0$または$\frac{1}{n} \le x$が成り立つ
  (5) ある$\varepsilon > 0$について, 任意の$\delta > 0$に対して$x, y \in [0, 1]$で
  $\abs{x - y} < \delta$かつ$\abs{f(x) - f(y)} \ge \varepsilon$
  を満たすものがある.
\end{ans}

\begin{ans}
  (1) $4 + 5 = 9 \ge 3$なので, 関係$R$がある.
  (2) $1 + (-1) = 0 < 3$なので, 関係$R$はない.
\end{ans}

\begin{ans}
  (1) $X = \mathbb{R},\ R = \{(x, y) \mid y = x^2\}$\\
  (2) $X = \mathbb{Z},\ R = \{(x, y) \mid x\text{は}y\text{で割り切れる.}\}$,
  この$R$は順序である.\\
  (3) $X = \mathbb{R}^n\setminus\{\bm{0}\},\ R = \{(\bm{x}, \bm{y}) \mid \exists \alpha \in \mathbb{R}\setminus\{0\},\ \bm{y} = \alpha \bm{x} \}$,
  この$R$は同値関係 (p47) である.
\end{ans}

\fakesubsection{1.2 well-definedと自然な対象}

\begin{ans}
  線形写像$f:V \rightarrow V$に対してトレース$\tr(f)$を
  $f$の行列表示$M = (m_{ij}) \in M_n(\mathbb{R})$を$1$つとって
  $\tr(f) = \tr(M) = \Sigma_{i = 1}^n m_{ii}$と定義したいとする.
  このとき, $\tr(f)$がwell-definedであること,
  すなわち行列表示のとり方によらず$\tr(f)$が定まるかどうかが問題となる.
  これがwell-definedであることは, 行列のトレースについて$\tr(AB) = \tr(BA)$が成り立つことを用いて,
  別の基底での$f$の行列表示$P^{-1}MP$について
  $\tr(P^{-1}MP) = \tr(MPP^{-1}) = \tr(M)$
  であることからしたがう.
\end{ans}

\begin{ans}
  解答例にないものでは, 双対空間$V^\ast$, テンソル積$V \otimes V$など.
\end{ans}

\fakesubsection{1.3 選択公理とツォルンの補題}

\begin{ans}
  (1) $X$の全順序部分集合を任意に取り,
  適当に添え字をつけて$\{(S_\lambda, f_\lambda)\}_{\lambda \in \Lambda}$
  と書くことにする.
  (上界の候補として) $S = \bigcup_{\lambda \in \Lambda} S_\lambda$と定めて,
  写像$f: S \rightarrow B$を,
  任意の$x \in S$に対して$x \in S_\lambda$なる$\lambda$を$1$つとって$f(x) = f_\lambda(x)$
  として定義したい. まずこれがwell-definedであることを見るために,
  別の$\lambda^\prime$に対して$x \in S_\lambda^\prime$であるとする.
  このとき$\{(S_\lambda, f_\lambda)\}_{\lambda \in \Lambda}$が全順序であることから,
  「$S_\lambda \subset S_{\lambda^\prime}$かつ$f_{\lambda^\prime}$は$f_\lambda$の拡張」(またはここで$\lambda$と$\lambda^\prime$を入れ替えたもの)
  が成り立つ. よって$f_\lambda(x) = f_{\lambda^\prime}(x)$なので,
  $f$は$\lambda$のとり方によらず, well-definedである.
  つぎに$f$が単射であることをみる.
  $f(x) = f(y)$とすると, ある$\lambda$, $\lambda^{\prime}$に対して
  $f_\lambda(x) = f_{\lambda^\prime}(y)$である.
  $\{(S_\lambda, f_\lambda)\}_{\lambda \in \Lambda}$が全順序であることから
  一方は他方の拡張であり, $f_\lambda(x) = f_\lambda(y)$または$f_{\lambda^\prime}(x) = f_{\lambda^\prime}(y)$.
  いずれの場合も$f_\lambda, f_{\lambda^\prime}$が単射であることから$x = y$.
  したがって$f$は単射である.
  さらに$(S, f)$が$\{(S_\lambda, f_\lambda)\}_{\lambda \in \Lambda}$の上界であることをみる.
  任意の$\lambda \in \Lambda$に対して
  $S_\lambda \subset S$であり, また$x \in S_\lambda$ならば$f(x) = f_\lambda(x)$が成り立つ.
  よって$(S_\lambda, f_\lambda) \le (S, f)$なので, $(S, f)$は上界である.
  以上により, $X$の任意の全順序部分集合が上界を持つことが示せたから,
  ツォルンの補題が適用でき, $X$に極大元が存在することがわかる.\\
  (2) $S_0 \neq A$かつ$f_0(S_0) \neq B$であると仮定して矛盾を導く.
  $a \in A \setminus S_0$と$b \in B \setminus f_0(S_0)$をとり,
  $f: S_0 \cup \{a\} \rightarrow B$を,
  $S_0$上では$f = f_0$, $f(a) = b$と定めると,
  $(S_0 \cup \{a\}, f) \in X$かつ$(S_0, f_0) < (S_0 \cup \{a\}, f)$となって,
  $(S_0, f_0)$が極大元であることと矛盾する.
  したがって$S_0 = A$または$f_0(S_0) = B$.
\end{ans}

\fakesection{第2章 群の基本}
\fakesubsection{2.1 群の定義}

\begin{ans}
  $1$が単位元だが$0$の逆元が存在しないので群ではない.
  (単位元の存在に加えて結合法則が成り立つのでモノイドである.)
\end{ans}

\begin{ans}
  $0$が単位元なので,
  $a \in \mathbb{R}$の逆元が$b \in \mathbb{R}$であるならば,
  $a + b + ab = 0$より$(1 + a)b = -a$が成り立つはずである.
  ところが$a = -1$のときこの等式はどんな$b$に対しても成り立たないから,
  $-1$の逆元は存在しない.
  したがってこの演算では群にならない.
  (結合法則は成り立つのでモノイドである.)
\end{ans}

\end{document}
