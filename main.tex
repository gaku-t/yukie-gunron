\documentclass{amsart}

\usepackage{amsmath, amssymb}
\usepackage{commath}
\usepackage{bm}
\usepackage{mathtools}
\usepackage{ytableau}

\theoremstyle{definition}
\newtheorem{ans}{}
\numberwithin{ans}{subsection}

\newcommand{\fakesection}[1]{%
  \par\refstepcounter{section}% Increase section counter
  \sectionmark{#1}% Add section mark (header)
  \addcontentsline{toc}{section}{\protect\numberline{\thesection}#1}% Add section to ToC
  % Add more content here, if needed.
}
\newcommand{\fakesubsection}[1]{%
  \par\refstepcounter{subsection}% Increase subsection counter
  \subsectionmark{#1}% Add subsection mark (header)
  \addcontentsline{toc}{subsection}{\protect\numberline{\thesubsection}#1}% Add subsection to ToC
  % Add more content here, if needed.
}

\DeclareMathOperator{\id}{id}
\DeclareMathOperator{\tr}{tr}

\newcommand{\transpose}[1]{{\prescript{t}{}{#1}}}
\newcommand{\Sp}[1]{\mathrm{Sp}(#1)}
\newcommand{\U}[1]{\mathrm{U}(#1)}
\newcommand{\SO}[1]{\mathrm{SO}(#1)}
\newcommand{\Z}[1]{\mathbb{Z}/#1\mathbb{Z}}
\newcommand{\Ad}[1]{\mathrm{Ad}(#1)}

\DeclarePairedDelimiter{\gen}{\langle}{\rangle}
\DeclarePairedDelimiter{\order}{\lvert}{\rvert}


\begin{document}

\noindent(第1版第11刷をもとにしている.)

\fakesection{第1章 集合論}
\fakesubsection{1.1 集合と論理の復習}

\begin{ans}
  $f: g,\ A: X,\ B: X$
\end{ans}

\begin{ans}
  (1) $f(S) = \{3, 4\}$
  (2) $f^{-1}(S_1) = \emptyset,\ f^{-1}(S_2) = \{1, 3, 4, 5\}$
  (3) $f(a) = 2 (\in B)$であるような$a \in A$が存在しないので, 全射ではない.
  (4) $f(1) = f(4)$なので, 単射ではない.
\end{ans}

\begin{ans}
  写像$f: A \rightarrow B$について考える.
  全射: $B$のすべての要素が集合$A$から来ている.\\
  単射: $A$の異なる要素は$B$の異なる要素に行く.
\end{ans}

\begin{ans}
  全射: $f(x) = x,\ x\sin x,\ x^3 - x$
  単射: $f(x) = x,\ e^x,\ \arctan x$
\end{ans}

\begin{ans}
  (逆写像$\Rightarrow$全単射) $f \circ g = \id_B$が全射なので, 次問より, $f$は全射.
  同様に$g \circ f = \id_A$が単射なので, $f$は単射.\\
  (全単射$\Rightarrow$逆写像)
  $b \in B$を任意に取ると, $f$の全射性から$f(a) = b$なる$a \in A$がある.
  しかも$f$の単射性により, このような$a$は一意である.
  そこで$g: B \rightarrow A$を$g(b) = a$と定めれば,
  $f \circ g(b) = f(a) = b$, $g \circ f(a) = g(b) = a$
  が成り立つので, $f \circ g = \id_B$, $g \circ f = \id_A$.
\end{ans}

\begin{ans}
  (1) $c \in C$を任意に取る.
  $g$の全射性から$g(b) = c$となる$b \in B$があり,
  さらに$f$の全射性から$f(a) = b$となる$a \in A$がある.
  すなわち$g \circ f(a) = c$となる$a \in A$がある.
  (2) $g \circ f(a) = g \circ f(a^\prime)$であるとする.
  このとき$g$の単射性から$f(a) = f(a^\prime)$,
  さらに$f$の単射性から$a = a^\prime$.
  (3) $g \circ f$が全射なので, 任意の$c \in C$に対して
  $g \circ f(a) = c$となる$a \in A$がある.
  すなわち$g(f(a)) = c$となる$f(a) \in B$がある.
  (4) $f(a) = f(a^\prime)$であるとする.
  このとき$g \circ f(a) = g \circ f(a^\prime)$であり,
  $g \circ f$の単射性より$a = a^\prime$.
\end{ans}

\begin{ans}
  (全射$\Rightarrow \forall S\ f(f^{-1}(S)) = S$).
  まず$f(f^{-1}(S)) \subset S$を示す.
  $b \in f(f^{-1}(S))$ならば,
  $f(a) = b,\ a \in f^{-1}(S)$となる$a$が存在する.
  逆像の定義から$f(a) \in S$なので, $b \in S$.
  (注意: ここでは$f$の全射性は用いなかった.)
  次に$f(f^{-1}(S)) \supset S$を示す.
  $b \in S$であるとすると, $f$の全射性から$f(a) = b$となる$a \in A$がある.
  $a \in f^{-1}(S)$なので, $b \in f(f^{-1}(S))$.
  \\
  ($\forall S\ f(f^{-1}(S)) = S \Rightarrow$全射).
  任意の$b \in B$に対して$\{b\} \subset f(f^{-1}(\{b\})) \subset f(A)$.
\end{ans}

\begin{ans}
  前問で示したように, $f(f^{-1}(S)) \subset S$なので,
  $f^{-1}(f(f^{-1}(S))) \subset f^{-1}(S)$.
  逆に$f^{-1}(f(f^{-1}(S))) \supset f^{-1}(S)$は次のようにして示される.
  $a \in f^{-1}(S)$ならば, $f(a) = b,\ b \in S$となる$b$が存在する.
  ふたたび$a \in f^{-1}(S)$より, $b \in f(f^{-1}(S))$.
  さらに$f(a) = b$より, $a \in f^{-1}(f(f^{-1}(S)))$.
\end{ans}

\begin{ans}
  (1) $x = 4.1$
  (2) $A = \mathbb{N},\ B = \emptyset$
  (3) $f: \{0, 1, 2\} \rightarrow \{0, 1\}$を
  $f(0) = f(2) = 0,\ f(1) = 1$と定めて,
  $S_0 = \{0, 1\},\ S_1 = \{1, 2\}$
  とすると, $f(S_0 \cap S_1) = f(\{1\}) = \{1\}$,
  $f(S_0) \cap f(S_1) = \{0, 1\} \cap \{0, 1\} = \{0, 1\}$.
\end{ans}

\begin{ans}
  (1) (d).
  $x = 4$が$A \Rightarrow B$の反例,
  $x = 1$が$B \Rightarrow A$の反例である.
  (2) (a).
  $B \Rightarrow A$は真だが,
  $x = 4$が$A \Rightarrow B$の反例である.
  (3) (b). $A \Rightarrow B$は真だが,
  $X = \mathbb{N},\ Y = \emptyset$が$B \Rightarrow A$の反例である
\end{ans}

\begin{ans}
  (1) $A$が成り立たないか$B$が成り立たない, かつ$C$が成り立たない
  (2) $A$が成り立ち, かつ$B$も$C$も成り立たない
  (3) $A$か$B$の一方のみが成り立つ
  (4) 自然数$n$があって, 任意の実数$x$に対して$x \le 0$または$\frac{1}{n} \le x$が成り立つ
  (5) ある$\varepsilon > 0$について, 任意の$\delta > 0$に対して$x, y \in [0, 1]$で
  $\abs{x - y} < \delta$かつ$\abs{f(x) - f(y)} \ge \varepsilon$
  を満たすものがある.
\end{ans}

\begin{ans}
  (1) $4 + 5 = 9 \ge 3$なので, 関係$R$がある.
  (2) $1 + (-1) = 0 < 3$なので, 関係$R$はない.
\end{ans}

\begin{ans}
  (1) $X = \mathbb{R},\ R = \{(x, y) \mid y = x^2\}$\\
  (2) $X = \mathbb{Z},\ R = \{(x, y) \mid x\text{は}y\text{で割り切れる.}\}$,
  この$R$は順序である.\\
  (3) $X = \mathbb{R}^n\setminus\{\bm{0}\},\ R = \{(\bm{x}, \bm{y}) \mid \exists \alpha \in \mathbb{R}\setminus\{0\},\ \bm{y} = \alpha \bm{x} \}$,
  この$R$は同値関係 (p47) である.
\end{ans}

\fakesubsection{1.2 well-definedと自然な対象}

\begin{ans}
  線形写像$f:V \rightarrow V$に対してトレース$\tr(f)$を
  $f$の行列表示$M = (m_{ij}) \in M_n(\mathbb{R})$を$1$つとって
  $\tr(f) = \tr(M) = \sum_{i = 1}^n m_{ii}$と定義したいとする.
  このとき, $\tr(f)$がwell-definedであること,
  すなわち行列表示のとり方によらず$\tr(f)$が定まるかどうかが問題となる.
  これがwell-definedであることは, 行列のトレースについて$\tr(AB) = \tr(BA)$が成り立つことを用いて,
  別の基底での$f$の行列表示$P^{-1}MP$について
  $\tr(P^{-1}MP) = \tr(MPP^{-1}) = \tr(M)$
  であることからしたがう.
\end{ans}

\begin{ans}
  解答例にないものでは, 双対空間$V^\ast$, テンソル積$V \otimes V$など.
\end{ans}

\fakesubsection{1.3 選択公理とツォルンの補題}

\begin{ans}
  (1) $X$の全順序部分集合を任意に取り,
  適当に添え字をつけて$\{(S_\lambda, f_\lambda)\}_{\lambda \in \Lambda}$
  と書くことにする.
  (上界の候補として) $S = \bigcup_{\lambda \in \Lambda} S_\lambda$と定めて,
  写像$f: S \rightarrow B$を,
  任意の$x \in S$に対して$x \in S_\lambda$なる$\lambda$を$1$つとって$f(x) = f_\lambda(x)$
  として定義したい. まずこれがwell-definedであることを見るために,
  別の$\lambda^\prime$に対して$x \in S_\lambda^\prime$であるとする.
  このとき$\{(S_\lambda, f_\lambda)\}_{\lambda \in \Lambda}$が全順序であることから,
  「$S_\lambda \subset S_{\lambda^\prime}$かつ$f_{\lambda^\prime}$は$f_\lambda$の拡張」(またはここで$\lambda$と$\lambda^\prime$を入れ替えたもの)
  が成り立つ. よって$f_\lambda(x) = f_{\lambda^\prime}(x)$なので,
  $f$は$\lambda$のとり方によらず, well-definedである.
  つぎに$f$が単射であることをみる.
  $f(x) = f(y)$とすると, ある$\lambda$, $\lambda^{\prime}$に対して
  $f_\lambda(x) = f_{\lambda^\prime}(y)$である.
  $\{(S_\lambda, f_\lambda)\}_{\lambda \in \Lambda}$が全順序であることから
  一方は他方の拡張であり, $f_\lambda(x) = f_\lambda(y)$または$f_{\lambda^\prime}(x) = f_{\lambda^\prime}(y)$.
  いずれの場合も$f_\lambda, f_{\lambda^\prime}$が単射であることから$x = y$.
  したがって$f$は単射である.
  さらに$(S, f)$が$\{(S_\lambda, f_\lambda)\}_{\lambda \in \Lambda}$の上界であることをみる.
  任意の$\lambda \in \Lambda$に対して
  $S_\lambda \subset S$であり, また$x \in S_\lambda$ならば$f(x) = f_\lambda(x)$が成り立つ.
  よって$(S_\lambda, f_\lambda) \le (S, f)$なので, $(S, f)$は上界である.
  以上により, $X$の任意の全順序部分集合が上界を持つことが示せたから,
  ツォルンの補題が適用でき, $X$に極大元が存在することがわかる.\\
  (2) $S_0 \neq A$かつ$f_0(S_0) \neq B$であると仮定して矛盾を導く.
  $a \in A \setminus S_0$と$b \in B \setminus f_0(S_0)$をとり,
  $f: S_0 \cup \{a\} \rightarrow B$を,
  $S_0$上では$f = f_0$, $f(a) = b$と定めると,
  $(S_0 \cup \{a\}, f) \in X$かつ$(S_0, f_0) \leq (S_0 \cup \{a\}, f)$かつ$(S_0, f_0) \neq (S_0 \cup \{a\}, f)$となって,
  $(S_0, f_0)$が極大元であることと矛盾する.
  したがって$S_0 = A$または$f_0(S_0) = B$.
\end{ans}

\fakesection{第2章 群の基本}
\fakesubsection{2.1 群の定義}

\begin{ans}
  $1$が単位元だが$0$の逆元が存在しないので群ではない.
  (単位元の存在に加えて結合法則が成り立つのでモノイドである.)
\end{ans}

\begin{ans}
  $0$が単位元なので,
  $a \in \mathbb{R}$の逆元が$b \in \mathbb{R}$であるならば,
  $a + b + ab = 0$より$(1 + a)b = -a$が成り立つはずである.
  ところが$a = -1$のときこの等式はどんな$b$に対しても成り立たないから,
  $-1$の逆元は存在しない.
  したがってこの演算では群にならない.
  (結合法則は成り立つのでモノイドである.)
\end{ans}

\begin{ans}
  略.
\end{ans}

\begin{ans}
  $((ab)c)d = (a(bc))d = a((bc)d)$.
\end{ans}

\begin{ans}
  両辺に左から$a^{-1}b^{-1}$, 右から$d^{-1}$を掛けると
  $c^{-1} = a^{-1}b^{-1}ab$.
  この逆元をとって$c = b^{-1}a^{-1}ba$.
\end{ans}

\begin{ans}
  (1) $\begin{pmatrix}
    4 & 1 & 2 & 3 \\
    1 & 2 & 3 & 4
  \end{pmatrix} =
  \begin{pmatrix}
    1 & 2 & 3 & 4 \\
    2 & 3 & 4 & 1
  \end{pmatrix} = (1\ 2\ 3\ 4)$ \\
  (2) $(2\ 4)^{-1}(1\ 3)^{-1} = (2\ 4)(1\ 3)$ \\
  (3) $\begin{pmatrix}
    1 & 3 & 4 & 2 \\
    4 & 2 & 3 & 1
  \end{pmatrix}\begin{pmatrix}
    1 & 2 & 3 & 4 \\
    1 & 3 & 4 & 2
  \end{pmatrix} = \begin{pmatrix}
    1 & 2 & 3 & 4 \\
    4 & 2 & 3 & 1
  \end{pmatrix} = (1\ 4)$ \\
  (4) $(2\ 4)(1\ 3)(1\ 3) = (2\ 4)$ \\
  (5) $\begin{pmatrix}
    1 & 2 & 3 & 4 \\
    1 & 3 & 4 & 2
  \end{pmatrix}\begin{pmatrix}
    1 & 2 & 3 & 4 \\
    4 & 1 & 2 & 3
  \end{pmatrix}\begin{pmatrix}
    1 & 3 & 4 & 2 \\
    1 & 2 & 3 & 4
  \end{pmatrix} = \begin{pmatrix}
    4 & 3 & 1 & 2 \\
    2 & 4 & 1 & 3
  \end{pmatrix}\begin{pmatrix}
    1 & 4 & 2 & 3 \\
    4 & 3 & 1 & 2
  \end{pmatrix}\\
  \begin{pmatrix}
    1 & 2 & 3 & 4 \\
    1 & 4 & 2 & 3
  \end{pmatrix} = \begin{pmatrix}
    1 & 2 & 3 & 4 \\
    2 & 4 & 1 & 3
  \end{pmatrix} = (1\ 2\ 4\ 3)$\\
  (6) $(2\ 4)(1\ 3)(1\ 3)(1\ 3)(2\ 4) = (2\ 4)(1\ 3)(2\ 4) = (1\ 3)(2\ 4)(2\ 4) = (1\ 3)$
\end{ans}

\fakesubsection{2.2 環・体の定義}

\begin{ans}
  (1) $9 \equiv 2 \pmod{7}$より, $\overline{2}$.
  (2) $-3 \equiv 4 \pmod{7}$より, $\overline{4}$.
  (3) $20 \equiv 6 \pmod{7}$より, $\overline{6}$.
  (4) $125 \equiv 6 \pmod{7}$より, $\overline{6}$.
  (5) $4^{32} \equiv 4^{30} \cdot 4^2 \equiv 64^{10} \cdot 16 \equiv 1^{10} \cdot 16 \equiv 2 \pmod{7}$より, $\overline{2}$.
\end{ans}

\begin{ans}
  (1) $34 \cdot 21 \equiv (-5) \cdot 21 \equiv -105 \equiv 12 \pmod{39}$,
  $12 \cdot 33 \equiv 12 \cdot (-6) \equiv -72 \equiv 6 \pmod{39}$より, $\overline{6}$
  (2) $18 \cdot 13 = 6 \cdot 39 \equiv 0 \pmod{39}$より, $\overline{0}$.
  (3) $16^2 \equiv 256 \equiv 22 \pmod{39}$, $16^4 = 22^2 \equiv 16 \pmod{39}$,
  $16^8 \equiv 16^2 \equiv 22 \pmod{39}$より, $\overline{22}$.
  (4) (3) に続けて$16^{16} \equiv 22^2 \equiv 16 \pmod{39}$, $16^{32} \equiv 16^2 \equiv 22 \pmod{39}$より,
  $16^{34} = 16^2 \cdot 16^{32} \equiv 22 \cdot 22 \equiv 16 \pmod{39}$. よって$\overline{16}$.
\end{ans}

\fakesubsection{2.3 部分群と生成元}

\begin{ans}
  命題2.3.2の条件と同値であることを確かめればよい.
  $H$が部分群であるとすると, 条件 (3) より$x^{-1} \in H$.
  さらに条件 (2) より$x^{-1}y \in H$.
  逆に, 群$G$の空でない部分集合$H$について
  「$x, y \in H \Rightarrow x^{-1}y \in H$」
  が成り立っているとする. $H$は空でないから, ある元$x_0 \in H$が存在する.
  よって, $1_G = x_0^{-1}x_0 \in H$. すなわち条件 (1) が成り立つ.
  $x \in H$ならば, $x, 1_G \in H$について$x^{-1} = x^{-1}1_G \in H$. すなわち条件 (2) が成り立つ.
  $x, y \in H$ならば, $x^{-1} \in H$なので$xy = (x^{-1})^{-1}y \in H$. すなわち条件 (3) が成り立つ.
\end{ans}

\begin{ans}
  $\transpose{I_{2n}}J_nI_{2n} = J_n$より, $I_{2n} \in \Sp{2n}$.
  $h_1, h_2 \in \Sp{2n}$ならば,
  $\transpose{(h_1h_2)}J_n(h_1h_2) = \transpose{h_2}(\transpose{h_1}J_nh_1)h_2 = \transpose{h_2}J_nh_2 = J_n$より
  $h_1h_2 \in \Sp{2n}$.
  また, $h \in \Sp{2n}$ならば,
  $\transpose{(h^{-1})}J_nh^{-1} = \transpose{(h^{-1})}(\transpose{h}J_nh)h^{-1} = J_n$.
  よって$h^{-1} \in \Sp{2n}$.
\end{ans}

\begin{ans}
  $\overline{AB} = \overline{A}\,\overline{B}$であることに注意する.
  (このことから, $\overline{g}^{-1} = \overline{g^{-1}}$であることも分かる.)
  $\transpose{\overline{I_n}I_n} = I_nI_n = I_n$より$I_n \in \U{n}$.
  $h_1, h_2 \in \U{n}$ならば,
  $\transpose{\overline{h_1h_2}}h_1h_2
  = \transpose{\overline{h_2}}\,\transpose{\overline{h_1}}h_1h_2
  = \transpose{\overline{h_2}}I_nh_2 = I_n$より$h_1h_2 \in \U{n}$.
  また, $h \in \U{n}$ならば,
  $\transpose{\overline{h^{-1}}}h^{-1}
  = \transpose{\overline{h}^{-1}}(\transpose{\overline{h}}h)h^{-1}
  = I_nI_n = I_n$.
  よって$h^{-1} \in \U{n}$.
\end{ans}

\begin{ans}
  (1) 明らかに$I_n \in B$.
  $b = (b_{ij}),\ b^\prime = (b^\prime_{ij})$をともにBの元とすると,
  $bb^\prime$の$(i, j)$成分は$\sum_{k = 1}^nb_{ik}b^\prime_{kj}$である.
  もし$i < j$ならば, どんな$k$に対しても$i < k$または$k < j$が成り立つので,
  各項が$0$となり$bb^\prime$の$(i, j)$成分は$0$である. すなわち$bb^\prime \in B$.
  $b \in B$とすると, $b$に
  (i) 行を$\lambda (\neq 0)$倍する
  (ii) 第$i$行に第$j$行の$\lambda$倍を足す ($i > j$),
  という$2$つの操作 (基本変形の一部) を繰り返すことにより, 単位行列にできる.
  これらの操作は$b$に$B$のある元を左から掛けることに対応しており, それらの積が$b^{-1}$に他ならない.
  $B$が積で閉じていることはすでに示していたから, $b^{-1} \in B$.\\
  (2) $n = 1$のとき$B = G = \mathbb{R}^\times$は可換群である.
  $n \ge 2$のとき, $B$は可換群ではない. たとえば$n = 2$では
  $\begin{pmatrix}
    1 & 0 \\
    1 & 1
  \end{pmatrix}\begin{pmatrix}
    1 & 0 \\
    0 & 2
  \end{pmatrix} \neq \begin{pmatrix}
    1 & 0 \\
    0 & 2
  \end{pmatrix}\begin{pmatrix}
    1 & 0 \\
    1 & 1
  \end{pmatrix}$.
  一般の$n \ge 2$の場合にも, たとえば単位行列の定数倍でない対角行列と,
  左下の成分がすべて$1$である行列との積は非可換である.
\end{ans}

\begin{ans}
  $1 \in \mathbb{R}_>$,
  また$x, y \in \mathbb{R}_>$ならば$xy \in \mathbb{R}_>$,
  また$x \in \mathbb{R}_>$ならば$x^{-1} = \frac{1}{x} \in {R}_>$.
\end{ans}

\begin{ans}
  $0 \notin \mathbb{R}_>$なので, 部分群ではない.
\end{ans}

\begin{ans}
  $H$の元$h$は, 絶対値が$1$なので$h = e^{i\theta}$と書ける.
  さらに, $h^n = 1$より$\theta = \frac{2\pi m}{n}\ (m \in \mathbb{Z})$である.
  $e^\frac{2\pi im}{n} = e^\frac{2\pi i m^\prime}{n}$が成り立つのは,
  $m^\prime = m + kn\ (k \in \mathbb{Z})$と書けるとき, またそのときに限るので,
  $H = \{1 = e^0, e^\frac{2\pi i}{n}, e^\frac{2\pi i \cdot 2}{n},..., e^\frac{2\pi i(n - 1)}{n}\}$.
  よって$H$の位数は$n$.
  また, $\gen{e^\frac{2\pi i}{n}} = H$なので, $H$は巡回群.
\end{ans}

\begin{ans}
  (1) $\mathfrak{S}_3$は可換群ではないから巡回群ではない.
  (別解: $\mathfrak{S}_3$には位数$2$の元が$3$つあるが, 巡回群$\Z{6}$には$1$つしかないから, 両者は一致しない.) \\
  (2) $\mathbb{Q}$が巡回群であると仮定して, 生成元が$r \neq 0$であるとする.
  このとき, $\mathbb{Q}$の任意の元は$nr\ (n \in \mathbb{Z})$という形で表せるはずだが, $\frac{r}{2} \in \mathbb{Q}$はそうではないので矛盾. \\
  (3) $\mathbb{Q}$の場合と同様. (別解: 巡回群の濃度はたかだか可算だが, $\mathbb{R}$の濃度は非可算なので一致しない.) \\
  (4) $\mathbb{Q}^\times$が巡回群であると仮定して, 生成元が$r$であるとする.
  明らかに$r \neq \pm 1$である. $-r \in \mathbb{Q}^\times$なので,
  $r^n = -r$, すなわち$r^{n-1} = -1$となるような$n \in \mathbb{Z}$が存在するはずである.
  ところがこの式を満たすような$r$は$-1$しかないので矛盾.
  \\
  (5) $\mathbb{Z} \times \mathbb{Z}$が巡回群であると仮定して, 生成元が$(x, y)$であるとする.
  $(nx, ny)\ (n \in \mathbb{Z})$が$\mathbb{Z} \times \mathbb{Z}$のすべての元を渉っている必要があるので,
  $x, y$それぞれが$\mathbb{Z}$の生成元でなれけばならない. よって少なくとも$x \neq 0$である.
  一方, $(x, y + 1) \in \mathbb{Z} \times \mathbb{Z}$を考えると,
  $(x, y + 1) = (nx, ny)$を満たす$n \in \mathbb{Z}$が存在するはずだが,
  $x = nx$かつ$x \neq 0$より$n = 1$. 一方第$2$成分は$y + 1 \neq y$となり矛盾.
\end{ans}

\begin{ans}
  (1) 数学的帰納法により示す. $n = 1$のときは明らか.
  $n = k$のときに正しいとして, $n = k + 1$の場合を示す.
  任意の$\sigma \in \mathfrak{S}_{k + 1}$をとり.
  $\sigma(1) = i$であるとする.
  $\tau = (1\ 2)(2\ 3)\cdots(i-1\ i)\sigma$とおけば$\tau(1) = 1$であるから,
  $n = k$の場合に帰着できる. よって$n = k + 1$の場合にも正しい. \\
  (2) $(1\ 2\ \cdots\ n)^{-(i-1)}(1\ 2)(1\ 2\ \cdots\ n)^{i-1} = (i\ i + 1)$なので,
  (1) よりこれらも$\mathfrak{S}_n$を生成する.
\end{ans}

\fakesubsection{2.4 元の位数}

\begin{ans}
  (1) 12, 144 (2) yes
\end{ans}

\begin{ans}
  (1)
  $395 = 1 \cdot 265 + 130$,
  $265 = 2 \cdot 130 + 5$,
  $130 = 26 \cdot 5$.
  よって$d = 5$.\\
  (2)
  $5
  = 265 - 2 \cdot 130
  = 265 - 2 \cdot (395 - 1 \cdot 265)
  = -2 \cdot 395 + 3 \cdot 265$.
  よって$(x, y) = (-2, 3)$が$1$つの解である.
\end{ans}

\begin{ans}
  (1) 位数が少ないので単純に総当りで調べる.
  $\overline{2}^{-1} = \overline{4}$,
  $\overline{3}^{-1} = \overline{5}$,
  $\overline{4}^{-1} = \overline{2}$,
  $\overline{5}^{-1} = \overline{3}$,
  $\overline{6}^{-1} = \overline{6}$.\\
  (2) $284x + 3y = 1$の$1$つの解をユークリッドの互除法によって求めると,
  $(x, y) = (-1, 95)$となる. よって$\overline{3}^{-1} = \overline{95}$.
\end{ans}

\begin{ans}
  $1,..., p^n$のうち$p^n$と互いに素でないのは$p$の倍数で, $p^{n-1}$個ある.
  よって位数は$p^n - p^{n-1}$.
\end{ans}

\begin{ans}
  $x^{35d} = 1$であることと$35d$が$60$の倍数であることは同値.
  さらにこのことは$7d$が$12$の倍数であることと同値.
  $7$と$12$は互いに素なので, これは$d$が$12$の倍数であることと同値である.
  よって位数は$12$.
\end{ans}

\begin{ans}
  $x^{nm} = 1$であることと$nm$が$d$の倍数であることは同値.
  さらにこのことは$nm/\gcd(n, d)$が$d/\gcd(n, d)$の倍数であることと同値.
  $n/\gcd(n, d)$と$d/\gcd(n, d)$は互いに素なので, これは$m$が$d/\gcd(n, d)$の倍数であることと同値である.
  よって位数は$d/\gcd(n, d)$.
\end{ans}

\begin{ans}
  位数$d$の群$G$を生成する元の位数は$d$である.
  前問より$\overline{n}$の位数は$d/\gcd(d, n)$であるから,
  $d$と互いに素であるような$n$についての$\overline{n}$を挙げればよい.
  たとえば$\Z{15}$では
  $\overline{1}, \overline{2}, \overline{4}, \overline{7}, \overline{8}, \overline{11}, \overline{13}, \overline{14}$.
  他も同様.
\end{ans}

\begin{ans}
  任意の$g, h \in G$に対し,
  $gh = (gh)^{-1} = h^{-1}g^{-1} = hg$.
\end{ans}

\begin{ans}
  (1) $g^2 = \begin{pmatrix}
    -1 & 0 \\
    0 & -1
  \end{pmatrix},\
  g^4 = \begin{pmatrix}
    1 & 0 \\
    0 & 1
  \end{pmatrix}$.
  よって$g$の位数は$4$の約数であり, かつ$1$, $2$ではないから, 位数は$4$.
  また$h$については,
  $h^2 = \begin{pmatrix}
    0 & 1 \\
    -1 & -1
  \end{pmatrix},\
  h^3 = \begin{pmatrix}
    -1 & 0 \\
    0 & -1
  \end{pmatrix},\
  h^6 = \begin{pmatrix}
    1 & 0 \\
    0 & 1
  \end{pmatrix}$.
  よって位数は$6$の約数であり, かつ$1$, $2$, $3$ではないから, 位数は$6$.\\
  (2) $gh = \begin{pmatrix}
    1 & 0 \\
    1 & 1
  \end{pmatrix} = \begin{pmatrix}
    1 & 0 \\
    0 & 1
  \end{pmatrix} + \begin{pmatrix}
    0 & 0 \\
    1 & 0
  \end{pmatrix}$
  と分解できるが, $\begin{pmatrix}
    0 & 0 \\
    1 & 0
  \end{pmatrix}^2 = 0$であることに注意すると, 任意の正の整数$n$に対して
  $(gh)^n = \begin{pmatrix}
    1 & 0 \\
    0 & 1
  \end{pmatrix} + n \cdot \begin{pmatrix}
    0 & 0 \\
    1 & 0
  \end{pmatrix} \neq \begin{pmatrix}
    1 & 0 \\
    0 & 1
  \end{pmatrix}$.
\end{ans}

\begin{ans}
  (1) $a^n = 1, b^m = 1$ならば, $(ab)^{nm} = a^{nm}b^{nm} = 1$.
  (2) $1$の位数は$0$なので$1 \in H$.
  また$x$の位数は$x^{-1}$の位数に等しいので, $x \in H$ならば$x^{-1} \in H$.
\end{ans}

\fakesubsection{2.5 準同型と同型}

\begin{ans}
  (1) 「…」が成り立つとする.
  $i_1 = m, i_2 = 0$とすると, $x^{i_1} = x^{i_2} = 1$.
  よって$y^{i_1} = y^{i_2}$, すなわち$y^m = 1$.
  命題2.4.18より, $m$は$n$の倍数である.
  逆に, $m$が$n$の倍数であるとする.
  $x^{i_1} = x^{i_2}$であるような任意の$i_1, i_2 \in \mathbb{Z}$に対して,
  $i_1 - i_2$は命題2.4.18より$m$の倍数であり, 仮定からこれは$n$の倍数でもある.
  よって$y^{i_1 - i_2} = 1$, すなわち$y^{i_1} = y^{i_2}$であるから, 「…」が成り立つ.
  以上により, 「…」が成り立つための必要十分条件は, $m$が$n$の倍数であることである.\\
  (2) $m$, $n$が (1) の性質を満たすことから, $\phi(x^i) = y^i$と定めると$\phi$はwell-definedである.
  $\phi(x^ix^j) = \phi(x^{i+j}) = y^{i+j} = y^iy^j = \phi(x^i)\phi(x^j)$
  より, $\phi$は準同型.
\end{ans}

\begin{ans}
  $\phi_n(gh) = (gh)^n = g^nh^n = \phi_n(gh)$.
\end{ans}

\begin{ans}
  (1) $g$の位数を$n$とすると,$\phi(g)^n = \phi(g^n) = \phi(1_G) = 1_H$.
  よって$\phi(g)$の位数は$n$の約数である.
  (2) $\phi(g)$の位数を$m$とすると, $\phi(g^m) = \phi(g)^m = 1_H$.
  $\phi$が単射なので, $g^m = 1_G$. よって$g$の位数は$m$の約数である.
  (1) と合わせて, $g$の位数と$\phi(g)$の位数は等しい.
\end{ans}

\begin{ans}
  $\Z{4}$には位数$4$の元があるが,
  $\Z{2} \times \Z{2}$にはないので,
  これらは同型ではない.
\end{ans}

\begin{ans}
  $n \ge 0$の場合を数学的帰納法により示す.
  $n = 0$の場合は明らかに真である.
  $n = k$で真であるとすると,
  $(xyx^{-1})^{k+1} = (xyx^{-1})^k(xyx^{-1}) = (xy^kx^{-1})(xyx^{-1}) = xy^{k+1}x^{-1}$より,
  $n = k + 1$についても真である.
  $n < 0$の場合は, $(xyx^{-1})^n = ((xyx^{-1})^{-n})^{-1} = (xy^{-n}x^{-1})^{-1} = xy^nx^{-1}$.
\end{ans}

\begin{ans}
  $\begin{pmatrix}
    a & b \\
    c & d
  \end{pmatrix}$を$\mathrm{GL}_2(\mathbb{R})$または$\mathrm{GL}_2(\mathbb{C})$の元として,
  \[
    \begin{pmatrix}
      1 & 1 \\
      0 & 1
    \end{pmatrix} = \begin{pmatrix}
      a & b \\
      c & d
    \end{pmatrix} \begin{pmatrix}
      1 & 0 \\
      1 & 1
    \end{pmatrix} \cdot \frac{1}{ad - bc} \begin{pmatrix}
      d & -b \\
      -c & a
    \end{pmatrix} = \frac{1}{ad - bc} \begin{pmatrix}
      a + b & b \\
      c + d & d
    \end{pmatrix} \begin{pmatrix}
      d & -b \\
      -c & a
    \end{pmatrix}
  \]
  であるとする. $(2, 1)$成分を比較すると$0 = d^2$であるから, $d = 0$.
  よって
  \[
    \begin{pmatrix}
      1 & 1 \\
      0 & 1
    \end{pmatrix} = - \frac{1}{bc} \begin{pmatrix}
      a + b & b \\
      c & 0
    \end{pmatrix} \begin{pmatrix}
      0 & -b \\
      -c & a
    \end{pmatrix} = - \frac{1}{bc} \begin{pmatrix}
      -bc & -b^2 \\
      0 & -bc
    \end{pmatrix} = \begin{pmatrix}
      1 & \frac{b}{c} \\
      0 & 1
    \end{pmatrix}
  \]
  $(1, 2)$成分を比較して$b = c$である.
  逆に, $d = 0$, $b = c$, $ad - bc \neq 0$となるように$a, b, c, d$を定めれば,
  \[
    \begin{pmatrix}
      1 & 1 \\
      0 & 1
    \end{pmatrix} = \begin{pmatrix}
      a & b \\
      c & d
    \end{pmatrix} \begin{pmatrix}
      1 & 0 \\
      1 & 1
    \end{pmatrix} \begin{pmatrix}
      a & b \\
      c & d
    \end{pmatrix}^{-1}
  \]
  が成り立つことが確かめられる.
  したがって$A$, $B$は$\mathrm{GL}_2(\mathbb{R})$では共役である.
  $d = 0$, $b = c$, $ad - bc \neq 0$という条件のもとで, もし$b \in \mathbb{R}$ならば$
  \begin{vmatrix}
    a & b \\
    c & d
  \end{vmatrix} = - b^2 < 0
  $なので, $A$, $B$は$\mathrm{SL}_2(\mathbb{R})$では共役でない.
  一方$b = i$とすれば$
  \begin{vmatrix}
    a & b \\
    c & d
  \end{vmatrix} = 1
  $なので, $\mathrm{SL}_2(\mathbb{C})$では共役である.
\end{ans}

\begin{ans}
  $G = \Z{15}$の場合のみを考える. (他も同様である.)
  自己準同型$\phi: G \rightarrow G$は,
  $1$つの生成元での値$\phi(\overline{1})$を与えれば一意に定まる.
  そうして定めた$\phi$が同型であるための必要十分条件は, $\phi(\overline{1})$がまた$G$の生成元であることである.
  $G$の生成元は演習問題2.4.7で求めたとおり,
  $\overline{1}, \overline{2}, \overline{4}, \overline{7}, \overline{8}, \overline{11}, \overline{13}, \overline{14}$
  の$8$通りなので, $\mathrm{Aut}(G)$の位数は$8$.
  また, $\overline{1} \mapsto \overline{k}$なる自己同型は$x \mapsto \overline{k}x$と書けるので,
  明らかに$\mathrm{Aut}(G)$はアーベル群である.
  有限アーベル群の基本定理 (定理4.8.1) より,
  $\mathrm{Aut}(G) \cong \Z{8}$
  または
  $\mathrm{Aut}(G) \cong \Z{2} \times \Z{4}$
  または
  $\mathrm{Aut}(G) \cong \Z{2} \times \Z{2} \times \Z{2}$
  であるが, $\mathrm{Aut}(G)$に単一の生成元がないこと,
  および$\overline{1} \mapsto \overline{2}$なる自己同型の位数が$4$であることが計算により確かめられるので,
  $\mathrm{Aut}(G) \cong \Z{2} \times \Z{4}$.
\end{ans}

\begin{ans}
  (1) $b(ab)b^{-1} = ba$.
  (2) $(ab)^n = 1$ならば$(ba)^n = (b(ab)b^{-1})^n = b(ab)^nb^{-1} = 1$.
  同様に$(ba)^n = 1$ならば$(ab)^n = 1$なので, $ab$と$ba$の位数は等しい.
\end{ans}

\begin{ans}
  $G$は互換$(1\ 2)$と$(2\ 3)$で生成されるので,
  $\mathrm{Aut}(G)$の元は$(1\ 2), (2\ 3)$の行き先を定めることで決定できる.
  これらは位数$2$の元なので, 同型写像による行き先は同じく位数$2$の異なる元でなければならず, 候補は$3$つある.
  したがって, $\mathrm{Aut}(G)$の位数はたかだか${}_3 \mathrm{P}_2 = 6$である.
  $G$の位数も$6$なので, $\phi$が単射 ($\mathrm{Ker}(\phi) = \{1\}$) であることを確かめれば, $\phi$が同型であることが分かる.
  $\sigma \in \mathrm{Ker}(\phi)$ならば, 任意の$\tau \in G$に対して
  $\sigma \tau \sigma^{-1} = \tau$, すなわち$\sigma$と$\tau$は可換である.
  互換$(1\ 2)$と可換な元は$1$と$(1\ 2)$のみであることが計算で確かめられ,
  また$(1\ 3)$についても同様なので, $G$のすべての元と可換であるのは$1$のみである.
  よって$\sigma = 1$.
\end{ans}

\fakesubsection{2.6 同値関係と剰余類}

\begin{ans}
  同値関係ではない.
  $\{(x, x) \mid x \in \mathbb{R}\} \subset R$なので, 反射律は満たされている.
  また
  $(a, b) \in \{(x, x) \mid x \in \mathbb{R}\}$ならば$(b, a) \in \{(x, x) \mid x \in \mathbb{R}\}$,
  $(a, b) \in \{(x, 2x) \mid x \in \mathbb{R}\}$ならば$(b, a) \in \{(2x, x) \mid x \in \mathbb{R}\}$,
  $(a, b) \in \{(2x, x) \mid x \in \mathbb{R}\}$ならば$(b, a) \in \{(x, 2x) \mid x \in \mathbb{R}\}$
  なので, 対称律も満たされている.
  しかし, 推移律は満たされていない.
  たとえば$(1, 2), (2, 4) \in R$であるが$(1, 4) \notin R$.
\end{ans}

\begin{ans}
  (反射律) $a = 1a1^{-1}$.
  (対称律) $a = gbg^{-1}$ならば$b = g^{-1}a(g^{-1})^{-1}$.
  (推移律) $a = gbg^{-1}$, $b = hch^{-1}$ならば$a = g(hch^{-1})g^{-1} = (gh)c(gh)^{-1}$.
\end{ans}

\begin{ans}
  系2.6.21よりしたがう.
\end{ans}

\begin{ans}
  $H \cap K$は$H$, $K$の部分群なので$\abs{H \cap K}$は$\abs{H}$と$\abs{K}$の公約数である.
  よって$\abs{H \cap K} = 1$なので$H \cap K = \{1_G\}$.
\end{ans}

\fakesubsection{2.7 両側剰余類}

\begin{ans}
  $\sigma \in G$とする. もし$\sigma(4) = 4$なら, $\sigma \in H$より
  $\sigma \in H1_GH$である.
  もし$\sigma(4) = i (i \neq 4)$なら, $\sigma (i\ 3)(3\ 4) \in H$すなわち$\sigma \in H(3\ 4)(i\ 3)$より
  $\sigma \in H(3\ 4)H$である.
  以上により, 任意の$\sigma \in G$に対して$\sigma \in H1_GH$または$\sigma \in H(3\ 4)H$が成り立つ.
  $(3\ 4) \notin H1_GH$より, $\{1_G, (3\ 4)\}$が$1$つの完全代表系である.
\end{ans}

\begin{ans}
  (1) $g$の第$i_n$行の$c\ (\in \mathbb{R})$倍を第$j\ (> i_n)$行に足すことは, $g$に$B$のある元を左から掛けることに等しい.
  また, $g$の第$n$列の$c\ (\in \mathbb{R})$倍を第$j\ (< n)$列に足すことは, $g$に$B$のある元を右から掛けることに等しい.
  したがって, $g$に対してこれらの操作 (行列の基本変形) を適当に繰り返すことによって条件を満たすような$h$に変形することを考えれば,
  ある$b_1, b_2 \in B$に対して$h = b_1gb_2$が成り立つ.\\
  (2) (1) において$n$について行った操作を$n-1, n-2,..., 1$について順に行い, さらに適当に対角行列を掛けると,
  $h = b_1gb_2$で, $h$の各列で$1$箇所だけが$1$で残りの成分はすべて$0$であるようなものがとれる.
  $h \in \mathrm{GL}_n(\mathbb{R})$なので, 各列で成分が$1$である行はすべて異なる.
  すなわち$h$は置換行列である.\\
  (3) $P_\sigma$の$(i, j)$成分が$\delta_{i\sigma(j)}$であることと,
  $j \neq n$ならば$b_{2, jn} = 0$であることから,
  \begin{align*}
    (b_1 P_\sigma b_2)_{in} &= \sum_{j, k}b_{ij}\delta_{j\sigma(k)}b_{2, kn} \\
    &= \sum_kb_{i\sigma(k)}b_{2, kn} \\
    &= b_{i\sigma(n)}b_{2, nn} \\
  \end{align*}
  ここで$i = \sigma(n)$とすると, (右辺) $\neq 0$であるが,
  これが$P_\tau$の$(\sigma(n), n)$成分 ($= \delta_{\sigma(n)\tau(n)}$) に等しいので,
  $\sigma(n) = \tau(n)$である.\\
  (4) (前半) (3) の式で$i \neq \sigma(n)$の場合を考えると,
  (右辺) $= 0$かつ$b_{2, nn} \neq 0$であるから$b_{1, i\sigma(n)} = 0$.\\
  (後半) 明らかに$P_\nu b_1 P_\nu^{-1}$は正則なので,
  その$(i, j)$成分が$i < j$のとき$0$であることを示せばよい.
  \begin{align*}
    (P_\nu b_1 P_\nu^{-1})_{ij} &= \sum_{k, l}P_{\nu, ik}b_{1, kl}(P_\nu^{-1})_{lj} \\
    &= \sum_{k, l}\delta_{i\nu(k)}b_{1, kl}\delta_{\nu(l)j} \\
    &= b_{1, \nu^{-1}(i)\nu^{-1}(j)}
  \end{align*}
  $\nu$の定め方から, $i < j$かつ$\nu^{-1}(i) > \nu^{-1}(j)$が成り立つのは$j = n$のときだけであり,
  このとき$\nu^{-1}(j) = \nu^{-1}(n) = \sigma(n)$.
  よって(4)より$j = n$のときも上式右辺は$0$である.\\
  (5) (3) の式で, $n$を$n-1$に置き換えたものを考えると,
  \begin{align*}
    (b_1 P_\sigma b_2)_{i(n-1)} &= \sum_{j, k}b_{1, ij}\delta_{j\sigma(k)}b_{2, k(n-1)} \\
    &= \sum_kb_{1, i\sigma(k)}b_{2, k(n-1)} \\
    &= b_{1, i\sigma(n-1)}b_{2, (n-1)(n-1)} + b_{1, i\sigma(n)}b_{2, n(n-1)} \\
  \end{align*}
  まず$i = \sigma(n-1)$とすると, (4) の前半より右辺第$2$項は$0$である.
  第$1$項$\neq 0$であるから右辺は全体として$\neq 0$であり,
  これが$P_\tau$の$(\sigma(n-1), n-1)$成分なので,
  (3)と同様に$\sigma(n-1) = \tau(n-1)$である.
  また, $i \neq \sigma(n-1), \sigma(n)$ならば, (4) の前半と同様にして$b_{1, i\sigma(n-1)} = 0$.
  以下同様に (3) の式で$n$を$n-2, n-3,..., 1$に置き換えたものを順に考えれば,
  $\sigma(n-2) = \tau(n-2),\ \sigma(n-3) = \tau(n-3),..., \sigma(1) = \tau(1)$が示せる.\\
  (メモ) (1)-(5) より両側剰余類$B\backslash G/B$の完全代表系はすべての置換行列からなる集合であることが分かった.\\
  (メモ2) (4) の後半を使っていないので, (5) は想定解答ではなさそうな気がする.
\end{ans}

\fakesubsection{2.8 正規部分群と剰余群}

\begin{ans}
  (1) 正規部分群でない. $g = (3\ 4)$, $h = (1\ 2)$とおくと$h \in H$で
  $ghg^{-1} = (3\ 4)(1\ 3)(3\ 4) = (1\ 4) \notin H$.\\
  (2) 正規部分群でない.
  $g = \begin{pmatrix}
    1 & 0 \\
    0 & 2
  \end{pmatrix}$, $h = \begin{pmatrix}
    \frac{1}{\sqrt{2}} & \frac{1}{\sqrt{2}} \\
    - \frac{1}{\sqrt{2}} & \frac{1}{\sqrt{2}}
  \end{pmatrix}$とおくと
  \begin{align*}
    ghg^{-1} &= \begin{pmatrix}
      1 & 0 \\
      0 & 2
    \end{pmatrix}\begin{pmatrix}
      \frac{1}{\sqrt{2}} & \frac{1}{\sqrt{2}} \\
      - \frac{1}{\sqrt{2}} & \frac{1}{\sqrt{2}}
    \end{pmatrix}\begin{pmatrix}
      1 & 0 \\
      0 & \frac{1}{2}
    \end{pmatrix} = \begin{pmatrix}
      \frac{1}{\sqrt{2}} & \frac{1}{2\sqrt{2}} \\
      - \sqrt{2} & \frac{1}{\sqrt{2}}
    \end{pmatrix} \\
    \transpose{(ghg^{-1})}(ghg^{-1}) &= \begin{pmatrix}
      \frac{1}{\sqrt{2}} & - \sqrt{2} \\
      \frac{1}{2\sqrt{2}} & \frac{1}{\sqrt{2}}
    \end{pmatrix}\begin{pmatrix}
      \frac{1}{\sqrt{2}} & \frac{1}{2\sqrt{2}} \\
      - \sqrt{2} & \frac{1}{\sqrt{2}}
    \end{pmatrix} \neq I_n
  \end{align*}
  したがって$ghg^{-1} \notin H$.\\
  (3) 正規部分群でない.
  $g = \begin{pmatrix}
    1 & 0 \\
    0 & i
  \end{pmatrix}$, $h = \begin{pmatrix}
    1 & 1 \\
    0 & 1
  \end{pmatrix}$とおくと,
  \[
    ghg^{-1} = \begin{pmatrix}
      1 & 0 \\
      0 & i
    \end{pmatrix}\begin{pmatrix}
      1 & 1 \\
      0 & 1
    \end{pmatrix}\begin{pmatrix}
      1 & 0 \\
      0 & -i
    \end{pmatrix} = \begin{pmatrix}
      1 & -i \\
      0 & 1
    \end{pmatrix} \notin \mathrm{GL}_2(\mathbb{R})
  \]
  (4) 正規部分群である. $(1\ 2)(3\ 4)(1\ 3)(2\ 4) = (1\ 4)(2\ 3)$より,
  $H$は$(1\ 2)(3\ 4)$と$(1\ 3)(2\ 4)$で生成される.
  また演習問題2.3.9より, $G$は$(1\ 2\ 3\ 4)$と$(1\ 2)$で生成される.
  命題2.8.7より, $h = (1\ 2)(3\ 4), (1\ 3)(2\ 4)$と$g = (1\ 2\ 3\ 4), (1\ 2)$の組み合わせについて
  $ghg^{-1} \in H$であることを見ればよい:
  \begin{align*}
    (1\ 2\ 3\ 4)(1\ 2)(3\ 4)(1\ 2\ 3\ 4)^{-1} = (1\ 4)(2\ 3) \in H\\
    (1\ 2)(1\ 2)(3\ 4)(1\ 2)^{-1} = (1\ 2)(3\ 4) \in H\\
    (1\ 2\ 3\ 4)(1\ 3)(2\ 4)(1\ 2\ 3\ 4)^{-1} = (1\ 3)(2\ 4) \in H \\
    (1\ 2)(1\ 3)(2\ 4)(1\ 2)^{-1} = (1\ 4)(2\ 3) \in H
  \end{align*}
  (5) 正規部分群である. $g = \begin{pmatrix}
    g_{11} & 0 \\
    g_{21} & g_{22}
  \end{pmatrix}$, $h = \begin{pmatrix}
    h_{11} & 0 \\
    h_{21} & h_{11}
  \end{pmatrix}$とおくと,
  $ghg^{-1}$の$(1, 1)$成分は$g_{11}h_{11}g_{11}^{-11} = h_{11}$,
  $(2, 2)$成分は$g_{22}h_{11}g_{22}^{-1} = h_{11}$であるから,
  $ghg^{-1} \in H$.
\end{ans}

\begin{ans}
  $h \in H$とする. $g \in H$ならば$ghg^{-1} \in H$は明らか.
  $g \notin H$ならば, $G = H \sqcup gH = H \sqcup Hg$より, $gH = Hg$すなわち$gHg^{-1} = H$.
  したがって任意の$g \in G$, $h \in H$に対して$ghg^{-1} \in H$.
\end{ans}

\begin{ans}
  ($N_1N_2$が部分群であること)
  $1 \in N_1N_2$.
  $N_1N_2$の任意の$2$つの元$h_1h_2, h_1^\prime h_2^\prime$ ($h_1, h_1^\prime \in N_1$かつ$h_2, h_2^\prime \in N_2$)について
  $h_1 h_2 h_1^\prime h_2^\prime = h_1 (h_2 h_1^\prime h_2^{-1}) h_2 h_2^\prime \in N_1N_2$.
  また$N_1N_2$の任意の元$h_1h_2$ ($h_1 \in N_1$, $h_2 \in N_2$)に対して$h_1h_2^{-1}h_1^{-1} \in N_2$より$(h_1h_2)^{-1} = h_2^{-1}h_1^{-1} \in N_1N_2$.\\
  (正規部分群であること) 任意の$g \in G$に対して$gN_1N_2g^{-1} = (gN_1g^{-1})(gN_2g^{-1}) \subset N_1N_2$.
\end{ans}

\begin{ans}
  $\mathfrak{S}_3$の位数は$6$なので, 部分群の位数は$1, 2, 3, 6$のいずれかである.
  位数$1$の部分群は$\{1\}$, 位数$6$の部分群は$\mathfrak{S}_3$である.
  $2$と$3$は素数なので, これらに対応する部分群は巡回群のみである.
  したがって, 部分群は
  $\{1\}$, $\mathfrak{S}_3$,
  $\gen{(1\ 2)}$, $\gen{(2\ 3)}$, $\gen{(1\ 3)}$,
  $\gen{(1\ 2\ 3)}$
  で尽くされる.
  これらの部分群のうち, $\{1\}$と$\mathfrak{S}_3$は明らかに正規部分群である.
  また$\gen{(1\ 2\ 3)}$は指数$2$の部分群であるから,
  演習問題2.8.2より正規部分群である.
  これら以外は正規部分群ではない. 例えば
  $(2\ 3)(1\ 2)(2\ 3)^{-1} = (1\ 3) \notin \gen{(1\ 2)}$
  より$\gen{(1\ 2)}$が正規部分群でないことが分かる. 他も同様.
\end{ans}

\begin{ans}
  四元数群を$G$と書くことにする.
  $G$の位数は$8$なので, 部分群の位数は$1, 2, 4, 8$のいずれかである.
  このうち位数が$1, 8$であるのは$\{1\}, G$である.
  それ以外の位数が$2$または$4$の部分群$H$について, もし$i \in H$ならば, $\gen{i} \subset H$であるが,
  $\order{\gen{i}} = 4$より$H = \gen{i}$である.
  $j, k$についても同様.
  $i, j, k \notin H$で$H \neq \{1\}$であるようなものは$H = \gen{-1}$のみである.
  以上により, $G$の部分群は$\{1\}, \gen{-1}, \gen{i}, \gen{j}, \gen{k}, G$である.
  これらの部分群のうち, 明らかに$\{1\}$と$G$は正規部分群である.
  また$\gen{i}, \gen{j}, \gen{k}$は指数$2$なので,
  演習問題2.8.2より正規部分群である.
  $\gen{-1}$が正規部分群であることも容易に確かめられる.
  以上により, $G$の部分群$\{1\}, \gen{-1}, \gen{i}, \gen{j}, \gen{k}, G$はすべて正規部分群である.
\end{ans}

\fakesubsection{2.9 群の直積}

\begin{ans}
  容易なので略.
\end{ans}

\begin{ans}
  まず$\phi_1: G_1 \rightarrow G_1$を定める. $g_1 \in G_1$に対して,
  $\phi(g_1, 1_{G_2}) = (g_1^\prime, g_2^\prime)$であるとする.
  $(g_1, 1_{G_2})$の位数は$n_1$の約数であり,
  $\phi(g_1, 1_{G_2}) = (g_1^\prime, g_2^\prime)$の位数は更にその約数であるから,
  $g_2^\prime$の位数は$n_1$の約数である.
  一方$G_2$の位数は$n_1$と互いに素であったから,
  $g_2^\prime$の位数は$1$, すなわち$g_2^\prime = 1_{G_2}$である.
  そこで$\phi_1(g_1) = g_1^\prime$として$\phi_1: G_1 \rightarrow G_1$を定めれば,
  $\phi_1$は$\phi(g_1, 1_{G_2}) = (\phi_1(g_1), 1_{G_2})$をみたす.
  このように定めた$\phi_1$が準同型であることは,
  $(\phi_1(gh), 1_{G_2})
  = \phi(gh, 1_{G_2})
  = \phi(g, 1_{G_2})\phi(h, 1_{G_2})
  = (\phi_1(g), 1_{G_2})(\phi_1(h), 1_{G_2})
  = (\phi_1(g)\phi_1(h), 1_{G_2})$
  からしたがう.
  同様にして$\phi(1_{G_1}, g_2) = (1_{G_1}, \phi_2(g_2))$をみたす
  準同型$\phi_2: G_2 \rightarrow G_2$を定めることができる.
  この$\phi_1, \phi_2$について,
  $\phi(g_1, g_2)
  = \phi(g_1, 1_{G_2})\phi(1_{G_1}, g_2)
  = (\phi_1(g_1), 1_{G_2})(1_{G_1}, \phi_2(g_2))
  = (\phi_1(g_1), \phi_2(g_2))$.
\end{ans}

\begin{ans}
  (1) $15 = 1 \cdot 8 + 7,\ 8 = 1 \cdot 7 + 1$より,
  $1 = 8 - 1 \cdot 7 = 8 - 1 \cdot (15 - 1 \cdot 8) = - 15 + 2 \cdot 8$.
  よって$(x, y) = (-1, 2)$が$15x + 8y = 1$の解の$1$つであるから,
  $15 \cdot (-1) \cdot 5 + 8 \cdot 2 \cdot 2 = -43$.
  (2) $35 = 1 \cdot 24 + 11,\ 24 = 2 \cdot 11 + 2,\ 11 = 5 \cdot 2 + 1$より,
  $1 = 11 - 5 \cdot 2
  = 11 - 5 \cdot (24 - 2 \cdot 11)
  = (-5) \cdot 24 + 11 \cdot 11
  = (-5) \cdot 24 + 11 \cdot (35 - 1 \cdot 24)
  = 11 \cdot 35 + (-16) \cdot 24$.
  よって$(x, y) = (11, -16)$が$35x + 24y = 1$の解の$1$つであるから,
  $35 \cdot 11 \cdot 5 + 24 \cdot (-16) \cdot 4 = 389$.
\end{ans}

\fakesubsection{2.10 準同型定理}

\begin{ans}
  $\phi: G \rightarrow H_2$を$\phi(re^{i\theta}) = r$と定義すると
  $\phi$は全射準同型であり, $\mathrm{Ker}\phi = H_1$である.
  したがって準同型定理より$G/H_1 \cong H_2$.
  また$\psi: G \rightarrow H_1$を$\psi(re^{i\theta}) = e^{i\theta}$と定義すると
  $\psi$は全射準同型であり, $\mathrm{Ker}\psi = H_2$である.
  したがって準同型定理より$G/H_2 \cong H_1$.
\end{ans}

\begin{ans}
  $\phi: \mathbb{R} \rightarrow \mathbb{R}/a\mathbb{Z}$を
  $\phi(x) = ax + a\mathbb{Z}$と定めると, $\phi$は全射準同型で
  $\mathrm{Ker}\phi = \mathbb{Z}$である. よって準同型定理より
  $\mathbb{R}/\mathbb{Z} \cong \mathbb{R}/a\mathbb{Z}$.
\end{ans}

\begin{ans}
  $\phi: G \rightarrow \mathbb{R}^\times$を
  $\begin{pmatrix}
    a_{11} & 0 \\
    a_{21} & a_{22}
  \end{pmatrix} \mapsto \frac{a_{11}}{a_{22}}$
  と定義すると, $\phi$は全射準同型で$\mathrm{Ker}\phi = H$であるから,
  準同型定理より$G/H \cong \mathbb{R}^\times$.
\end{ans}

\begin{ans}
  $G/H = \{H, x_1 + H, x_2 + H,..., x_{n-1} + H\}$であるとする.
  これが位数$n$の群をなすことから, $nx_i + H = H$すなわち$nx_i \in H (i = 1,..., n-1)$である.
  一方, 任意の$x \in G$はある$i$と$h \in H$について$x = x_i + h$と書けるから,
  $nx = nx_i + nh \in H$. よって$nG \subset H$.
\end{ans}

\begin{ans}
  一般に, $G$の指数$p$の部分群の数を求める.
  $H$を$G$の指数$p$の部分群とすると, 前問より$H$は$pG$を含む.
  そこで, 例題2.10.12と同様に, $H$は$G/pG$の指数$p$の部分群と1対1に対応する.
  $G/pG \cong \Z{p} \times \Z{p}$であるから,
  $\Z{p} \times \Z{p}$の指数$p$の, すなわち位数$p$の部分群の数を求めればよい.
  この部分群は巡回部分群であり, $\langle(x, y)\rangle$ と書ける.
  $x = \overline{0}$であるようなものは, $\langle(\overline{0}, \overline{1})\rangle$のみである.
  $x \neq \overline{0}$の場合は, $x = \overline{1}$と仮定しても一般性を失わない.
  このとき$y$の取りうる値は$y = \overline{0}, \overline{1},..., \overline{p-1}$の$p$通りであるが,
  これらすべてについて, 生成される部分群が互いに異なることは明らかである.
  したがって, 指数$p$の部分群の数は$p + 1$個である.
\end{ans}

\begin{ans}
  中国剰余定理より
  $G \cong \Z{45} \times \Z{3} \times \Z{8} \times \Z{2} \times \Z{7}$
  である. $\Z{45}$, $\Z{3}$, $\Z{7}$において$2$倍写像は全単射なので,
  $G/2G \cong \Z{2} \times \Z{2}$である.
  したがって例題2.10.12と同様にして, $G$の指数$2$の部分群の個数は$3$である.
\end{ans}

\begin{ans}
  (1) $G = \Z{12}$とおく. $\abs{G} = 12$なので, 部分群の位数の候補は$1, 2, 3, 4, 6, 12$である.
  これらのうち, $1, 12$にはそれぞれ自明な部分群$\{0\}, G$が対応する.
  また, 中国剰余定理より$G \cong \Z{3} \times \Z{4}$である.
  以下, $G$の部分群$H$の位数が$2, 3, 4, 6$の場合を考える.\\
  \underline{位数$2$の場合}:
  $H$は巡回群で, ある元$(\overline{a}, \overline{b}) \in \Z{3} \times \Z{4}$で生成される.
  $2\overline{a} = \overline{0}\ (\in \Z{3})$より$\overline{a} = \overline{0}$.
  よって$\overline{b} \in \Z{4}$の位数が$2$なので,
  $(\overline{a}, \overline{b}) = (\overline{0}, \overline{2})$.
  逆に, $\langle (\overline{0}, \overline{2}) \rangle$は位数$2$の部分群なので,
  これが位数$2$の唯一の部分群である.\\
  \underline{位数$3$の場合}:
  $H$は巡回群で, ある元$(\overline{a}, \overline{b}) \in \Z{3} \times \Z{4}$で生成される.
  $3\overline{b} = \overline{0}\ (\in \Z{4})$より$\overline{b} = \overline{0}$.
  よって$H$は$\Z{3}$の部分群と同型であるが, 位数の比較により, $\Z{3}$自体と同型である.
  すなわち位数$3$の部分群は$\langle (\overline{1}, \overline{0}) \rangle$のみである.\\
  \underline{位数$4$の場合}:
  $(\overline{a}, \overline{b}) \in H$とすると,
  $4\overline{a} = \overline{0}\ (\in \Z{3})$より$\overline{a} = 0$.
  あとは位数$3$の場合と同様の議論により, 位数$4$の部分群は$\langle (\overline{0}, \overline{1}) \rangle$のみである.\\
  \underline{位数$6$の場合}:
  $(\overline{a}, \overline{b}) \in H$とすると,
  $6\overline{b} = \overline{0}\ (\in \Z{4})$より$\overline{a} = \overline{0}, \overline{2}$.
  これをみたす$(\overline{a}, \overline{b})$は$6$組しかないから, これらが部分群$H$をなすはずである.
  実際, $\langle (\overline{1}, \overline{2}) \rangle$が位数$6$の部分群である.\\
  (2) $G = \Z{18}$とおく. $\abs{G} = 18$なので, 部分群の位数の候補は$1, 2, 3, 6, 9, 18$である.
  これらのうち, $1, 18$にはそれぞれ自明な部分群$\{0\}, G$が対応する.
  また, 中国剰余定理より$G \cong \Z{2} \times \Z{9}$である.
  以下, $G$の部分群$H$の位数が$2, 3, 4, 6$の場合を考える.\\
  \underline{位数$2$の場合}:
  (1) の位数$3$の場合と同様で, $\langle (\overline{1}, \overline{0}) \rangle$のみ.\\
  \underline{位数$3$の場合}:
  (1) の位数$2$の場合と同様で, $\langle (\overline{0}, \overline{3}) \rangle$のみ.\\
  \underline{位数$6$の場合}:
  (1) の位数$6$の場合と同様で, $\langle (\overline{1}, \overline{3}) \rangle$のみ.\\
  \underline{位数$9$の場合}:
  (1) の位数$4$の場合と同様で, $\langle (\overline{0}, \overline{1}) \rangle$のみ.\\
  以上を一般化すると, $p, q$を素数として$\Z{p^2q}$の部分群を求める問題となる. (追記するかも.)
\end{ans}

\begin{ans}
  (1) $G$の元の位数の候補は$1, 2, 3, 6$である.
  もし位数$6$の元が存在すれば$G$は巡回群$\Z{6}$に同型であり,
  $\overline{2}$ (に同型で対応する元) が位数$3$である.
  そこで位数$6$の元が存在しない場合を考える.
  この場合に位数$3$の元が存在することを背理法で示そう.
  もし位数$3$の元が存在しなければ, 単位元以外の元はすべて位数$2$なので,
  演習問題2.4.8より$G$は可換である. よって$x$を単位元以外の元とすると
  $H = \langle x \rangle$は$G$の位数$2$の正規部分群であり, $G/H$は位数$3$の巡回群である.
  そこで$G/H$の生成元である剰余類の代表元を$g$とすると,
  $gHgH \neq H$より$(g)^2 \notin H$であるが,
  このことから$g$の位数が$3$以上であることになり, 矛盾.
  よって$G$に位数$6$の元が存在しない場合にも, 位数$3$の元が存在することが分かった.\\
  (2) $G$にもし位数$6$の元が存在すれば, (1) と同様に, $\overline{3}$が位数$2$である.
  そこで位数$6$の元が存在しない場合を考える.
  演習問題2.8.2より, $H$は$G$の正規部分群であることに注意する.
  位数$2$の巡回群$G/H$の生成元である剰余類の代表元を$g$とすると,
  $gHgHgH \neq H$より$g^3 \notin H$であるが,
  このことから$g$の位数は$3$ではない. $gH$が$G/H$の生成元なので, $g$の位数は$1$でもない.
  したがって$g$の位数は$2$である.\\
  (3) $x$を$G$の位数$3$の元, $y$を位数$2$の元とする.
  $H = \langle x \rangle$は (2) で見たように正規部分群であるが,
  $G$が可換なら$K = \langle y \rangle$も正規部分群である.
  また, 各元の位数を比較することにより$H \cap K = 1_G$である.
  $h_1, h_2 \in H$, $k_1, k_2 \in K$について, もし$h_1k_1 = h_2k_2$ならば$h_2^{-1}h_1 = k_2k_1^{-1} = 1_G$より
  $h_1 = h_2$かつ$k_1 = k_2$である. したがって$\abs{HK} = \abs{H}\abs{K} = 6$なので$HK = G$.
  命題2.9.2より$G \cong H \times K \cong \Z{3} \times \Z{2} \cong \Z{6}$.\\
  (4) $G$が非可換であるとする. $x$を位数$3$の元, $H = \langle x \rangle$として, $y$を$G/H$の生成元の代表元とする.
  (2) でみたように, $y$は位数$2$の元である.
  また$G$の任意の元は, $H$の生成元$x$を用いて$y^kx^l\ (k = 0, 1,\ l = 0, 1, 2)$と書ける.
  もし$y$が$x$と可換であれば, $y^{k_1}x^{l_1}y^{k_2}x^{l_2} = y^{k_2}x^{l_2}y^{k_1}x^{l_1}$であるから,
  $G$は可換となり矛盾. したがって$y$と$x$は非可換である.
  また, $(x^2)^2 = x$より, $y$は$x^2$とも非可換である. (もし$y$が$x^2$と可換なら$x = (x^2)^2$とも可換となって矛盾.)
  そこで, $y, xyx^{-1}, x^2yx^{-2}$を考えると,
  $y$と$x$が非可換かつ$y$と$x^2$が非可換であることからこれらは互いに異なる元であり,
  また明らかに互いに共役である.
  さらに$y$が位数$2$であることからこれらは単位元とは異なる元であり, $(x^kyx^{-k})^2 = x^ky^2k^{-k} = 1_G$より位数は$2$である.
  $G$にはこれら$3$つの元の他には$H$の元 (位数は$1$か$3$) しかないので, 位数$2$の元はこの$3$つだけである.\\
  (5) $\rho(g)(i) = \rho(g)(j) \Rightarrow gx_ig^{-1} = gx_jg^{-1} \Rightarrow x_i = x_j$であるから,
  $\rho(g)$は単射. したがって$\rho(g) \in \mathfrak{S}_3$.
  また$x_{\rho(gh)(i)}
  = ghx_i(gh)^{-1}
  = gx_{\rho(h)(i)}g^{-1}
  = x_{\rho(g)(\rho(h)(i))}
  = x_{(\rho(g)\rho(h))(i)}$
  より, $\rho(gh) = \rho(g)\rho(h)$. すなわち$\rho$は準同型である.
  もし$\rho(g) = 1_{\mathfrak{S}_3}$ならば$gx_ig^{-1} = x_i\ (i = 1, 2, 3)$
  すなわち$g$と$x_i$は可換であるが, (4) で見たように$y$は$x, x^2$と非可換なので$g$は$x, x^2$とは異なる元である.
  また, $x_1, x_2, x_3$は$yH$の元なので$y, yx, yx^2$と書けるはずであるが,
  これらは$y$が$x, x^2$と非可換であることから互いに非可換である. よって$g$は$x_1, x_2, x_3$のいずれとも異なる.
  したがって$g = 1_G$であるから, $\rho$は単射. $G$と$\mathfrak{S}_3$の位数が等しいことから, $\rho$は同型.
\end{ans}

\fakesection{第3章 群を学ぶ理由}
\fakesection{第4章 群の作用}

\fakesubsection{4.1 群の作用}

\begin{ans}
  $x_2x_1 = x_2,\ x_2x_2 = x_1,\ x_2x_3 = x_4,\ x_2x_4 = x_3$
  より, $\rho(x_2) = (1\ 2)(3\ 4)$.
  同様にして, $\rho(x_3) = (1\ 3)(2\ 4)$, $\rho(x_4) = (1\ 4)(2\ 3)$.
\end{ans}

\begin{ans}
  \begin{align*}
    &(2\ 3)1 = (2\ 3)\\
    &(2\ 3)(1\ 2) = (1\ 3\ 2)\\
    &(2\ 3)(1\ 3) = (1\ 2\ 3)\\
    &(2\ 3)(2\ 3) = 1\\
    &(2\ 3)(1\ 2\ 3) = (1\ 3)\\
    &(2\ 3)(1\ 3\ 2) = (1\ 2)\\
  \end{align*}
  となるので$\rho((2\ 3)) = \begin{pmatrix*}
    1 & 2 & 3 & 4 & 5 & 6\\
    4 & 6 & 5 & 1 & 3 & 2
  \end{pmatrix*} = (1\ 4)(2\ 6)(3\ 5)$.
\end{ans}

\begin{ans}
  \begin{align*}
    &(1\ 3\ 2)x_1 = (1\ 3\ 2) \in x_3H\\
    &(1\ 3\ 2)x_2 = 1 \in x_1H\\
    &(1\ 3\ 2)x_3 = (1\ 2\ 3)
  \end{align*}
  となるので$\rho((1\ 3\ 2)) = (1\ 3\ 2)$.
\end{ans}

\begin{ans}
  \begin{align*}
    &\Ad{(1\ 2\ 3)}(1) = 1\\
    &\Ad{(1\ 2\ 3)}((1\ 2)) = (1\ 2\ 3)(1\ 2)(1\ 3\ 2) = (2\ 3)\\
    &\Ad{(1\ 2\ 3)}((1\ 3)) = (1\ 2\ 3)(1\ 3)(1\ 3\ 2) = (1\ 2)\\
    &\Ad{(1\ 2\ 3)}((2\ 3)) = (1\ 2\ 3)(2\ 3)(1\ 3\ 2) = (1\ 3)\\
    &\Ad{(1\ 2\ 3)}((1\ 2\ 3)) = (1\ 2\ 3)\\
    &\Ad{(1\ 2\ 3)}((1\ 3\ 2)) = (1\ 3\ 2)
  \end{align*}
  となるので$\rho((1\ 2\ 3)) = (2\ 4\ 3)$.
\end{ans}

\begin{ans}
  (1)
  \begin{align*}
    &ix_1 = i \cdot 1 = i = x_3\\
    &ix_2 = i \cdot (-1) = -i = x_4\\
    &ix_3 = i \cdot i = -1 = x_2\\
    &ix_4 = i \cdot (-i) = 1 = x_1\\
    &ix_5 = i \cdot j = k = x_7\\
    &ix_6 = i \cdot (-j) = -k = x_8\\
    &ix_7 = i \cdot k = -j = x_6\\
    &ix_8 = i \cdot (-k) = j = x_5
  \end{align*}
  となるので$\rho(i) = (1\ 3\ 2\ 4)(5\ 7\ 6\ 8)$.\\
  (2)
  \begin{align*}
    &kx_1 = k \cdot 1 = k = x_7\\
    &kx_2 = k \cdot (-1) = -k = x_8\\
    &kx_3 = k \cdot i = j = x_5\\
    &kx_4 = k \cdot (-i) = -j = x_6\\
    &kx_5 = k \cdot j = -i = x_4\\
    &kx_6 = k \cdot (-j) = i = x_3\\
    &kx_7 = k \cdot k = -1 = x_2\\
    &kx_8 = k \cdot (-k) = 1 = x_1
  \end{align*}
  となるので$\rho(k) = (1\ 7\ 2\ 8)(3\ 5\ 4\ 6)$.
\end{ans}

\begin{ans}
  (1)
  \begin{align*}
    &yy^{-1} = 1\\
    &yxy^{-1} = x^3\\
    &yx^2y^{-1} = (yxy^{-1})^2 = (x^3)^2 = x^6\\
    &yx^3y^{-1} = (yxy^{-1})^3 = (x^3)^3 = x^9 = x^2\\
    &yx^4y^{-1} = (yxy^{-1})^4 = (x^3)^4 = x^{12} = x^5\\
    &yx^5y^{-1} = (yxy^{-1})^5 = (x^3)^5 = x^{15} = x\\
    &yx^6y^{-1} = (yxy^{-1})^6 = (x^3)^6 = x^{18} = x^4\\
  \end{align*}
  となるので, $\Ad{y}$の$\langle x \rangle$への作用は置換
  $\sigma = (1\ 3\ 2\ 6\ 4\ 5)$で表される. (ただしここでは$x^i = i$と考えている.)
  $\sigma^{100}(1) = \sigma^{6 \cdot 16 + 4}(1) = \sigma^{4}(1) = 4$なので, $y^{100}xy^{-100} = x^4$.\\
  (2)
  \begin{align*}
    &yy^{-1} = 1\\
    &yxy^{-1} = x^5\\
    &yx^2y^{-1} = (yxy^{-1})^2 = (x^5)^2 = x^{10} = x^3\\
    &yx^3y^{-1} = (yxy^{-1})^3 = (x^5)^3 = x^{15} = x\\
    &yx^4y^{-1} = (yxy^{-1})^4 = (x^5)^4 = x^{20} = x^6\\
    &yx^5y^{-1} = (yxy^{-1})^5 = (x^5)^5 = x^{25} = x^4\\
    &yx^6y^{-1} = (yxy^{-1})^6 = (x^5)^6 = x^{30} = x^2\\
  \end{align*}
  となるので, $\Ad{y}$の$\langle x \rangle$への作用は置換
  $\sigma = (1\ 5\ 4\ 6\ 2\ 3)$で表される. (ただしここでは$x^i = i$と考えている.)
  $\sigma^{1000}(1) = \sigma^{6 \cdot 166 + 4}(1) = \sigma^{4}(1) = 2$なので, $y^{1000}xy^{-1000} = x^2$.\\
\end{ans}

\begin{ans}
  $G$のある元$g$について
  $g\frac{\bm{x}}{\norm{\bm{x}}} = \frac{\bm{y}}{\norm{\bm{y}}}$
  ならば
  $g\bm{x} = \bm{y}$
  なので, はじめから$\norm{\bm{x}} = \norm{\bm{y}} = 1$であるとして一般性を失わない.
  また$\bm{x} = [1, 0,..., 0]$に対して$g_1\bm{x} = \bm{y}$かつ$g_2\bm{x} = \bm{y}^\prime$ならば$g_2g_1^{-1}\bm{y} = \bm{y}^\prime$なので,
  $\bm{x} = [1, 0,..., 0]$の場合を示せばよい.
  $\bm{y_1} = \bm{y}$とし, $\bm{y_2},..., \bm{y_n}$を, Gram-Schmidtの直交化により
  $\{\bm{y_1},.., \bm{y_n}\}$が$V$の正規直交基底であるものとしてとる.
  さらに$\bm{y_2}$の符号を必要ならば入れ替えることにして
  $g = (\bm{y_1},..., \bm{y_n})$と定めると, $g \in \SO{n}$かつ$g\bm{x} = \bm{y_1} = \bm{y}$である.
\end{ans}

\begin{ans}
  (1) $\sigma((2, 4)) = (\sigma(2), \sigma(4)) = (1, 4)$.\\
  (2) 任意の$\sigma \in G$に対して$\sigma((1, 1)) = (\sigma(1), \sigma(1))$なので,
  $(1, 1)$の軌道は$(i, i)$の形をしている. 逆に任意の$i \in X$に対して$\sigma = (1\ i)$と定めれば
  $\sigma((1, 1)) = (i, i)$なので, $(1, 1)$の軌道は$\{(i, i) \mid i \in X\}$である.
  他の軌道として$(1, 2)$の軌道を考えると, 任意の$\sigma \in G$に対して$\sigma(1) \neq \sigma(2)$なので,
  $(1, 2)$の軌道は$(i, j)$ ($i \neq j$) の形をしている. 逆に$i, j \in X, i \neq j$に対して適当に$\sigma \in G$を定めて
  $\sigma(1) = i$, $\sigma(2) = j$となるようにできて,$\sigma((1, 2)) = (i, j)$となるので,
  $(1, 2)$の軌道は$\{(i, j) \mid i, j \in X, i \neq j\}$である.
  以上の$2$つの軌道で$Y$の元は尽くされるので, これらが$Y$における$G$の軌道の全てである.\\
  (3) $\sigma((1, 1)) = (1, 1) \Longleftrightarrow \sigma(1) = 1$なので, $(1, 1)$の安定化群は$\mathfrak{S}_{n-1}$.
  また$\sigma((1, 2)) = (1, 2) \Longleftrightarrow \sigma(1) = 1 \land \sigma(2) = 2$なので, $(1, 2)$の安定化群は$\mathfrak{S}_{n-2}$.
  ($n = 2$のとき$\mathfrak{S}_0 = \{1\}$である.)
\end{ans}

\begin{ans}
  (1) $g\bm{x} = \bm{x}$なる$g \in G$は, $[1, 0]$を$[1, 0]$に, $[0, 1]$を$[1, 0]$と線形独立なベクトルに移す行列であるから,
  $G_{\bm{x}} = \biggl\{\begin{pmatrix*}
    1 & a_{12}\\
    0 & a_{22}
  \end{pmatrix*} \biggm\vert a_{22} \neq 0\biggr\}$.\\
  (2) 任意の$g \in G$に対して$g\bm{x} \neq [0, 0]$なので,
  $[0, 0]$は$g$の軌道に属さない.
  逆に, $[a_{11}, a_{21}] \neq [0, 0]$ならば
  $g = \begin{pmatrix*}
    a_{11} & -a_{21}\\
    a_{21} & a_{11}
  \end{pmatrix*}$とすると
  $g \in G$で$g\bm{x} = [a_{11}, a_{21}]$なので,
  $[0, 0]$以外の点はすべて$\bm{x}$の軌道に属する.
  したがって$G \cdot \bm{x} = \{y \mid y \in \mathbb{R}^2\setminus[0, 0]\}$.
\end{ans}

\begin{ans}
  (1) $1$を含む共役類は明らかに$\{1\}$である.
  また$-1$も他の元と可換なので, $-1$を含む共役類は$\{-1\}$.
  また$i$については$ij = j \cdot (-i)$, $ik = k \cdot (-i)$などから$i$を含む共役類は$\{\pm i\}$.
  同様に, 他の共役類は$\{\pm j\}$, $\{\pm k\}$である.\\
  (2) $1$はすべての元と可換なので中心化群は$G$.
  $i$と可換であるのは$\{\pm 1, \pm i\}$だけなので、これが$i$の中心化群である.
\end{ans}

\begin{ans}
  (1) $\sigma$は
  $A_1 \rightarrow A_2 \rightarrow A_3 \rightarrow A_4 \rightarrow A_5 \rightarrow A_6 \rightarrow A_7 \rightarrow A_8 \rightarrow A_1$
  と作用するので,
  $\sigma l_1 = l_2$, $\sigma l_2 = l_3$, $\sigma l_3 = l_4$, $\sigma l_4 = l_1$である.
  よって$\rho(\sigma) = (1\ 2\ 3\ 4)$.
  $\tau$は$A_1 \rightarrow A_1$,
  $A_2 \rightarrow A_8 \rightarrow A_2$,
  $A_3 \rightarrow A_7 \rightarrow A_3$,
  $A_4 \rightarrow A_6 \rightarrow A_4$,
  $A_5 \rightarrow A_5$と作用するので,
  $\sigma l_1 = l_1$, $\sigma l_2 = l_4$, $\sigma l_3 = l_3$, $\sigma l_4 = l_2$である.
  よって$\rho(\tau) = (2\ 4)$.\\
  (2) $D_8 \ni g$で$l_1$が不変であることは, $g$によって$A_1$が$A_1$または$A_5$に移ることと同値である.
  $A_1$が$A_1$に移すのは$1, \tau$であり, $A_1$を$A_5$に移すのは$\sigma^4, \tau\sigma^4$なので,
  $l_1$の安定化群は$\{1, \sigma^4, \tau, \tau\sigma^4\}$.
\end{ans}

\begin{ans}
  命題4.1.10の表記に倣って, $D_n = \{1, t, \cdots , t^{n-1}, r, rt, \cdots , rt^{n-1}\}$と書くことにする.\\
  (a)(1)
  \begin{align*}
    &t^lt^k(t^l)^{-1} = t^k\\
    &rt^lt^k(rt^l)^{-1} = rt^lt^kt^{-l}r = rt^kr = t^{-k}\\
    &t^lrt^k(t^l)^{-1} = rt^{-l}t^kt^{-l} = rt^{k-2l}\\
    &rt^lrt^k(rt^l)^{-1} = rt^lrt^kt^{-l}r = rt^lt^{-k+l} = rt^{-k+2l}
  \end{align*}
  より, $D_4$の共役類は
  $\{
    \{1\},
    \{t, t^3\},
    \{t^2\},
    \{r, rt^2\},
    \{rt, rt^3\}
  \}$
  である.\\
  (a)(2) $\{1\}$, $\{t^2\}$については, 明らかに中心化群は$D_4$である.
  $t$については, 上の式から$\{1, t, t^2, t^3\}$が中心化群である (代表元として$t^3$を選んでも同じ).
  $r$については, 上の$3$つ目の式で$l = 0, 2$, 上の$4$つ目の式で$l = 0, 2$の場合が対応するので,
  中心化群は$\{1, t^2, r, rt^2\}$である (代表元として$rt^2$を選んでも同じ).
  $rt$については, 上の$3$つ目の式で$l = 0, 2$, 上の$4$つ目の式で$l = 1, 3$の場合が対応するので,
  中心化群は$\{1, t^2, rt, rt^3\}$である (代表元として$rt^3$を選んでも同じ).\\
  (b)(1)
  上の$4$つの式から, $D_5$の共役類は
  $\{
    \{1\},
    \{t, t^4\},
    \{t^2, t^3\},
    \{r, rt, rt^2, rt^3, rt^4\}
  \}$
  である.\\
  (b)(2) $\{1\}$については, 明らかに中心化群は$D_5$である.
  $D_5$の元については任意の$k$に対して$t^k \neq t^{-k}$なので,
  $\{t, t^4\}, \{t^2, t^3\}$については,
  どの代表元を選んでも中心化群は$\{1, t, t^2, t^3, t^4\}$である.
  $rt^k$の形の元については, 上の$3$つ目の式が$r^k$に等しくなるのは$k - 2l \equiv k\ (\mathrm{mod}\ 5)$すなわち$l = 0$の場合のみ,
  上の$4$つ目の式が$r^k$に等しくなるのは$-k + 2l \equiv k\ (\mathrm{mod}\ 5)$すなわち$k + l \equiv 0\ (\mathrm{mod}\ 5)$の場合のみであるので,
  $r$の中心化群は$\{1, r\}$である. 代表元として他の$rt^k$をとると, $\{1, rt^{5-k}\}$が中心化群となる.\\
  (考察1) ある共役類が$1$つの元のみからなるならば, その元は$G$のすべての元と可換であるから, 中心化群は$G$となる.\\
  (考察2) 共役類が$2$つの元から成るならば, $2$つの元それぞれの中心化群は等しい.
  このことを以下で示そう.
  共役類$\{g_1, g_2\}$を考えているとすると,
  任意の$g \in Z_G(g_1)$について, $g$の共役による作用によって$g_2$は共役類の元$g_1, g_2$のいずれかに移るはずであるが, $g_1$に移ることはありえない.
  なぜなら, $g^{-1} \in Z_G(g_1)$であって$g^{-1}$の共役による作用で$g_1$が$g_2$に移ることはないからである.
  したがって$g \in Z_G(g_2)$なので$Z_G(g_1) \subset Z_G(g_2)$. 逆も同様に成り立つので$Z_G(g_1) = Z_G(g_2)$である.
\end{ans}

\begin{ans}
  (1)
  \begin{align*}
    \begin{pmatrix*}
      a & b \\
      c & d
    \end{pmatrix*} A = \begin{pmatrix*}
      2a & b \\
      2c & d
    \end{pmatrix*},\ A \begin{pmatrix*}
      a & b \\
      c & d
    \end{pmatrix*} = \begin{pmatrix*}
      2a & 2b \\
      c & d
    \end{pmatrix*}
  \end{align*}
  であるから, 両者が等しいことは$b = 0$かつ$c = 0$であることと同値である.
  よって$A$の中心化群は$\biggl\{\begin{pmatrix*}
    a & 0 \\
    0 & d \\
  \end{pmatrix*} \biggm\vert a, d \neq 0 \biggr\}$である.\\
  (2)
  \begin{align*}
    \begin{pmatrix*}
      a & b \\
      c & d
    \end{pmatrix*} A = \begin{pmatrix*}
      2a & a + 2b \\
      2c & c + 2d
    \end{pmatrix*},\ A \begin{pmatrix*}
      a & b \\
      c & d
    \end{pmatrix*} = \begin{pmatrix*}
      2a + c & 2b + d \\
      2c & 2d
    \end{pmatrix*}
  \end{align*}
  であるから, 両者が等しいことは$c = 0$かつ$a = d$であることと同値である.
  よって$A$の中心化群は$\biggl\{\begin{pmatrix*}
    a & b \\
    0 & a \\
  \end{pmatrix*} \biggm\vert a \neq 0 \biggr\}$である.
\end{ans}

\begin{ans}
  (1) 問題文に書いていないが$g \in G$とする.
  $c \neq 0$ならば$\mathrm{Im}(cz + d) \neq 0$より$cz + d \neq 0$である.
  $c = 0$ならば$g \in \mathrm{SL}_2(\mathbb{R})$より$d \neq 0$であるのでやはり$cz + d \neq 0$.
  また,
  \begin{align*}
    \frac{az + b}{cz + d}
    = \frac{(az + b)(c\bar{z} + d)}{(cz + d)(c\bar{z} + d)}
    = \frac{ac\abs{z}^2 + bd + adz + bc\bar{z}}{\abs{cz + d}^2}
  \end{align*}
  より, $\mathrm{Im}(gz) = \frac{\mathrm{Im}(z)(ad - bc)}{\abs{cz + d}^2} > 0$.
  よって$gz \in \mathbb{H}$.\\
  (2) $g = 1_G = \begin{pmatrix*}
    1 & 0 \\
    0 & 1
  \end{pmatrix*}$ならば$gz = z$.\\
  また, $G$の任意の$2$つの元を$g = \begin{pmatrix*}
    a & b \\
    c & d
  \end{pmatrix*}$, $h = \begin{pmatrix*}
    k & l \\
    m & n
  \end{pmatrix*}$とおくと,
  \begin{align*}
    g(hz) &= g\biggl(\frac{kz + l}{mz + n}\biggr)
    = \frac{a(kz + l) + b(mz + n)}{c(kz + l) + d(mz + n)}
    = \frac{(ak + bm)z + (al + bn)}{(ck + dm)z + (cl + dn)}\\
    (gh)z &= \begin{pmatrix*}
      ak + bm & al + bn \\
      ck + dm & cl + dn
    \end{pmatrix*}z = \frac{(ak + bm)z + (al + bn)}{(ck + dm)z + (cl + dn)}
  \end{align*}
  よって$g(hz) = (gh)z$.\\
  (3) $\mathbb{H}$の任意の元$x + yi$ ($y > 0$)に対して,
  $g = \begin{pmatrix*}
    \sqrt{y} & \frac{x}{\sqrt{y}} \\
    0 & \frac{1}{\sqrt{y}}
  \end{pmatrix*}$と定めると, $g \in G$であり,
  $gi = \frac{\sqrt{y} \sqrt{-1} + \frac{x}{\sqrt{y}}}{\frac{1}{\sqrt{y}}} = x + y\sqrt{-1}$.
  よって$G \cdot \sqrt{-1} = \mathbb{H}$.\\
  (4) $g \sqrt{-1} = \sqrt{-1}
  \Leftrightarrow \frac{a\sqrt{-1} + b}{c\sqrt{-1} + d} = \sqrt{-1}
  \Leftrightarrow (a - d)\sqrt{-1} + (b + c) = 0
  \Leftrightarrow g = \begin{pmatrix*}
    a & b \\
    -b & a
  \end{pmatrix*}$.
  また$\det g = 1$より, 安定化群は$\biggl\{\begin{pmatrix*}
    \cos\theta & \sin\theta \\
    -\sin\theta & \cos\theta
  \end{pmatrix*} \biggm\vert \theta \in [0, 2\pi) \biggr\}$.
\end{ans}

\begin{ans}
  (1) $(\rho(1_G)f)(h) = f(1_Gh) = f(h)$より, $\rho(1_G)f = f$.
  また, $(\rho(g_1g_2)f)(h) = f(g_1g_2h) = (\rho(g_1)f)(g_2h) = (\rho(g_2)(\rho(g_1)f))(h)$
  より, $\rho$は右作用である.\\
  (2) $(\rho(1_G)f)(h) = f(h1_G^{-1}) = f(h)$より, $\rho(1_G)f = f$.
  また, $(\rho(g_1g_2)f)(h) = f(hg_2^{-1}g_1^{-1}) = (\rho(g_1)f)(hg_2^{-1}) = (\rho(g_2)(\rho(g_1)f))(h)$
  より, $\rho$は右作用である.\\
  (3) $(\rho(1_G)f)(h) = f(1_G^{-1}h1_G) = f(h)$より, $\rho(1_G)f = f$.
  また, $(\rho(g_1g_2)f)(h) = f(g_2^{-1}g_1^{-1}hg_1g_2) = (\rho(g_2)f)(g_1^{-1}hg_1) = (\rho(g_1)(\rho(g_2)f))(h)$
  より, $\rho$は左作用である.\\
  (補足) 一般に, 左作用$\phi: G \times X \rightarrow X$があるとき,
  $\psi: G \times X \rightarrow X$を$\psi(g, x) = \phi(g^{-1}, x)$と定めると,
  \begin{align*}
    \psi(1_G, x) &= \phi(1_G, x) \\
    &= x \\
    \psi(g, \psi(h, x)) &= \phi(g^{-1}, \phi(h^{-1}, x)) \\
    &= \phi(g^{-1}h^{-1}, x) \\
    &= \psi(hg, x)
  \end{align*}
  より, $\psi$は右作用である. 左右を逆にしても同様のことが言える.
\end{ans}

\begin{ans}
  $G$が推移的に作用するとし, $G \cdot x = \{1, \cdots , n\}$であるとする. ($x$は実のところ何でもよい.)
  このとき命題4.1.23より, $\abs{G} = \abs{G \cdot x}\abs{G_x} = n\abs{G_x}$.
\end{ans}

\begin{ans}
  (1) $N$は巡回群であるから, $N$の生成元を$1$つ固定して,
  その元がどの生成元に移るかを定めることによって, $\mathrm{Aut}N$の元が定まる.
  逆に, $\mathrm{Aut}N$の元は$N$の生成元を生成元に移すものであるから, $\abs{\mathrm{Aut}N}$は生成元の個数, すなわち$16$に等しい.\\
  (2) $N$が$G$の中心に含まれないと仮定して, 矛盾を導く.
  $n \mapsto gng^{-1}$は$N$の自己同型を与えるが, この形の自己同型全体は, $\mathrm{Aut}N$の部分群をなすことに注意する.
  この部分群は$G$の内部自己同型全体のなす群において$N$上で等しいものを同一視したものであるから,
  以後$\mathrm{Inn}_GN$と書くことにしよう (これはたぶん一般的な記法ではない).
  (1) より$\abs{\mathrm{Aut}N} = 16$であったから, $\abs{\mathrm{Inn}_GN}$は$2$の冪である.
  また, $N$が$G$の中心に含まれないから, $g_0n_0g_0^{-1} \neq n_0$なる$g_0 \in G$, $n_0 \in N$がある.
  したがって, $\mathrm{Inn}_GN$の$N$への作用$(\phi, n) \mapsto \phi(n)$による$n_0$の軌道の濃度は,
  $\abs{\mathrm{Inn}_GN}$の約数で, $1$と異なるので偶数である.
  一方, この作用は$G$の$N$への共役による作用とみなせるから, $n_0$の軌道の濃度は$\abs{G}$の約数でもあるはずである.
  しかし$\abs{G}$は奇数であったから, 矛盾.\\
  (考察) $N$の位数を一般の (奇) 素数としたときにも同じ結論が得られるか? $G$の共役による$N\setminus\{1_G\}$への作用が推移的であればOK. (これは真か?)
\end{ans}

\begin{ans}
  (1) $3$は$8$の約数でない, (4) $1$の数が$8$の約数でない
\end{ans}

\fakesubsection{4.2 対称群の共役類}

\begin{ans}
  容易なので略.
\end{ans}

\begin{ans}
  (1) $n = 5$のヤング図形は以下の$7$つである.
  \begin{align*}
    \ytableausetup{smalltableaux}
    \ydiagram{5}\ \ydiagram{4, 1}\ \ydiagram{3, 2}\ \ydiagram{3, 1, 1}\ \ydiagram{2, 2, 1}\ \ydiagram{2, 1, 1, 1}\ \ydiagram{1, 1, 1, 1, 1}
  \end{align*}
  共役類の代表元としては, これらのヤング図形の長さ$2$以上の行に適当に数を当てはめたものをとればよい.
  たとえば, $(1\ 2\ 3\ 4\ 5)$, $(1\ 2\ 3\ 4)$, $(1\ 2\ 3)(4\ 5)$, $(1\ 2\ 3)$, $(1\ 2)(3\ 4)$, $(1\ 2)$, $1$
  が代表元である.\\
  (2) 共役類の中では$\mathrm{sgn}$は一致すること,
  また$A_5$の共役類は$\mathfrak{S}_5$の共役類と一致するか, またはその細分であることに注意する.
  このことから, $A_5$の共役類を求めるためには, $\mathfrak{S}_5$の共役類のうち$A_5$に含まれるもののそれぞれが, $A_5$の共役類として細分されているかどうか,
  もし細分されているならどのように細分されているかを調べればよい.
  (1) に挙げた代表元のうち, $A_5$に含まれるのは$(1\ 2\ 3\ 4\ 5)$, $(1\ 2\ 3)$, $(1\ 2)(3\ 4)$, $1$の$4$つである.

  \underline{$(1\ 2\ 3\ 4\ 5)$について.}
  長さ$5$の巡回置換はすべて
  $(\sigma(1)\ \sigma(2)\ \sigma(3)\ \sigma(4)\ \sigma(5)) = \sigma(1\ 2\ 3\ 4\ 5)\sigma^{-1}$
  と書けることから, $A_5$において$(1\ 2\ 3\ 4\ 5)$と共役であることは, この$\sigma$が偶置換であることと同値である.
  一方, $\sigma$が奇置換であるときには,
  $\sigma(1\ 2\ 3\ 4\ 5)\sigma^{-1} = \sigma(1\ 2)(2\ 1\ 3\ 4\ 5)(1\ 2)\sigma^{-1}$
  より,この巡回置換は$A_5$において$(2\ 1\ 3\ 4\ 5)$と共役である.
  したがって$\mathfrak{S}_5$における$(1\ 2\ 3\ 4\ 5)$の共役類は,
  $A_5$において$(1\ 2\ 3\ 4\ 5)$を代表元とする共役類と$(2\ 1\ 3\ 4\ 5)$を代表元とする共役類とに細分される.

  \underline{$(1\ 2\ 3)$について.}
  $A_5$の共役による作用についての$(1\ 2\ 3)$の安定化群を$H$とする.
  明らかに, $\langle(1\ 2\ 3)\rangle \subset H$である.
  逆に$\nu \in H$とすると, $(\nu(1)\ \nu(2)\ \nu(3)) = \nu(1\ 2\ 3)\nu^{-1} = (1\ 2\ 3)$より
  数の組として$(\nu(1), \nu(2), \nu(3))$は$(1, 2, 3)$, $(2, 3, 1)$, $(3, 1, 2)$のいずれかに等しい.
  いずれの場合も, $\nu$を$\{1, 2, 3\}$上に制限したものは, $\langle(1\ 2\ 3)\rangle$に属する.
  また, $\nu$を$\{1, 2, 3\}$上に制限したものは偶置換であるから, 偶置換$\nu$は$\{4, 5\}$上では恒等写像である.
  したがって$\nu \in \langle(1\ 2\ 3)\rangle$であるから, $H \subset \langle(1\ 2\ 3)\rangle$.
  以上により$H = \langle(1\ 2\ 3)\rangle$であることが言えたから,
  $A_5$の共役による作用についての$(1\ 2\ 3)$の軌道 ($(1\ 2\ 3)$の共役類) の濃度は$\abs{A_5}/\abs{H} = \abs{A_5}/\abs{\langle(1\ 2\ 3)\rangle} = 60 / 3 = 20$.
  一方, 長さ$3$の巡回置換の個数も${}_5\mathrm{C}_3 \cdot 2 = 20$であるから,
  長さ$3$の巡回置換はすべて$A_5$において$(1\ 2\ 3)$と共役である.
  すなわち, $(1\ 2\ 3)$を代表元とする$\mathfrak{S}_5$共役類は$A_5$において細分されておらず, $A_5$の共役類としても$(1\ 2\ 3)$がそのまま代表元となる.

  \underline{$(1\ 2)(3\ 4)$について.}
  $\sigma$を$(2, 2)$型の置換とする. すなわち$\sigma = (i\ j)(k\ l)$の形で書ける.
  $\mathfrak{S}_5$における共役として,
  $\tau = \begin{pmatrix*}
    1 & 2 & 3 & 4 & 5 \\
    i & j & k & l & m
  \end{pmatrix*}$とおいて$\tau(1\ 2)(3\ 4)\tau^{-1} = \sigma$という関係があるが,
  もし$\tau$が奇置換ならば$\tau$の代わりに$\tau(1\ 2)$を考えることにより,
  $A_5$においても$\sigma$と$(1\ 2)(3\ 4)$が共役であることが分かる.
  すなわち, $(1\ 2)(3\ 4)$を代表元とする$\mathfrak{S}_5$の共役類は$A_5$において細分されておらず, $A_5$の共役類としても$(1\ 2)(3\ 4)$がそのまま代表元となる.

  \underline{$1$について.} 共役類が$1$元集合なので, そのまま$A_5$の共役類の代表元となっている.
\end{ans}

\begin{ans}
  (1) $\begin{pmatrix*}
    1 & 2 & 3 & 4 & 5 & 6 \\
    4 & 1 & 3 & 2 & 6 & 5
  \end{pmatrix*}$\\
  (2) $\{1, 2, 3\}$を$\{4, 1, 3\}$に移すような$\nu$については,
  $\nu(1)$と$\nu(4)$の値を決めれば$1$つに定まる. したがって$\nu$は$3 \cdot 3 = 9$通りある.
  $\{1, 2, 3\}$を$\{2, 6, 5\}$に移すような$\nu$についても同様. したがって全部で$18$通りである.
\end{ans}

\begin{ans}
  (1) 共役による作用での$(1\ 2)$の軌道 (共役類) は$(2)$型の置換からなり, これは全部で${}_4\mathrm{C}_2 = 6$個あるから,
  $Z_G(\sigma)$の位数, すなわち共役による作用での$(1\ 2)$の安定化群の位数は$\abs{G}/6 = 4$である.
  一方, 明らかに$(1\ 2), (3\ 4) \in Z_G(\sigma)$であるから,
  $\langle (1\ 2), (3\ 4) \rangle \subset Z_G(\sigma)$である.
  $(1\ 2)$と$(3\ 4)$は可換でいずれも位数$2$であって, $\langle (1\ 2), (3\ 4) \rangle$の任意の元は$(1\ 2)^i(3\ 4)^j (i, j = 0, 1)$と書けるので,
  $\abs{\langle (1\ 2), (3\ 4) \rangle} = 4$である.
  したがって$Z_G(\sigma) = \langle (1\ 2), (3\ 4) \rangle$.\\
  (2) 共役による作用での$(1\ 2)(3\ 4)$の軌道は$(2, 2)$型の置換からなり, これは${}_4\mathrm{C}_2/2 = 3$個ある.
  したがって$\abs{Z_G(\sigma)} = \abs{G}/3 = 8$.
  まず, 共役による作用で$(1\ 2)$をそれ自身に移すものとして$1, (1\ 2), (3\ 4), (1\ 2)(3\ 4)$があり,
  $(1\ 2)$を$(3\ 4)$に移すものとして$(1\ 3)(2\ 4), (1\ 4)(2\ 3), (1\ 3\ 2\ 4), (1\ 4\ 2\ 3)$がある.
  以上を合わせて$8$個なので, これらが$Z_G(\sigma)$をなす.
  まとめると$Z_G(\sigma) = \langle (1\ 2), (3\ 4), (1\ 3)(2\ 4) \rangle$.\\
  (3) 共役による作用での$(1\ 2\ 3)$の軌道は$(3)$型の置換からなり, これは${}_4\mathrm{C}_3 \cdot 2 = 8$個ある.
  したがって$\abs{Z_G(\sigma)} = \abs{G}/8 = 3$.
  一方, 明らかに$\langle (1\ 2\ 3) \rangle \subset Z_G(\sigma)$であるが,
  $\abs{\langle (1\ 2\ 3) \rangle} = 3$なので$Z_G(\sigma) = \langle (1\ 2\ 3) \rangle$.\\
  (4) 共役による作用での$(1\ 2\ 3)$の軌道は$(3)$型の置換からなり, これは${}_5\mathrm{C}_3 \cdot 2 = 20$個ある.
  したがって$\abs{Z_G(\sigma)} = \abs{G}/20 = 6$.
  一方, 明らかに$\langle (1\ 2\ 3), (4\ 5) \rangle \subset Z_G(\sigma)$であるが,
  $\abs{\langle (1\ 2\ 3), (4\ 5) \rangle} = 6$なので,
  $Z_G(\sigma) = \langle (1\ 2\ 3), (4\ 5) \rangle$.\\
  (5) 共役による作用での$(1\ 2\ 3)(4\ 5\ 6)$の軌道は$(3, 3)$型の置換からなり, これは${}_6\mathrm{C}_3 \cdot 2 \cdot 2 / 2 = 40$個ある.
  したがって$\abs{Z_G(\sigma)} = \abs{G}/40 = 18$.
  まず, 明らかに$\langle (1\ 2\ 3), (4\ 5\ 6), (1\ 4)(2\ 5)(3\ 6) \rangle \subset Z_G(\sigma)$であり,
  $(1\ 2\ 3)$, $(4\ 5\ 6)$, $(1\ 4)(2\ 5)(3\ 6)$の位数はそれぞれ$3$, $3$, $2$である.
  そこで, $(1\ 2\ 3)^i(4\ 5\ 6)^j((1\ 4)(2\ 5)(3\ 6))^k$という形の元を考えることにする.
  $i, j, i^\prime, j^\prime = 0, 1, 2$, また$k, k^\prime = 0, 1$として,
  \begin{align*}
    (1\ 2\ 3)^i(4\ 5\ 6)^j((1\ 4)(2\ 5)(3\ 6))^k = (1\ 2\ 3)^{i^\prime}(4\ 5\ 6)^{j^\prime}((1\ 4)(2\ 5)(3\ 6))^{k^\prime}
  \end{align*}
  ならば,
  \begin{align*}
    (1\ 2\ 3)^{i-i^\prime}(4\ 5\ 6)^{j-j^\prime} = ((1\ 4)(2\ 5)(3\ 6))^{k^\prime - k}
  \end{align*}
  となるが, 左辺の位数は$3$の約数, 右辺の位数は$2$の約数なので, 両辺は$1$である.
  したがって$(i, j, k) = (i^\prime, j^\prime, k^\prime)$なので,
  $(1\ 2\ 3)^i(4\ 5\ 6)^j((1\ 4)(2\ 5)(3\ 6))^k$の形の元は互いに異なり, $3 \cdot 3 \cdot 2 = 18$個ある.
  よって$Z_G(\sigma) = \langle (1\ 2\ 3), (4\ 5\ 6), (1\ 4)(2\ 5)(3\ 6) \rangle$.\\
  (6) 共役による作用での$(1\ 2)(3\ 4)(5\ 6)$の軌道は$(2, 2, 2)$型の置換からなり, これは$5 \cdot 3 = 15$個ある ($1$の行き先と, 残りの$4$個から$1$つ選んだものの行き先とを決めると考えればよい).
  したがって$\abs{Z_G(\sigma)} = \abs{G}/15 = 48$.
  まず, 明らかに$\langle (1\ 2), (3\ 4), (5\ 6), (1\ 3)(2\ 4), (1\ 3\ 5)(2\ 4\ 6) \rangle \subset Z_G(\sigma)$であり,
  生成元の位数はそれぞれ$2$, $2$, $2$, $2$, $3$である.
  そこで, $(1\ 2)^i(3\ 4)^j(5\ 6)^k((1\ 3)(2\ 4))^l((1\ 3\ 5)(2\ 4\ 6))^m$という形の元を考えることにする.
  $i, j, k, l, i^\prime, j^\prime, k^\prime, l^\prime = 0, 1$, また$m, m^\prime = 0, 1, 2$として,
  \begin{align*}
    (1\ 2)^i&(3\ 4)^j(5\ 6)^k((1\ 3)(2\ 4))^l((1\ 3\ 5)(2\ 4\ 6))^m\\
    = &(1\ 2)^{i^\prime}(3\ 4)^{j^\prime}(5\ 6)^{k^\prime}((1\ 3)(2\ 4))^{l^\prime}((1\ 3\ 5)(2\ 4\ 6))^{m^\prime}
  \end{align*}
  ならば,
  \begin{align*}
    (1\ 2)^{i - i^\prime}(3\ 4)^{j - j^\prime}(5\ 6)^{k - k^\prime}
     = ((1\ 3)(2\ 4))^{l^\prime}((1\ 3\ 5)(2\ 4\ 6))^{m^\prime - m}((1\ 3)(2\ 4))^l
  \end{align*}
  となるが, 両辺の$1$, $3$, $5$の行き先の候補を比較すると, 共通するものはそれぞれ$1$, $3$, $5$しかないので,
  $(i, j, k) = (i^\prime, j^\prime, k^\prime)$である. したがって
  \begin{align*}
    ((1\ 3)(2\ 4))^{l + l^\prime} = ((1\ 3\ 5)(2\ 4\ 6))^{m^\prime - m}
  \end{align*}
  となるが, 左辺の位数は$2$の約数, 右辺の位数は$3$の約数なので, 両辺は$1$である.
  したがって$(l, m) = (l^\prime, m^\prime)$なので, 結局$(i, j, k, l, m) = (i^\prime, j^\prime, k^\prime, l^\prime, m^\prime)$.
  よって$(1\ 2)^i(3\ 4)^j(5\ 6)^k((1\ 3)(2\ 4))^l((1\ 3\ 5)(2\ 4\ 6))^m$という形の元は互いに異なり,
  $2 \cdot 2 \cdot 2 \cdot 2 \cdot 3 = 48$個ある.
  $\abs{Z_G(\sigma)} = 48$かつ
  $\langle (1\ 2), (3\ 4), (5\ 6), (1\ 3)(2\ 4), (1\ 3\ 5)(2\ 4\ 6) \rangle \subset Z_G(\sigma)$であったから,
  $Z_G(\sigma) = \langle (1\ 2), (3\ 4), (5\ 6), (1\ 3)(2\ 4), (1\ 3\ 5)(2\ 4\ 6) \rangle$.
\end{ans}

\end{document}
